\section{Co-ranking for Keyphrase Annotation}
\label{sec:topicrankpp}
    This section presents TopicCoRank\footnote{TopicCoRank is open source and publicly available at \url{https://github.com/adrien-bougouin/KeyBench/tree/coling_2016/}}, our keyphrase annotation method built on the existing method TopicRank~\cite{bougouin2013topicrank} to which we add keyphrase assignment.
    We first detail TopicRank, then present our contributions.
    
    \subsection{TopicRank}
    \label{subsec:topicrank}
        TopicRank is a graph-based keyphrase extraction method that relies on the following five steps:
        \begin{enumerate}
            \item{\textbf{Keyphrase candidate selection.}
                Following previous work~\cite{hassan2010conundrums,wan2008expandrank}, keyphrase candidates are selected from the sequences of adjacent nouns and adjectives that occur within the document (\texttt{/(N|A)+/}).
            }
            \item{\textbf{Topical clustering.}
                Similar keyphrase candidates $c$ are clustered into topics based on the words they share.
                \newcite{bougouin2013topicrank} use a Hierarchical Agglomerative Clustering (HAC) with a stem overlap similarity (see equation~\ref{math:jaccard}) and an average linkage.
                At the beginning, each keyphrase candidate is a single cluster, then candidates sharing an average of $\unitfrac{1}{4}$ stemmed words with the candidates of another cluster are iteratively added to the latter.
                 
                \vspace{-1em}\begin{align}
                    \text{sim}(c_i, c_j) &= \frac{|\text{stems}(c_i) \cap \text{stems}(c_j)|}{|\text{stems}(c_i) \cup \text{stems}(c_j)|} \label{math:jaccard}
                \end{align}
                where $\text{stems}(c_i)$ is the set of stemmed words of the keyphrase candidate $c_i$.
            }
            \item{\textbf{Graph construction.}
                A complete graph is built, in which nodes are topics and edges are weighted according to the strength of the semantic relation between the connected topics.
                The closer are the pairs of candidates $\langle{}c_i, c_j\rangle{}$ of two topics $t_i$ and $t_j$ within the document, the stronger is their semantic relation $w_{i,j}$:
                
                \vspace{-1em}\begin{align}
                    w_{i, j} &= \mathlarger{\sum}_{c_i \in t_i}\ \mathlarger{\sum}_{c_j \in t_j} \text{dist}(c_i, c_j) \label{math:semantic_relatedness}\\
                    \text{dist}(c_i, c_j) &= \sum_{p_i \in \text{pos}(c_i)}\ \sum_{p_j \in \text{pos}(c_j)} \frac{1}{|p_i - p_j|} \label{math:distance}
                \end{align}
                where $\text{pos}(c_i)$ represents all of the offset positions of the first word of the keyphrase candidate $c_i$.
            }
            \item{\textbf{Topic ranking.}
                Topics $t$ are ranked using the importance score $S(t_i)$ of the Text\-Rank formula, as modified by \newcite{wan2008expandrank} to leverage edge weights:
                 
                \vspace{-1em}\begin{align}
                    S(t_i) = (1 - \lambda) + \lambda \sum_{t_j \in E(t_i)}{\frac{w_{ij}S(t_j)}{\mathlarger\sum_{t_k \in E(t_j)}{w_{jk}}}}\label{math:singlerank}
                \end{align}
            }
            \item{\textbf{Keyphrase selection.}
                One keyphrase candidate is selected from each of the $N$ most important topics: the first occurring keyphrase candidate.
            }
        \end{enumerate}
        
    %\TODO{modification of TopicCoRank compared to TopicRank}
    Our work extends TopicRank to assign domain-specific keyphrases that do not necessarily occur within the document.
    First, we add a second graph representing the domain and unify it to the topic graph.
    Second, we define a co-ranking scheme that leverages the new graph.
    Finally, we redefine the keyphrase selection step for both extracting and assigning keyphrases.

    \subsection{Unified graph construction}
    \label{subsec:graph_construction}
        TopicCoRank operates over a unified graph that connects two graphs representing the document topics, the controlled keyphrases and the relations between them (see Fig.~\ref{fig:topicrankpp_graph}).
        The controlled keyphrases are the keyphrases that were manually assigned to training documents.
        Considering the manually assigned keyphrases as the controlled vocabulary circumvents the need for a manually produced controlled vocabulary and also allows us to further take advantage of the semantic relatonship between the domain-specific (controlled) keyphrases.
        Because controlled keyphrases are presumably non-redundant, we do not topically cluster them as we do for keyphrase candidates.
        
        \begin{figure}[htb!]
            \newcommand{\xslant}{0.25}
            \newcommand{\yslant}{0}
        
            \centering
            \begin{tikzpicture}[transform shape, scale=.65]
                % frame
                \node [draw,
                       rectangle,
                       minimum width=.8\linewidth,
                       minimum height=7em,
                       xslant=\xslant,
                       yslant=\yslant] (domain_graph) {};
                \node [above=of domain_graph,
                       xshift=.4\linewidth,
                       yshift=5.5em,
                       anchor=south east] (domain_graph_label) {controlled keyphrases};
            
                \node [draw,
                       circle,
                       above=of domain_graph,
                       xshift=.3\linewidth,
                       yshift=2.5em] (domain_node1) {$k_1$};
                \node [draw,
                       circle,
                       above=of domain_graph,
                       xshift=-.3\linewidth,
                       yshift=2.5em] (domain_node2) {$k_2$};
                \node [draw,
                       circle,
                       above=of domain_graph,
                       yshift=2.5em] (domain_node3) {$k_3$};
                \node [draw,
                       circle,
                       above=of domain_graph,
                       xshift=.15\linewidth,
                       yshift=-.75em] (domain_node4) {$k_4$};
                \node [draw,
                       circle,
                       above=of domain_graph,
                       xshift=-.15\linewidth,
                       yshift=-.75em] (domain_node5) {$k_5$};
            
                \draw (domain_node1) -- (domain_node3);
                \draw (domain_node2) -- (domain_node3);
                \draw (domain_node2) -- (domain_node4);
                \draw (domain_node4) -- (domain_node5);
                \draw (domain_node4) -- (domain_node3);
            
                % document
                \node [draw,
                       rectangle,
                       minimum width=.8\linewidth,
                       minimum height=7em,
                       xslant=\xslant,
                       yslant=\yslant,
                       above=of domain_graph,
                       xshift=-1.5em,
                       yshift=-1.5em] (document_graph) {};
                \node [below=of document_graph,
                       xshift=-.4\linewidth,
                       yshift=-5.56em,
                       anchor=north west] (document_graph_label) {document topics};
            
                \node [draw,
                       circle,
                       below=of document_graph,
                       xshift=.3\linewidth,
                       yshift=-2.5em] (document_node1) {$t_1$};
                \node [draw,
                       circle,
                       below=of document_graph,
                       xshift=-.3\linewidth,
                       yshift=-2.5em] (document_node2) {$t_2$};
                \node [draw,
                       circle,
                       below=of document_graph,
                       yshift=-2.5em] (document_node3) {$t_3$};
                \node [draw,
                       circle,
                       below=of document_graph,
                       xshift=.15\linewidth,
                       yshift=.5em] (document_node4) {$t_4$};
            
                \draw (document_node2) -- (document_node3);
                \draw (document_node3) -- (document_node1);
                \draw (document_node1) -- (document_node4);
                \draw (document_node3) -- (document_node4);
            
                % extra link
                \draw [dashed] (document_node2) -- (domain_node2);
                \draw [dashed] (document_node3) -- (domain_node3);
                \draw [dashed] (document_node4) -- (domain_node1);
                \draw [dashed] (document_node3) -- (domain_node4);
            
                % legend
                \node [right=of document_graph, yshift=-6.5em] (legend_title) {\underline{Legend:}};
                \node [below=of legend_title, xshift=-1em, yshift=2em] (begin_inner) {};
                \node [right=of begin_inner] (end_inner) {: $E_\textnormal{\textit{in}}$};
                \node [below=of begin_inner, yshift=1.5em] (begin_outer) {};
                \node [right=of begin_outer] (end_outer) {: $E_\textnormal{\textit{out}}$};
            
                \draw (legend_title.north  -| end_outer.east) rectangle (end_outer.south -| legend_title.west);
            
                \draw (begin_inner) -- (end_inner);
                \draw [dashed] (begin_outer) -- (end_outer);
            \end{tikzpicture}
            \caption{
                Example of a unified graph constructed by TopicCoRank and its two kinds of edges
                \label{fig:topicrankpp_graph}
            }
        \end{figure}
    
        \REVIEWOK{The notations of the vertices for the controlled keyphrases and topics are very confused.}
        Let $G = (V=T \cup K, E=E_{\textnormal{\textit{in}}} \cup E_{\textnormal{\textit{out}}})$ denote the unified graph.
        Topics $T=\{t_1, t_2, ..., t_n\}$ and controlled keyphrases $K=\{k_1, k_2, ..., k_m\}$ are vertices $V$ connected to their fellows by edges $E_\textnormal{\textit{in}} \subseteq T \times T \cup K \times K$ and connected to the other vertices by edges $E_\textnormal{\textit{out}} \subseteq K \times T$ (see Fig.~\ref{fig:topicrankpp_graph}).

        To unify the two graphs, we consider the controlled keyphrases as a category map and connect the document to its potential categories.
        We create an unweighted edge $\langle{}k_i, t_j\rangle{} \in E_{\textnormal{\textit{out}}}$ to connect a controlled keyphrase $k_i$ and a topic $t_j$ if the controlled keyphrase is a member of the topic, i.e. a keyphrase candidate of the topic\footnote{To accept inflexions, such as plural inflexions, we follow \newcite{bougouin2013topicrank} and perform the comparison with stems.}.
        We create an edge $\langle{}t_i, t_j\rangle{} \in E_\textnormal{\textit{in}}$ or $\langle{}k_i, k_j\rangle{} \in E_\textnormal{\textit{in}}$ between two topics $t_i$ and $t_j$ or two controlled keyphrases $k_i$ and $k_j$ when they co-occur within a sentence of the document or as keyphrases of a training document, respectively.
        Edges $\langle{}t_i, t_j\rangle{} \in E_{\textnormal{\textit{in}}}$ are weighted by the number of times ($w_{i, j}$) topics $t_i$ and $t_j$ occur in the same sentence within the document.
        Edges $\langle{}k_i, k_j\rangle{} \in E_{\textnormal{\textit{in}}}$ are weighted by the number of times ($w_{i, j}$) keyphrases $k_i$ and $k_j$ are associated to the same document among the training documents.
        Doing so, the weighting scheme of edges $E_\textnormal{\textit{in}}$ is equivalent for both topics and controlled keyphrases.
        This equivalence is essential to ensure that not only controlled keyphrases occurring in the document can be assigned by properly co-ranking topics and controlled keyphrases.

    \subsection{Graph-based co-ranking}
    \label{subsec:graph_based_co_ranking}
        TopicCoRank gives an importance score $S(t_i)$ or $S(k_i)$ to every topic or controlled keyphrase using graph co-ranking (see equations~\ref{math:topiccorank_t} and~\ref{math:topiccorank_k}).
%        Graph co-ranking has been applied with success to many Natural Language Processing tasks  such as
%        %text summarization~\cite{wan2011corankingsummarization},
%        tweet recommendation~\cite{yan2012corankingtweetrecommendation} or 
%        opinion mining~\cite{liu2014corankingopinionmining}.
        %
        Our graph co-ranking simulates the voting concept based on inner and outer recommendations.
        
        The inner recommendation is similar to the recommendation computed in previous work~\cite{bougouin2013topicrank,mihalcea2004textrank,wan2008expandrank}.
        The inner recommendation $R_\textnormal{\textit{in}}$ comes from nodes of the same graph (see equation~\ref{math:rin}). %:
        A topic or a controlled keyphrase is important if it is strongly connected to other topics or controlled keyphrases, respectively.
        %\begin{itemize}
        %  \item{a topic is important if it is strongly connected to other topics;}
        %  \item{a controlled keyphrase is important if it is strongly connected to other controlled keyphrases.}
        %\end{itemize}
        
        The outer recommendation influences the ranking of topics by controlled keyphrases and of controlled keyphrases by topics.
        The outer recommendation $R_\textnormal{\textit{out}}$ comes from nodes of the other graph (see equation~\ref{math:rout}). %:
        A topic or a controlled keyphrase gain more importance if it is connected to important controlled keyphrases or an important topic, respectively.
        
        \begin{align}
          S(t_i) &= (1 - \lambda_t)\ R_{out}(t_i) + \lambda_t\ R_{in}(t_i)\label{math:topiccorank_t}\\[1em]
          S(k_i) &= (1 - \lambda_k)\ R_{out}(k_i) + \lambda_k\ R_{in}(k_i)\label{math:topiccorank_k}\\[1em]
          R_{in}(v_i) &= \sum_{v_j \in E_{\text{in}}(v_i)}{\frac{w_{ij} S(v_j)}{\mathlarger\sum_{v_k \in E_{\text{in}}(v_j)}{{w_{jk}}}}}\label{math:rin}\\[1em]
          R_{out}(v_i) &= \sum_{v_j \in E_{\text{out}}(v_i)}{\frac{S(v_j)}{|E_{\text{\textit{out}}}(v_j)|}}\label{math:rout}
        \end{align}
        
        
        \noindent where $v_i$ is a node representing a keyphrase or a topic.
        $\lambda_t$ and $\lambda_k$ are parameters that control the influence of the inner recommendation over the outer recommendation ($0~\leq~\lambda_t~\leq~1$ and $0~\leq~\lambda_k~\leq~1$) for the topics and the controlled keyphrases, respectively.
        %A higher $\lambda_t$ or $\lambda_k$ gives more influence to the inner recommendation.

    \subsection{Keyphrase annotation}
    \label{subsec:keyphrase_assignment_and_extraction}
        Keyphrases are extracted and assigned from the N-best ranked topics and controlled keyphrases, regardless of their nature.
        
        We extract topic keyphrases using the former TopicRank strategy. Only one keyphrase is extracted per topic: the keyphrase candidate that first occurs within the document.
        
        We assign controlled keyphrases only if they are directly or transitively connected to a topic of the document. If the ranking of a controlled keyphrase has not been affected by a topic of the document nor by controlled keyphrases connected to topics, then its importance score is not related to the content of the document and it should not be assigned.


        At this step, two variants of TopicCoRank performing either extraction or assignment can be proposed, namely TopicCoRank$_\textit{extr}$ and TopicCoRank$_\textit{assign}$.
        If keyphrases are only extracted from the topics, we obtain TopicCoRank$_\textit{extr}$.
        If keyphrases are only assigned from the controlled keyphrases, we obtain TopicCoRank$_\textit{assign}$.
