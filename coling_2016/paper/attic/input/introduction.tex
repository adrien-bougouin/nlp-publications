\section{Introduction}
\label{sec: introduction}
  Keyphrases are words or phrases that give a synoptic picture of what is
  important within a document. They are useful in many tasks such as
  document indexing~\cite{gutwin1999keyphrasesfordigitallibraries}, text
  categorization~\cite{hulth-megyesi:2006:COLACL} or
  summarization~\cite{litvak2008graphbased}. However, most documents do not
  provide keyphrases, and the daily flow of new documents makes the manual
  keyphrase annotation impractical. As a consequence, automatic keyphrase
  annotation has received special attention in the Natural Language
  Processing (NLP) community and many different methods have been 
  proposed~\cite{hasan2014state_of_the_art}.


  The task of automatic keyphrase annotation consists in identifying the main 
  concepts, or topics, addressed in a document.
  Keyphrase annotation methods fall into two broad categories: keyphrase 
  extraction and keyphrase assignment methods.
  Keyphrase extraction methods extract the most important words or phrases
  occurring in a document, while assignment methods provide controlled 
  keyphrases from a domain-specific terminology (controlled vocabulary).

  The automatic keyphrase annotation task is often reduced to the sole 
  keyphrase extraction task.
  Unlike assignment methods, extraction methods do not require controlled 
  vocabularies that are costly to create and to maintain. 
  Furthermore, they are able to identify new concepts.
  However, extraction methods often output ill-formed or inappropriate
  keyphrases~\cite{medelyan2008smalltrainingset}, and they produce only
  keyphrases that actually occur in the document.
  %Our intuition is that keyphrase annotation must output appropriate
  %keyphrases, i.e., mostly from controlled vocabulary, and extract
  %keyphrases that represent new concepts, i.e., within the document.

  
  Observations made on manually annotated keyphrases show that human annotators both extract keyphrases from the content of the document and
  assign keyphrases based on their knowledge of the domain~\cite{liu2011vocabularygap}. Here, we
  propose an approach that mimics this behavior and jointly extracts and assigns keyphrases.
  We use two graph representations, one for the document and one for the domain, and apply a co-ranking algorithm to perform both keyphrase extraction and assignment in a mutual reinforcing manner.
  %
  %We automatically allow this behavior by
  %unifying a graph representing reference keyphrases of a document's specific
  %domain to another graph representing keyphrase candidates selected from its
  %content. The first graph contains potential keyphrases that can be assigned,
  %while the second graph contains the potential keyphrases that can be extracted.
  %Then, we apply a co-ranking algorithm to infer the most important keyphrases.
  %
  %This paper presents a new keyphrase annotation method that uses a
  %unified-graph to represent the context of a document and its specific
  %domain, then applies a co-ranking algorithm to extract and assign the
  %keyphrases in a collaborative manner.
  
  Along with this approach come two contributions.
  First, we propose a reproducible extension to graph-based keyphrase
  extraction methods for assigning keyphrases. Second, we circumvent the need of controlled vocabulary by leveraging the reference keyphrases from training
  data.
%
%  In this work, we present a graph co-ranking approach to keyphrase 
%  annotation that simultaneously performs keyphrase assignment and 
%  extraction in a mutually reinforcing manner.
%  We rely on documents annotated with keyphrases to build a co-occurrence 
%  graph representation of the controlled vocabulary.
%  A second co-occurrence graph is built from the document and connected 
%  to the latter.
%  Keyphrase candidates extracted from the document and entries 
%  from the controlled vocabulary are then simultaneously ranked.
%  \ANNOTATE{Our assumption is that there is a mutually reinforcing relationship between
%  the ranking of the keyphrases in the document and the ranking of
%  the entries in the controlled vocabulary.}{More details to show our originality}

  We perform experiments on four datasets of different languages and domains.
  Results show that our co-ranking approach significantly improves performance over state-of-the-art approaches.

  The rest of this paper is organized as follows. Section~\ref{sec:related_work}
  presents the previous work on automatic keyphrase annotation,
  section~\ref{sec:topicrankpp} describes our method,
  section~\ref{sec:experimental_settings} presents our experimental settings and
  section~\ref{sec:results} discusses the results.
  Finally, section~\ref{sec:conclusion} concludes and shows some directions 
  for future work.
