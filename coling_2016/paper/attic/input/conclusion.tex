\section{Conclusion and Future Work}
\label{sec:conclusion}
  In this paper, we have proposed a co-ranking approach to performing keyphrase
  extraction and keyphrase assignment jointly. Our method,
  TopicCoRank, builds two graphs: one with the
  document topics and one with controlled keyphrases (training keyphrases). We
  design a strategy to unify the two graphs and rank by importance topics and
  controlled keyphrases using a co-ranking vote.
  
  We performed experiments on four datasets of different languages and
  domains. Results showed that our approach benefits from both controlled
  keyphrases and document topics, improving both keyphrase extraction and keyphrase assignment
  baselines.
  %
  TopicCoRank can be used to annotate keyphrases in the way of professional
  indexers.
  %TopicCoRank can also be adapted to more specific problems. For instance,
  %in some cases it might be interesting to assign very specific keyphrases to the
  %document, whereas in other cases it may be interesting to assign more general keyphrases. The
  %minimum depth between a controlled keyphrase and a document topic within the unified graph encodes this
  %degree of generalization. A controlled keyphrase directly connected to a
  %document topic has the lowest generalization regarding the document,
  %followed by its connected controlled keyphrases, etc.
  %
  %In future work, we plan to investigate unsupervised variants of TopicCoRank.
  %Indeed, Wikipedia or Freebase data could be used to replace training keyphrases as controlled keyphrases.

  In future work, we will explore unsupervised constructions of the domain
  graph (using Wikipedia data, bootstrapping techniques, etc.) and explore
  more complex unification strategies in order to more accurately connect the
  document and the domain graph.
