%\section{Introduction}
  \begin{frame}{Introduction}
    \framesubtitle{Problem statement}

    \begin{block}<1->{Keyphrases}
      \begin{itemize}
        \item{Word or multi-word expressions}
        \item{\textbf{Overview} of a document's content}
      \end{itemize}
    \end{block}

    \uncover<2->{
      \begin{block}{Applications}
        \begin{multicols}{2}
          \begin{itemize}
            \item{Document indexing}
            \item{Document clustering}
            \item{Text summarization}
            \item{Query expansion}
            \item{Targeted advertising}
            \item{etc.}
          \end{itemize}
        \end{multicols}
      \end{block}
    }

    \uncover<3->{
      \begin{alertblock}{Lack of annotated documents}
        Many documents have \textbf{no associated keyphrases}.
      \end{alertblock}
    }
  \end{frame}

  \begin{frame}{Introduction}
    \framesubtitle{Automatic keyphrase extraction}

    \tikzstyle{io}=[
      ellipse,
      minimum width=5cm,
      minimum height=2cm,
      fill=green!20,
      draw=green!33,
      transform shape,
      font={\Large\bfseries}
    ]
    \tikzstyle{component}=[
      text centered,
      thick,
      rectangle,
      minimum width=10cm,
      minimum height=2.5cm,
      fill=cyan!20,
      draw=cyan!33,
      transform shape,
      font={\Large\bfseries}
    ]

    \begin{center}
      \begin{tikzpicture}[thin,
                          align=center,
                          scale=.4,
                          node distance=2cm,
                          every node/.style={text centered, transform shape}]
        \node[io] (document) {document};
        \node[component] (preprocessing) [right=of document] {Linguistic Preprocessing};
        \node[component] (candidate_extraction) [below=of preprocessing] {Candidate Extraction};
        \node[component] (candidate_classification_and_ranking) [below=of candidate_extraction] {
          \begin{tabular}{r|lcl}
            Candidate & Classification\\
                      & Ranking
          \end{tabular}
        };
        \node[component] (keyphrase_selection) [below=of candidate_classification_and_ranking] {Keyphrase Selection};
        \node[io] (keyphrases) [right=of keyphrase_selection] {keyphrases};

        \path[->, thick] (document) edge (preprocessing);
        \path[->, thick] (preprocessing) edge (candidate_extraction);
        \path[->, thick] (candidate_extraction) edge (candidate_classification_and_ranking);
        \path[->, thick] (candidate_classification_and_ranking) edge (keyphrase_selection);
        \path[->, thick] (keyphrase_selection) edge (keyphrases);

        \visible<2->{
          \coordinate[xshift=-1em, yshift=1em] (classification) at (candidate_classification_and_ranking.east);
          \coordinate[xshift=5em, yshift=5em] (supervised_coordinates) at (candidate_classification_and_ranking.east);
          \coordinate[xshift=-4.5em, yshift=-1em] (ranking) at (candidate_classification_and_ranking.east);
          \coordinate[xshift=5em, yshift=-5em] (unsupervised_coordinates) at (candidate_classification_and_ranking.east);

          \node (supervised) at (supervised_coordinates) {\Large\textbf{supervised}};
          \node (unsupervised) at (unsupervised_coordinates) {\Large\textbf{unsupervised}};

          \path[->, red, thick] (supervised) edge (classification);
          \path[->, red, thick] (unsupervised) edge (ranking);
          \visible<3->{
            \node[draw=red, ellipse, thick] (unsupervised) at (unsupervised_coordinates) {\Large\textbf{unsupervised}};
          }
        }
      \end{tikzpicture}
    \end{center}
  \end{frame}

  \begin{frame}{Introduction}
    \framesubtitle{Example}

    \begin{exampleblock}{\small \textbf{Project Euclid} and the role of
                         \textbf{research libraries} in \textbf{scholarly
                         publishing}}
      \justifying
      \small \textbf{Project Euclid}, a \textbf{joint electronic journal
      publishing initiative} of \textbf{Cornell University Library} and
      \textbf{Duke University Press} is discussed in the broader contexts of the
      changing patterns of \textbf{scholarly communication} and the publishing
      scene of \textbf{mathematics}. Specific aspects of the project such as
      \textbf{partnerships} and the creation of an \textbf{economic model} are
      presented as well as what it takes to be a publisher. Libraries have
      gained important and relevant experience through the creation and
      management of digital libraries, but they need to develop further skills
      if they want to adopt a new role in the life cycle of \textbf{scholarly
      communication}.
    \end{exampleblock}
  \end{frame}

