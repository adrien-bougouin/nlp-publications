\section{TopicRank}
\begin{frame}[label=topicrank]{TopicRank}
    \begin{columns}
      \begin{column}{.45\textwidth}
        \begin{enumerate}
          \item<2->{Candidate extraction}
          \alt<2>{
            \item[$\Rightarrow$]{\small \hyperlink{candidate_extraction_backup}{$(\text{NOUN} | \text{ADJ})^+$}}
            \item<3->{Candidate clustering}
          }{
            \item<3->{Candidate clustering}
          }\alt<3>{
            \item[$\Rightarrow$]{\small \hyperlink{candidate_clustering_backup}{Hierarchical clustering}}
            \item<4->{Graph construction}
          }{
            \item<4->{Graph construction}
          }\alt<4>{
            \item[$\Rightarrow$]{\small \hyperlink{graph_construction_backup}{Complete graph}}
            \item<5->{Topic ranking}
          }{
            \item<5->{Topic ranking}
          }\alt<5>{
          \item[$\Rightarrow$]{\small \hyperlink{topic_ranking_backup}{PageRank's random walk}}
            \item<6->{Keyphrase selection}
          }{
            \item<6->{Keyphrase selection}
            \alt<2->{
            }{
              \item[]{}
            }
          }\alt<6->{
            \item[$\Rightarrow$]{\small First appearing candidate}
            \item[]{}
          }{
            \item[]{}
          }
        \end{enumerate}

        \visible<7>{
          \begin{textblock*}{.95\textwidth}(.025\textwidth, -.7\textheight)
            \setbeamertemplate{blocks}[rounded][shadow=true]

            \begin{exampleblock}{\footnotesize Project Euclid and the role of
                                 \textcolor{red}{research libraries} in
                                 scholarly publishing}
              \justifying
              \footnotesize Project Euclid, a joint electronic journal
              publishing initiative of \textcolor{BurntOrange}{Cornell University
              Library} and Duke University Press is discussed in the broader
              contexts of the changing patterns of scholarly communication and
              the publishing scene of mathematics. [\dots]
              \textcolor{BurntOrange}{Libraries} have gained important and relevant
              experience through the creation and management of
              \textcolor{BurntOrange}{digital libraries}, but they need to develop
              further skills if they want to adopt a new role in the life cycle
              of scholarly communication.
            \end{exampleblock}
          \end{textblock*}
        }

        \visible<9>{
          \begin{textblock*}{.95\textwidth}(.025\textwidth, -.65\textheight)
            \setbeamertemplate{blocks}[rounded][shadow=true]

            \begin{exampleblock}{\footnotesize Project Euclid and the role of
                                 research libraries in
                                 \textcolor{red}{scholarly publishing}}
              \justifying
              \footnotesize Project Euclid, a joint electronic journal
              publishing initiative of Cornell University Library and Duke
              University Press is discussed in the broader contexts of the
              changing patterns of scholarly communication and the
              \textcolor{BurntOrange}{publishing scene} of mathematics. Specific
              aspects of the project such as partnerships and the creation of
              an economic model are presented as well as what it takes to be a
              \textcolor{BurntOrange}{publisher}. [\dots]
            \end{exampleblock}
          \end{textblock*}
        }

        \visible<11>{
          \begin{textblock*}{.95\textwidth}(.025\textwidth, -.52\textheight)
            \setbeamertemplate{blocks}[rounded][shadow=true]

            \begin{exampleblock}{\footnotesize \textcolor{red}{Project
                                 Euclid} and the role of research libraries in
                                 scholarly publishing}
              \justifying
              \footnotesize [\dots] Specific aspects of the
              \textcolor{BurntOrange}{project such} as partnerships and the
              creation of an economic model are presented as well as what it
              takes to be a publisher. [\dots]
            \end{exampleblock}
          \end{textblock*}
        }

        \visible<13>{
          \begin{textblock*}{.95\textwidth}(.025\textwidth, -.55\textheight)
            \setbeamertemplate{blocks}[rounded][shadow=true]

            \begin{exampleblock}{\footnotesize Project Euclid and the
                                 \textcolor{red}{role} of research
                                 libraries in scholarly publishing}
              \justifying
              \footnotesize [\dots] Libraries have gained important and relevant
              experience through the creation and management of digital
              libraries, but they need to develop further skills if they want to
              adopt a \textcolor{BurntOrange}{new role} in the life cycle of
              scholarly communication.
            \end{exampleblock}
          \end{textblock*}
        }
      \end{column}

      \begin{column}{.55\textwidth}
        \minipage[c][0.8\textheight]{\columnwidth}
          \alt<2->{ % next slides
            \alt<3->{ % next slides
              \alt<4->{ %next slides
                \alt<6->{ % next slides
                  \alt<16->{ % next slides
                    \alt<17->{ % next slides
                      \resizebox{\linewidth}{!}{
                        \begin{tabular}{l}
                          \toprule
                          Keyphrase\\
                          \midrule
                          \cellcolor{pink}{research libraries}\\
                          \cellcolor{pink}{scholarly publishing}\\
                          \cellcolor{pink}{project euclid}\\
                          role\\
                          creation\\
                          \cellcolor{pink}{scholarly communication}\\
                          \cellcolor{pink}{mathematics}\\
                          specific aspects\\
                          \cellcolor{pink}{joint electronic journal publishing initiative}\\
                          \cellcolor{pink}{partnerships}\\
                          \dots\\
                          \bottomrule
                        \end{tabular}
                      }
                    }{ % current slide
                      \resizebox{\linewidth}{!}{
                        \begin{tabular}{l}
                          \toprule
                          Keyphrase\\
                          \midrule
                          research libraries\\
                          scholarly publishing\\
                          project euclid\\
                          role\\
                          creation\\
                          scholarly communication\\
                          mathematics\\
                          specific aspects\\
                          joint electronic journal publishing initiative\\
                          partnerships\\
                          \dots\\
                          \bottomrule
                        \end{tabular}
                      }
                    }
                  }{ % current slide
                    \resizebox{\linewidth}{!}{
                      \begin{tabular}{ccl}
                        \toprule
                        Rank & ID & Topic\\
                        \midrule
                        01 & \multirow{1}{*}{C01} & cornell university library; digital libraries;\\
                        & & \alt<8->{\textcolor{red}{research libraries}}{research libraries}; libraries\\
                        02 & \multirow{1}{*}{C03} & publishing scene; \alt<10->{\textcolor{red}{scholarly publishing}}{scholarly publishing};\\
                        & & publisher\\
                        03 & C02 & \alt<12->{\textcolor{red}{project euclid}}{project euclid}; project such\\
                        04 & C04 & \alt<14->{\textcolor{red}{role}}{role}; new role\\
                        05 & C16 & \alt<15->{\textcolor{red}{creation}}{creation}\\
                        06 & C06 & \alt<15->{\textcolor{red}{scholarly communication}}{scholarly communication}\\
                        07 & C09 & \alt<15->{\textcolor{red}{mathematics}}{mathematics}\\
                        08 & C12 & \alt<15->{\textcolor{red}{specific aspects}}{specific aspects}\\
                        09 & C10 & \alt<15->{\textcolor{red}{joint electronic journal publishing initiative}}{joint electronic journal publishing initiative}\\
                        10 & C08 & \alt<15->{\textcolor{red}{partnerships}}{partnerships}\\
                        \dots & \dots\\
                        \bottomrule
                      \end{tabular}
                    }
                  }
                }{ % current slide
                  \begin{tikzpicture}[scale=.185,
                                      align=center,
                                      every node/.style={transform shape},
                                      main node/.style={text centered,
                                                        circle,
                                                        draw=JungleGreen,
                                                        fill=JungleGreen!20,
                                                        inner sep=1.5pt,
                                                        font=\Large\bfseries}]
                    \foreach \number/\pos in {1/above,2/above,3/above,4/above,5/above,6/above,7/above,8/above,9/above,10/above,11/below,12/below,13/below,14/below,15/below,16/below,17/below,18/below,19/below}{
                      \mycount=\number
                      \advance\mycount by -1
                      \multiply\mycount by 19
                      \advance\mycount by 0
                      \ifthenelse{\number > 9}{
                        \node[main node, label={\pos:\Large\textbf{\visible<5->{\textcolor{red}{\pagerank{\number}}}}}] (N-\number) at (\the\mycount:15cm) {C\number};
                      }{
                        \node[main node, label={\pos:\Large\textbf{\visible<5->{\textcolor{red}{\pagerank{\number}}}}}] (N-\number) at (\the\mycount:15cm) {C0\number};
                      }
                    }
                    \foreach \number in {1,...,18}{
                      \mycount=\number
                      \advance\mycount by 1
                      \foreach \numbera in {\the\mycount,...,19}{
                        \path[JungleGreen!50] (N-\number) edge (N-\numbera);
                      }
                    }
                  \visible<4>{\path[red] (N-11) edge node [left] {\Huge\textbf{offset}\\\Huge\textbf{position}\\\Huge\textbf{weighting}} (N-12);}
                    \foreach \number/\pos in {1/above,2/above,3/above,4/above,5/above,6/above,7/above,8/above,9/above,10/above,11/below,12/below,13/below,14/below,15/below,16/below,17/below,18/below,19/below}{
                      \mycount=\number
                      \advance\mycount by -1
                      \multiply\mycount by 19
                      \advance\mycount by 0
                      \ifthenelse{\number > 9}{
                        \node[main node, label={\pos:\Large\textbf{\visible<5->{\textcolor{red}{\pagerank{\number}}}}}] (N-\number) at (\the\mycount:15cm) {C\number};
                      }{
                        \node[main node, label={\pos:\Large\textbf{\visible<5->{\textcolor{red}{\pagerank{\number}}}}}] (N-\number) at (\the\mycount:15cm) {C0\number};
                      }
                    }
                  \end{tikzpicture}
                }
              }{ % current slide
                \resizebox{\linewidth}{!}{
                  \begin{tabular}{cl}
                    \toprule
                    ID & Topic\\
                    \midrule
                    \multirow{1}{*}{C01} & cornell university library; digital libraries;\\
                    & research libraries; libraries\\
                    C02 & project euclid; project such\\
                    \multirow{1}{*}{C03} & publishing scene; scholarly publishing;\\
                    & publisher\\
                    C04 & role; new role \textcolor{red}{$\longleftarrow \text{stem overlap} \geq \frac{1}{4}$}\\
                    C05 & important\\
                    C06 & scholarly communication\\
                    C07 & further skills\\
                    C08 & partnerships\\
                    C09 & mathematics\\
                    C10 & joint electronic journal publishing initiative\\
                    C11 & contexts\\
                    C12 & specific aspects\\
                    C13 & economic model\\
                    C14 & duke university press\\
                    C15 & relevant experience\\
                    C16 & creation\\
                    C17 & life cycle\\
                    C18 & patterns\\
                    C19 & management\\
                    \bottomrule
                  \end{tabular}
                }
              }
            }{ % current slide
              \begin{exampleblock}{\footnotesize \textcolor{red}{Project Euclid}
                                   and the \textcolor{red}{role} of
                                   \textcolor{red}{research libraries} in
                                   \textcolor{red}{scholarly publishing}}
                \justifying
                \footnotesize \textcolor{red}{Project Euclid}, a
                \textcolor{red}{joint electronic journal publishing initiative}
                of \textcolor{red}{Cornell University Library} and
                \textcolor{red}{Duke University Press} is discussed in the
                broader \textcolor{red}{contexts} of the changing
                \textcolor{red}{patterns} of \textcolor{red}{scholarly
                communication} and the \textcolor{red}{publishing scene} of
                \textcolor{red}{mathematics}. \textcolor{red}{Specific aspects}
                of the \textcolor{red}{project such} as
                \textcolor{red}{partnerships} and the \textcolor{red}{creation}
                of an \textcolor{red}{economic model} are presented as well as
                what it takes to be a \textcolor{red}{publisher}.
                \textcolor{red}{Libraries} have gained
                \textcolor{red}{important} and \textcolor{red}{relevant
                experience} through the \textcolor{red}{creation} and
                \textcolor{red}{management} of \textcolor{red}{digital
                libraries}, but they need to develop \textcolor{red}{further
                skills} if they want to adopt a \textcolor{red}{new role} in the
                \textcolor{red}{life cycle} of \textcolor{red}{scholarly
                communication}.
              \end{exampleblock}
            }
          }{ % current slide
            \begin{exampleblock}{\footnotesize Project Euclid and the role of
                                 research libraries in scholarly publishing}
              \justifying
              \footnotesize Project Euclid, a joint electronic journal
              publishing initiative of Cornell University Library and Duke
              University Press is discussed in the broader contexts of the
              changing patterns of scholarly communication and the publishing
              scene of mathematics. Specific aspects of the project such as
              partnerships and the creation of an economic model are presented
              as well as what it takes to be a publisher. Libraries have gained
              important and relevant experience through the creation and
              management of digital libraries, but they need to develop further
              skills if they want to adopt a new role in the life cycle of
              scholarly communication.
            \end{exampleblock}
          }
        \endminipage
      \end{column}
    \end{columns}
  \end{frame}

