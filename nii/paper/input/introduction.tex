\section{Introduction}
\label{sec:introduction}
  \TODO{check}
  Since the last decade, the amount of information available on the web is
  constantly increasing. While the number of documents continues to grow, the
  need for efficient information retrieval methods becomes increasingly
  important. One way to improve retrieval effectiveness is to use
  keyphrases~\cite{jones1999phrasier}. Keyphrases are single or multi-word
  expressions that represent the main content of a document. As they describe
  the key topics in documents, keyphrases are also useful for tasks such as
  summarization~\cite{avanzo2005keyphrase} or document
  indexing~\cite{medelyan2008smalltrainingset}. There is, however, only a small
  number of documents that have keyphrases associated with them. Keyphrase
  extraction has then attracted a lot of attention recently and many different
  approaches were proposed~\cite{hasan2014state_of_the_art}.

  \TODO{check}
  Generally speaking, keyphrase extraction methods can be categorized into two
  main categories: supervised and unsupervised approaches. Supervised approaches
  treat keyphrase extraction as a binary classification task, where each phrase
  is labeled either as ``keyphrase'' or ``non-keyphrase'',
  e.g.~\cite{witten1999kea}. Conversely, unsupervised approaches usually rank
  phrases by importance and select the top-ranked ones as keyphrases,
  e.g.~\cite{mihalcea2004textrank}.
  \TODO{Bennefits of both supervised and unsupervised approaches}

  \TODO{Contributions}

  The rest of this paper is organized as follows. Section~\ref{sec:background}
  gives an overview of previous work on keyphrase extraction, Section~\ref{sec:}
  describes our new approach, Section~\ref{sec:experimental_settings} details
  our experimentations and Section~\ref{sec:results} presents the evaluation
  results. Finally, we briefly analyse the outputs of our method in
  Section~\ref{sec:error_analysis} and conclude our work in
  Section~\ref{sec:conclusion_and_futur_work}.

