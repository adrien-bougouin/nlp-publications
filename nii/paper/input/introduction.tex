\section{Introduction}
\label{sec:introduction}
  Since the last decade, the amount of information available on the web is
  constantly increasing. While the number of documents continues to grow, the
  need for efficient information retrieval methods becomes increasingly
  important. One way to improve retrieval effectiveness is to use
  keyphrases~\cite{jones1999phrasier}. Keyphrases are single or multi-word
  expressions that represent the main content of a document. However,
  only a small number of documents have keyphrase metadata. Keyphrase extraction
  has then attracted a lot of attention recently and many different approaches
  were proposed~\cite{hasan2014state_of_the_art}.

  Generally speaking, keyphrase extraction methods can be categorized into two
  main categories: supervised and unsupervised approaches. Supervised approaches
  treat keyphrase extraction as a binary classification task, where each phrase
  is labeled either as ``keyphrase'' or ``non-keyphrase''. Many supervised
  methods have been proposed, applying classifiers, such as Naive Bayes
  classifiers~\cite{witten1999kea}, SVMs~\cite{zhang2006svm} and multilayer
  perceptrons~\cite{sarkar2010neuralnetwork}, with various features, such as the
  first position~\cite{witten1999kea}, document
  sections~\cite{nguyen2007keadocumentstructure} and known keyphrase
  distributions~\cite{frank1999keafrequency}. Conversely, unsupervised
  approaches usually rank phrases by importance and select the top-ranked ones
  as keyphrases. Unsupervised approaches proposed so far have involved a number
  of techniques including clustering~\cite{liu2009keycluster}, graph-based
  ranking~\cite{mihalcea2004textrank} and even
  both~\cite{bougouin2013topicrank}.

  In the current state of the keyphrase extraction task, supervised methods
  outperform unsupervised methods. Supervised methods take advantage of features
  extracted from training documents paired with reference keyphrases to identify
  among candidate keyphrases the ones most likely to be keyphrases. In
  opposition, most unsupervised methods only rely on the document to analyse.
  They do not extract the candidates most likely to be keyphrases in a training
  context. Instead, unsupervised methods generally extract the most important
  candidates compared to the others in the local context, the document.

  In order to extract the most likely keyphrases without neglecting their actual
  importance within the analysed document, we present a work that combines both
  supervised and unsupervised visions of the keyphrase extraction task. Using
  the unsupervised keyphrase extraction method
  Topic\-Rank~\cite{bougouin2013topicrank}, we group candidate keyphrases that
  represent the same topic, determine the most important topics and apply
  machine learning to extract the most likely keyphrase of each important topic.
  Our experiments show that a supervised method can benefit from features
  related to topical clusters. Our method significantly outperforms TopicRank
  and the most popular supervised method, KEA.

%  The rest of this paper is organized as follows. Section~\ref{sec:topicrank}
%  presents TopicRank,
%  Section~\ref{sec:supervised_keyphrase_selection_from_topical_clusters}
%  describes our new approach, Section~\ref{sec:experimental_settings} details
%  our experimentations and Section~\ref{sec:results} presents the evaluation
%  results. Finally, we conclude and discuss about future work in
%  Section~\ref{sec:conclusion_and_future_work}.

