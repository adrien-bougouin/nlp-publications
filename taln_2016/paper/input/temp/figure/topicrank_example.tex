\begin{figure} % 44960
  \centering
  \framebox[\linewidth]{
    \parbox{.99\linewidth}{\footnotesize
      \textbf{Météo du 19 août 2012~: alerte à la canicule sur la Belgique et le
      Luxembourg}\\

      A l'exception de la province de Luxembourg, en alerte jaune,
      l'ensemble de la Belgique est en vigilance orange à la canicule. Le
      Luxembourg n'est pas épargné par la vague du chaleur : le nord du pays
      est en alerte orange, tandis que le sud a était placé en alerte rouge.

      ~~~~En Belgique, la température n'est pas descendue en dessous des 23°C
      cette nuit, ce qui constitue la deuxième nuit la plus chaude jamais
      enregistrée dans le royaume. Il se pourrait que ce dimanche soit la
      journée la plus chaude de l'année. Les températures seront comprises
      entre 33 et 38°C. Une légère brise de côte pourra faiblement
      rafraichir l'atmosphère. Des orages de chaleur sont a prévoir dans la
      soirée et en début de nuit.

      ~~~~Au Luxembourg, le mercure devrait atteindre 32°C ce dimanche sur
      l'Oesling et jusqu'à 36°C sur le sud du pays, et 31 à 32°C lundi. Une
      baisse devrait intervenir pour le reste de la semaine. Néanmoins, le
      record d'août 2003 (37,9°C) ne devrait pas être atteint.

      ~\\\textbf{Termes-clés de référence~:} luxembourg~; alerte~; météo~;
      belgique~; août 2012~; chaleur~; température~; chaude~; canicule~;
      orange~; la plus chaude.

    }
  }~\\

  \vspace{1.5em}

  \begin{overpic}[width=.7\linewidth]{include/44960_topicrank_empty.eps}
    \put(38,100.5){\small[soirée]}
    \put(22,96){\small[oesling]}
    \put(55,100){\small[nord]}
    \put(69,95){\small[belgique]}
    \put(0,86){\small[août 2003~; août 2012]}
    \put(83,83.5){\small[36\degre{}c]}
    \put(0,73){\small[37,9\degre{}c]}
    \put(89,70.5){\small[royaume]}
    \put(-5,57.5){\small[année]}
    \put(92.5,53.5){\small[dimanche]}
    \put(-11,41){\small[vigilance orange]}
    \put(88.5,38){\small[luxembourg]}
    \put(1.5,25){\small[début]}
    \put(77,23){\small[nuit~; deuxième nuit]}
    \put(11,13.5){\small[exception]}
    \put(72,10.5){\small[canicule]}
    \put(25,5){\small[journée]}
    \put(60,4.5){\small[côte]}
    \put(17,1){\small[alerte rouge~; alerte jaune~; alerte orange~; alerte]}
    \put(40,84){\small[ensemble]}
    \put(24,80){\small[légère brise]}
    \put(58,82.5){\small[record]}
    \put(15.5,68.5){\small[météo]}
    \put(69.5,72){\small[orages]}
    \put(11,53){\small[chaude]}
    \put(79,58){\small[baisse]}
    \put(10,37){\small[température]}
    \put(72,42){\small[atmosphère]}
    \put(26,26){\small[23\degre{}c]}
    \put(71,27){\small[sud]}
    \put(39,17){\small[lundi]}
    \put(55.5,20.5){\small[38\degre{}c]}
    \put(31,63){\small[chaleur]}
    \put(49.5,68){\small[reste]}
    \put(27,46){\small[province]}
    \put(62,56){\small[vague]}
    \put(41,34.5){\small[pays]}
    \put(56,39){\small[mercure]}
    \put(43,51){\small[semaine]}
  \end{overpic}~\\

  \vspace{1em}

  \framebox[\linewidth]{
    \parbox{.99\linewidth}{\footnotesize
      \textbf{Sortie de TopicRank~:} \underline{luxembourg}~;
      \underline{alerte}~; nuit~; \underline{belgique}~; \underline{août 2012}~;
      \underline{chaleur}~; \underline{température}~; \underline{chaude}~;
      \underline{canicule}~; dimanche.
    }
  }~\\

  \vspace{1em}

  \framebox[\linewidth]{
    \parbox{.99\linewidth}{\footnotesize
      \textbf{Sortie de \textsc{Tf-Idf}~:} \underline{luxembourg}~;
      \underline{belgique}~; \underline{alerte}~; \underline{canicule}~;
      \underline{chaude}~; nuit~; \underline{chaleur}~; sud~; dimanche~;
      deuxième nuit.

      \textbf{Sortie de TextRank~:} \underline{août 2012}~; août 2003~; alerte
      orange~; vigilance orange~; deuxième nuit~; légère brise.

      \textbf{Sortie de SingleRank~:} alerte orange~; alerte jaune~; alerte
      rouge~; \underline{alerte}~; deuxième nuit~; \underline{août 2012}~; août
      2003~; vigilance orange~; légère brise~; \underline{luxembourg}.
    }
  }

  \caption[
    Exemple d'extraction de termes-clés avec TopicRank, comparé à
    \textsc{Tf-Idf}, TextRank et SingleRank, sur un article
    journalistique de Wikinews
  ]{
    Exemple d'extraction de termes-clés avec TopicRank, comparé à
    \textsc{Tf-Idf}, TextRank et SingleRank, sur un article
    journalistique de Wikinews. Les termes-clés soulignés sont les
    termes-clés correctement extraits.
    \label{fig:exemple_topicrank}
  }
\end{figure}

