\section{Extraction de termes-clés avec TopicRank}
\label{sec:extraction_de_termes_cles_avec_topicrank}
  % Qu'est-ce que TopicRank ?
  TopicRank est une méthode non supervisée d'extraction de termes-clés qui
  modélise un document sous la forme d'un graphe de sujets.
  % Quels problèmes résout-il ?
  Elle se différencie des autres méthodes à base de graphe, car, plutôt que de
  chercher les mots importants du document, elle cherche ses sujets importants.
  % Quel en est le fonctionnement général ?
  Notre méthode repose sur les trois étapes suivantes qui seront détaillées dans
  la suite~:
  \begin{itemize*}
    \item[]{identification des sujets~;}
    \item[]{ordonnancement des sujets~;}
    \item[]{sélection des termes-clés.}
  \end{itemize*}

  \subsection{Identification des sujets}
  \label{subsec:identification_des_sujets}
    Dans ce travail, un sujet est une information générale ou le plus souvent
    spécifique véhiculée par au moins une unité textuelle présente dans le
    document analysé. Nous ne nous reposons donc pas sur une formulation
    approfondie d'un sujet, nous groupons seulement les termes-clés candidats
    lorsque ceux-ci véhiculent la même information.

    % Que nous faut-il pour identifier les sujets ?
    La première étape de l'identification des sujets consiste à sélectionner les
    termes-clés candidats.
    % Quels candidats composent les sujets ?
    Afin de réaliser une identification de qualité des sujets, nous excluons la
    sélection des n-grammes qui fournit beaucoup plus de candidats non
    pertinents que les autres méthodes. Concernant le choix entre la sélection
    des \textit{chunks} nominaux et la sélection des plus longues séquences de
    noms et d'adjectifs, nous suivons \newcite{wan2008expandrank} et
    \newcite{hassan2010conundrums} en sélectionnant les plus longues séquences
    de noms et d'adjectifs. Cette méthode présente l'avantage de fournir des
    candidats grammaticalement corrects et de ne nécessiter qu'une adaptation
    limitée pour le traitement de documents d'une autre langue.

    % Comment détectons nous deux candidats appartenant au même sujet ?
    La seconde étape de l'identification des sujets consiste à grouper les
    termes-clés candidats lorsqu'ils appartiennent au même sujet. Dans le souci
    de proposer une méthode ne faisant pas l'usage de données supplémentaires,
    nous optons pour un groupement quelque peu naïf des candidats. Deux
    candidats $c_1$ et $c_2$ sont groupés en fonction d'une similarité de
    Jaccard par laquelle ils sont considérés comme des sacs de mots tronqués
    selon la méthode de racinisation\footnote{Cette racinisation a pour effet de
    grouper les candidats qui varient uniquement en termes de flexion ou de
    dérivation.} de \newcite{porter1980suffixstripping}~:
    \begin{align}
      \text{sim}(c_1, c_2) &= \frac{\|c_1 \cap c_2\|}{\|c_1 \cup c_2\|} \label{equa:jaccard}
    \end{align}
    Cette mesure est naïve dans le sens où l'ordre des mots, leur ambiguïté et
    les liens de synonymie ne sont pas pris en compte. À cela s'ajoute aussi des
    erreurs introduites par l'usage de la méthode de
    \newcite{porter1980suffixstripping} (par exemple les mots \og{}empire\fg{} et
    \og{}empirique\fg{} partagent le même radical, \og{}empir\fg{}).

    % Comment groupons nous les candidats d'un même sujet ?
    Une fois la similarité connue entre toutes les paires de candidats, nous
    appliquons l'algorithme de groupement hiérarchique agglomératif
    (\textit{Hierarchical Agglomerative Clustering -- HAC}). Initialement,
    chaque candidat représente un groupe et, jusqu'à l'obtention d'un nombre
    prédéfini de groupes, les deux groupes ayant la plus forte similarité sont
    unis pour ne former qu'un seul groupe. Afin de ne pas fixer le nombre de
    sujets à créer comme condition d'arrêt de l'algorithme, nous définissons un
    seuil de similarité $\zeta$ entre les groupes deux à deux. Cette similarité
    entre deux groupes est déterminée à partir de la similarité de Jaccard
    calculée entre les candidats de chaque groupe. Il existe trois stratégies
    pour calculer la similarité entre deux groupes~:
    \begin{itemize}
      \item{simple~: la plus grande valeur de similarité entre les candidats
            des deux groupes sert de similarité entre eux~;}
      \item{complète~: la plus petite valeur de similarité entre les candidats
            des deux groupes sert de similarité entre eux~;}
      \item{moyenne~: la moyenne de toutes les similarités entre les
            candidats des deux groupes sert de similarité entre eux (compromis
            entre les stratégies simple et complète).}
    \end{itemize}
    L'une ou l'autre de ces stratégies est à privilégier en fonction du type des
    candidats extraits. Pour des candidats qui ont de forts recouvrements, tels
    que les n-grammes, il semble plus pertinent d'utiliser la
    stratégie complète qui est la moins agglomérative. Dans le cas de TopicRank,
    où les candidats sont de meilleure qualité que les n-grammes, la stratégie
    moyenne est une meilleure alternative.

  \subsection{Ordonnancement des sujets}
  \label{subsec:ordonnancement_des_sujets}
    % Quel est le but de l'ordonnancement ?
    % Comment est-il effectué ?
    L'ordonnancement des sujets a pour objectif de trouver quels sont ceux qui
    ont le plus d'importance dans le document analysé. À l'instar de
    \newcite{mihalcea2004textrank}, l'importance des sujets est déterminée à
    partir d'un graphe.

    % Comment le graphe est-il construit ?
    Les sujets du document analysé composent les n\oe{}uds $V$ du graphe complet
    $G = (V, E)$, où $E$ est l'ensemble des liens entre les
    n\oe{}uds\footnote{$E = \{(v_1, v_2)\ |\ \forall{v_1, v_2 \in V}, v_1 \neq
    v_2\}$, car $G$ est un graphe complet.}. Du fait que le graphe utilisé soit
    un graphe complet, la pondération de ses arêtes est l'étape la plus
    importante pour rendre possible un ordonnancement efficace des sujets. Pour
    cette pondération, nous suivons \newcite{wan2008expandrank} et utilisons la
    force du lien sémantique entre les sujets. Cependant, parce que nous
    utilisons un graphe complet, il ne nous est pas possible de représenter
    cette force par le nombre de cooccurrences entre les sujets. Pour préserver
    l'intuition derrière l'usage du nombre de cooccurrences, nous représentons
    la force du lien sémantique entre deux sujets par la distance entre les
    candidats des sujets dans le document~:
    \begin{align}
      \text{poids}(s_i, s_j) &= \sum_{c_i \in s_i}\ \sum_{c_j \in s_j} \text{dist}(c_i, c_j) \label{math:ponderation}\\
      \text{dist}(c_i, c_j) &= \sum_{p_i \in \text{pos}(c_i)}\ \sum_{p_j \in \text{pos}(c_j)} \frac{1}{|p_i - p_j|} \label{math:distance}
    \end{align}
    où $\text{poids}(s_i, s_j)$ est le poids de l'arête entre les sujets $s_i$
    et $s_j$, et où $\text{dist}(c_i, c_j)$ représente la force sémantique entre
    les candidats $c_i$ et $c_j$, calculée à partir de leurs positions
    respectives, $\text{pos}(c_i)$ et $\text{pos}(c_j)$, dans le document.

    % Comment le graphe est-il utilisé pour ordonner les sujets ?
    % Quelle est l'intuition de PageRank/TextRank ?
    Une fois le graphe construit, l'algorithme d'ordonnancement de TextRank est
    utilisé pour identifier quels sont les sujets les plus importants du
    document. Cet ordonnancement se fonde sur le principe de recommandation (de
    vote), c'est-à-dire un sujet est d'autant plus important s'il est fortement
    connecté avec un grand nombre de sujets et si les sujets avec lesquels il
    est fortement connecté sont importants~:
    \begin{align}
      \text{importance}(s_i) = (1 - \lambda) + \lambda \times \sum_{s_j \in V_i} \frac{\text{poids}(s_i, s_j) \times \text{importance}(s_j)}{\sum_{s_k \in V_j} \text{poids}(s_j, s_k)} \label{math:textrank}
    \end{align}
    où $V_i$ est l'ensemble des sujets connectés au
    sujet\footnote{$V_i = \{v_i\ |\ \forall{v_j \in V}, v_j \neq v_i\}$, car $G$
    est un graphe complet.} $s_i$ et où $\lambda$ est un facteur d'atténuation
    définit à 0,85 d'après les recommandations de \newcite{brin1998pagerank}.

  \subsection{Sélection des termes-clés}
  \label{subsec:selection_des_termes_cles}
    % De quoi s'agit-il ?
    La sélection des termes-clés est la dernière étape de TopicRank. Elle
    consiste à chercher les termes-clés candidats qui représentent le mieux les
    sujets importants. Dans le but de ne pas extraire de termes-clés redondants,
    un seul candidat est sélectionné par sujet.
    % Quel en est le but ?
    Ainsi, pour $k$ sujets, $k$ termes-clés non redondants couvrant exactement
    $k$ sujets sont extraits.

    % Quelles sont les différentes stratégies envisageable ?
    La difficulté de ce principe de sélection réside dans la capacité à trouver
    parmi plusieurs termes-clés candidats d'un même sujet celui qui le
    représente le mieux. Nous proposons trois stratégies de sélection pouvant
    répondre à ce problème~:
    \begin{itemize}
      \item{la première position~: en supposant qu'un sujet est tout d'abord
            introduit sous sa forme la plus appropriée, le terme-clé candidat
            sélectionné pour un sujet est celui qui apparaît en premier dans le
            document analysé;}
      \item{la fréquence~: en supposant que la forme la plus représentative d'un
            sujet est sa forme la plus fréquente, le terme-clé candidat
            sélectionné pour un sujet est celui qui est le plus fréquent dans le
            document analysé;}
      \item{le centroïde~: le terme-clé candidat sélectionné pour un sujet est
            celui qui est le plus similaire aux autres candidats du sujet (voir
            l'équation~\ref{equa:jaccard}).}
    \end{itemize}
    % Laquelle des trois stratégies semble être la mieux ?
    Parmi ces trois stratégies, celle qui semble la plus appropriée est la
    stratégie qui se fonde sur la première position des termes-clés candidats.
    Sélectionner les candidats les plus fréquents risque de ne pas être une
    solution stable selon les types de documents, en particulier selon leur
    taille, tandis que sélectionner les centroïdes risque de ne pas fournir les
    termes-clés les plus précis.

  % Que donne l'extraction ? (exemple)
  La figure~\ref{fig:exemple_topicrank} donne un exemple d'extraction de
  termes-clés avec TopicRank à partir de l'exemple de la
  figure~\ref{fig:recurrent_example}. Dans
  cet exemple, nous observons un groupement correct de toutes les variantes
  de \og alertes~\fg, mais aussi un groupement erroné de \og août 2003~\fg\ avec
  \og août 2012~\fg. Dans ce dernier cas, TopicRank est tout de même capable
  d'extraire \og août 2012~\fg\ grâce à la sélection du candidat apparaissant en
  premier. Globalement, l'extraction des termes-clés est correcte et huit
  termes-clés sur les dix extraits apparaissent dans l'ensemble de termes-clés
  assignés par des humains pour ce document.
  \begin{figure}
    \centering
    \cornersize{0.1}\Ovalbox{
      \parbox{.95\linewidth}{\small
        Météo du 19 août 2012~: alerte à la canicule sur la Belgique et le
        Luxembourg\\

        \setlength\parindent{12pt}\vspace{-0.5em}
        A l'exception de la province de Luxembourg, en alerte jaune,
        l'ensemble de la Belgique est en vigilance orange à la canicule. Le
        Luxembourg n'est pas épargné par la vague du chaleur : le nord du pays
        est en alerte orange, tandis que le sud a était placé en alerte rouge.

        En Belgique, la température n'est pas descendue en dessous des 23°C
        cette nuit, ce qui constitue la deuxième nuit la plus chaude jamais
        enregistrée dans le royaume. Il se pourrait que ce dimanche soit la
        journée la plus chaude de l'année. Les températures seront comprises
        entre 33 et 38°C. Une légère brise de côte pourra faiblement
        rafraichir l'atmosphère. Des orages de chaleur sont a prévoir dans la
        soirée et en début de nuit.

        Au Luxembourg, le mercure devrait atteindre 32°C ce dimanche sur
        l'Oesling et jusqu'à 36°C sur le sud du pays, et 31 à 32°C lundi. Une
        baisse devrait intervenir pour le reste de la semaine. Néanmoins, le
        record d'août 2003 (37,9°C) ne devrait pas être atteint.
      \setlength\parindent{0pt}}
    }~\\

    \vspace{2em}

    \begin{overpic}[width=.8\linewidth]{include/44960_topicrank_empty.eps}
      \put(38,100.5){\small[soirée]}
      \put(22,96){\small[oesling]}
      \put(55,100){\small[nord]}
      \put(69,95){\small[belgique]}
      \put(0,86){\small[août 2003~; août 2012]}
      \put(83,83.5){\small[36\degre{}c]}
      \put(0,73){\small[37,9\degre{}c]}
      \put(89,70.5){\small[royaume]}
      \put(-5,57.5){\small[année]}
      \put(92.5,53.5){\small[dimanche]}
      \put(-11,41){\small[vigilance orange]}
      \put(88.5,38){\small[luxembourg]}
      \put(1.5,25){\small[début]}
      \put(77,23){\small[nuit~; deuxième nuit]}
      \put(11,13.5){\small[exception]}
      \put(72,10.5){\small[canicule]}
      \put(25,5){\small[journée]}
      \put(60,4.5){\small[côte]}
      \put(17,1){\small[alerte rouge~; alerte jaune~; alerte orange~; alerte]}
      \put(40,84){\small[ensemble]}
      \put(24,80){\small[légère brise]}
      \put(58,82.5){\small[record]}
      \put(15.5,68.5){\small[météo]}
      \put(69.5,72){\small[orages]}
      \put(11,53){\small[chaude]}
      \put(79,58){\small[baisse]}
      \put(10,37){\small[température]}
      \put(72,42){\small[atmosphère]}
      \put(26,26){\small[23\degre{}c]}
      \put(71,27){\small[sud]}
      \put(39,17){\small[lundi]}
      \put(55.5,20.5){\small[38\degre{}c]}
      \put(31,63){\small[chaleur]}
      \put(49.5,68){\small[reste]}
      \put(27,46){\small[province]}
      \put(62,56){\small[vague]}
      \put(41,34.5){\small[pays]}
      \put(56,39){\small[mercure]}
      \put(43,51){\small[semaine]}
    \end{overpic}~\\

    \vspace{1.5em}

    \cornersize{0.1}\Ovalbox{
      \parbox{.95\linewidth}{\small
        \textbf{Termes-clés extraits par des humains~:}

        Luxembourg~; alerte~; météo~; belgique~; août 2012~; chaleur~;
        température~; chaude~; canicule~; orange~; la plus chaude.

        \textbf{Termes-clés extraits par TopicRank~:}

        \underline{Luxembourg}~; \underline{alerte}~; nuit~;
        \underline{belgique}~; \underline{août 2012}~; \underline{chaleur}~;
        \underline{température}~; \underline{chaude}~; \underline{canicule}~;
        dimanche.
      }
    }

    \caption{Exemple d'extraction de termes-clés avec TopicRank. Les termes-clés
             soulignés sont les termes-clés correctement extraits.
             \label{fig:exemple_topicrank}}
  \end{figure}

