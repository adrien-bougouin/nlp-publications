\section{Introduction}
\label{sec:introduction}
  % % Que sont les termes-clés ?
  % Les termes-clés sont des mots ou des expressions polylexicales qui représentent les sujets principaux du document auquel ils se réfèrent.
  % % À quoi sont-ils utiles ?
  % Du fait de leur propriété synthétique, les termes-clés sont utilisés dans de nombreuses applications telles que l'indexation de documents~\cite{medelyan2008smalltrainingset}, le résumé automatique~\cite{litvak2008graphbased} ou encore la classification de documents~\cite{han2007webdocumentclustering}.
  %   % Avec l'augmentation exponentielle de la quantité de documents à notre disposition et le besoin d'accéder toujours plus rapidement à ceux-ci, les enrichissements susmentionnés deviennent une nécessité.
  % Avec la croissance exponentielle du nombre de documents disponibles, en particulier sur le web, les termes-clés constituent un moyen efficace pour accéder rapidement aux informations pertinentes.
  % Cependant, la plupart des documents ne sont pas pourvus de termes-clés et l'annotation manuelle de ces derniers est une tâche beaucoup trop coûteuse pour être envisagée.
  % % Que faut-il faire pour les obtenir ?
  %   % De ce fait, nous observons un intérêt grandissant pour la tâche d'extraction automatique de termes-clés.
  % C'est la raison pour laquelle la problématique de l'extraction automatique de termes-clés suscite de plus en plus l'intéret des chercheurs.

  % % Comment cela a-t-il déjà été fait ?
  % L'extraction automatique de termes-clés peut être perçue soit comme une tâche
  % de classification binaire de termes-clés candidats~\cite{witten1999kea}, soit
  % comme une tâche d'ordonnancement de termes-clés
  % candidats~\cite{mihalcea2004textrank}. Dans le premier cas, l'extraction est
  % le plus souvent supervisée, elle nécessite une collection de documents annotés
  % en termes-clés pour une phase d'apprentissage préliminaire. Dans le second
  % cas, l'extraction est le plus souvent non-supervisée, elle ne nécessite pas de
  % documents préalablement annotés. Les méthodes non-supervisées ont des
  % performances plus faibles que les méthodes supervisées actuelles. Ceci
  % s'explique par le fait que les méthodes supervisées apprennent les critères
  % discriminants pour l'extraction de termes-clés à partir de documents annotés.
  % Cependant, la dépendance de ces méthodes au domaine des documents utilisés
  % lors de l'apprentissage pousse de plus en plus de chercheurs à s'intéresser à
  % l'extraction non-supervisée de termes-clés.

  % % Que proposons-nous dans cet article ?
  % % Comment le fait-on ?
  % Dans cet article, nous présentons TopicRank, une méthode d'extraction
  % non-supervisée de termes-clés qui se fonde sur les travaux de
  % \newcite[TextRank]{mihalcea2004textrank} pour l'ordonnancement à base de
  % graphe des unités textuelles d'un document. TopicRank groupe les termes-clés
  % candidats selon leur appartenance à un sujet, ordonne les sujets par
  % importance dans le document, puis sélectionne pour chacun des meilleurs sujets
  % le candidat qui le représente le mieux (son terme-clé associé).
  % % Quelle est la nouveauté ?
  % % En quoi cela semble-t-il meilleur ?
  % Contrairement à l'ordonnancement des mots tel qu'il est fait avec TextRank,
  % le groupement des candidats en sujets utilisés pour l'ordonnancement permet de
  % tirer partie d'informations complémentaires extraites de différents candidats
  % d'un même sujet. De plus, le fait de ne sélectionner qu'un seul candidat par
  % sujet permet d'éviter l'extraction de termes-clés redondants.

  % % Que peut-on dire sur les évaluation réalisées ?
  % Pour évaluer TopicRank, nous utilisons quatre collections de données dont les
  % propriétés diffèrent (nature, langue, taille des documents, etc.) afin de
  % mieux observer ses avantages et ses faiblesses~\cite{hassan2010conundrums}. En
  % addition, nous comparons TopicRank avec trois méthodes non-supervisées, l'une
  % extrayant des statistiques à partir de documents supplémentaires et les deux
  % autres appartenant à la catégorie des méthodes à base de graphe. Pour trois
  % des collections utilisées, TopicRank donne de meilleurs résultats que les
  % autres méthodes. De plus, cette amélioration est significative vis-à-vis des
  % deux méthodes à base de graphe.

  Un terme-clé est un mot ou une expression polylexicale permettant de caractériser le contenu d'un document.
  Un ensemble de termes-clés permet ainsi de définir les thématiques représentées dans un document.
  Du fait de leur propriété synthétique, les termes-clés sont utilisés dans de nombreuses applications du Traitement Automatique des Langues (TAL) telles que l'indexation automatique de documents~\cite{medelyan2008smalltrainingset}, le résumé automatique~\cite{avanzo2005keyphrase}, la compression de phrase~\cite{boudin2013multisentencecompression} ou encore la classification de documents~\cite{han2007webdocumentclustering}.
  Malheureusement, la plupart des documents auxquels nous avons accès ne sont pas pourvus de termes-clés, et leur annotation manuelle n'est évidemment pas envisageable.
  De nombreux chercheurs se sont donc penchés sur la problématique de l'extraction automatique termes-clés.

  Parmi les différentes méthodes d'extraction automatique de termes-clés proposées dans la littérature, deux grandes catégories émergent~: les méthodes supervisées et les méthodes non-supervisées.
  Les premières réduisent la tâche d'extraction de termes-clés en une tâche de classification binaire~\cite{witten1999kea}, où il s'agit d'attribuer la classe \og{}\textit{terme-clé}\fg{} ou \og{}\textit{non terme-clé}\fg{} aux différents candidats extraits à partir du document.
  Une collection de documents annotés en termes-clés est alors nécessaire pour l'apprentissage du modèle de classification.
  Les méthodes non-supervisées, quant à elles, attribuent un score d'importance à chaque candidat en fonction de divers indicateurs comme par exemple la fréquence, le nombre de co-occurrences ou la position dans le document.
  Bien que les méthodes supervisées soient en général plus performantes, la faible quantité de documents annotés en termes-clés disponibles couplée à la forte dépendance des modèles de classification vis-à-vis du type de documents sur lesquels ils ont été appris, poussent les chercheurs à s'intéresser de plus en plus aux méthodes non-supervisées.

  Les méthodes non-supervisées les plus étudiées sont sans conteste celles basée sur TextRank~\cite{mihalcea2004textrank}, qui est une méthode d'ordonnancement d'unités textuelles à partir de graphe.
  Les graphes sont un moyen naturel de représenter les unités textuelles et les relations qui les relient, et ils sont utilisés dans de nombreuses applications du TAL~\cite{kozareva2013textgraphs}.
  Pour l'extraction de termes-clés, l'idée est de représenter le document sous la forme d'un graphe dans lequel les noeuds correspondent aux mots et les arêtes correspondent à des relations de co-occurrences dans une fenêtre de mots.
  Un score d'importance est alors calculé pour chaque mot selon un principe de recommendation, c'est-à-dire un mot est d'autant plus important s'il co-occurre avec un grand nombre de mots et si les mots avec lesquels il co-occurre sont eux aussi importants.
  Les mots les plus importants sont ensuite assemblés pour générer les termes-clés.

  Dans cet article, nous présentons TopicRank, une méthode non-supervisée d'extraction de termes-clés basée sur TextRank.
  TopicRank groupe les termes-clés candidats selon leur appartenance à un sujet, représente le document sous la forme d'un graphe complet de sujets, ordonne les sujets selon leur importance, puis sélectionne pour chacun des meilleurs sujets le candidat le plus représentatif.
  Notre approche possède plusieurs avantages par rapport à TextRank que nous détaillons ci-dessous~:
  %
  %
  \begin{enumerate}
    \item Le regroupement des termes-clés candidats en sujets supprime en amont les problèmes de redondance dans les termes-clés extraits.
    \item Le fait d'utiliser des sujets à la place des mots permet de construire un graphe plus compact, de renforcer le poids des arêtes dans le graphe et d'améliorer la qualité de l'ordonnancement.
    \item La construction d'un graphe complet permet de supprimer le paramètre de la fenêtre de mots et de représenter de manière plus précise les relations entre les sujets.
  \end{enumerate}

  Pour évaluer notre méthode, nous utilisons quatre collections de test aux propriétés différentes (nature des documents, taille des documents, langue, etc.).
  Nous comparons TopicRank à trois autres méthodes non-supervisées et détaillons l'impact de chacune des contributions que nous proposons.

  % Quel est notre plan ?
  L'article est structuré comme suit.
  Après un état de l'art des méthodes d'extraction automatique de termes-clés en section~\ref{sec:etat_de_l_art}, nous décrivons le fonctionnement de TopicRank en section~\ref{sec:extraction_de_termes_cles_avec_topicrank} et présentons son évaluation approfondie en section~\ref{sec:evaluation}.
  Enfin, nous concluons et discutons des travaux futurs dans la section~\ref{sec:conclusion_et_perspectives}.
