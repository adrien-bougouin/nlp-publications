\section{Extraction de termes-clés avec TopicRank}
\label{sec:extraction_de_termes_cles_avec_topicrank}
  % Qu'est-ce que TopicRank ?
  TopicRank est une méthode non-supervisée qui extrait les termes-clés d'un
  document à partir de sa représentation sous la forme d'un graphe de sujets.
  % Quels problèmes résout-il ?
  Elle se différencie des autres méthodes à base de graphe, car, plutôt que de
  chercher les mots importants du document, elle cherche ses sujets importants,
  ceci quels qu'en soient leurs vecteurs (termes-clés candidats). Ce nouveau
  procédé présente l'intérêt de rassembler des informations complémentaires
  véhiculées par des candidats différents, mais appartenant tout de même au même
  sujet.
  % Quel en est le fonctionnement général ?
  Dans un premier temps, les termes-clés candidats sont groupés par sujets, puis
  les sujets sont ordonnés et enfin, les candidats les plus représentatifs des
  sujets les plus importants sont extraits comme termes-clés.

  \subsection{Identification des sujets}
  \label{subsec:identification_des_sujets}
    % Comment détectons nous deux candidats appartenant au même sujet ?
    Dans le soucis de proposer une méthode efficace ne faisant pas l'usage de
    ressources externes, nous optons pour un groupement quelque peu naïf des
    termes-clés candidats appartenant au même sujet. En effet, les candidats
    sont groupés en fonction d'une similatité de Jaccard (voir
    l'équation~\ref{equa:jaccard}) dans laquelle ils sont considérés comme des
    sacs de mots. En addition, les mots sont tronqués selon la méthode de
    \newcite{porter1980suffixstripping}, afin de considérer identiques deux
    variantes flexionnelles d'un même mot. Cette mesure est naïve dans le sens
    où l'ordre des mots, leur ambiguïté et les liens de synonymie ne sont pas
    pris en compte.
    \begin{align}
      \text{sim}(c_1, c_2) &= \frac{\|c_1 \cap c_2\|}{\|c_1 \cup c_2\|} \label{equa:jaccard}
    \end{align}

    % Comment groupons nous les candidats d'un même sujet ?
    Une fois la similarité connue entre tous les candidats deux à deux, nous
    appliquons l'algorithme de groupement hiérarchique agglomératif
    (\textit{Hierarchical Agglomerative Clustering -- HAC}). Initialement,
    chaque candidat représente un groupe. À chaque itération de l'algorithme,
    les deux groupes ayant la plus forte similarité sont groupés. La condition
    d'arrêt est définie par un seuil, fixé à $0,25$, pour la similarité entre
    deux groupes. Cette similarité entre deux groupes est calculée en fonction
    de la similarité entre les candidats de chaque groupe. Il existe trois
    stratégies pour calculer cette similarité~:
    \begin{itemize}
      \item{simple~: la plus petite valeur de similarité entre les candidats
            des deux groupes sert de similarité entre eux~;}
      \item{moyenne~: la moyenne de toutes les similarités entre les
            candidats des deux groupes sert de similarité entre eux~;}
      \item{complète~: la plus grande valeur de similarité entre les candidats
            des deux groupes sert de similarité entre eux.}
    \end{itemize}
    Nous suggérons d'utiliser l'une ou l'autre de ces stratégies en fonction des
    termes-clés candidats qui sont utilisés. Par exemple, certains ensemble de
    candidats, tels que les n-grammes pour $n \in 1..m$,
    contenant de nombreux candidats se recouvrant partiellement risquent de
    donner lieu à des groupes non consistants avec la stratégie complète. En
    revanche, la stratégie simple a tendance à moins regrouper, elle est donc
    plus adaptée à ces ensembles de candidats. Dans l'optique de comparer
    TopicRank en utilisant différentes méthodes d'extraction de termes-clés
    candidats, nous choisissons d'utiliser la stratégie moyenne, qui se place
    comme étant le compromis entre les deux autres.

  \subsection{Ordonnancement des sujets}
  \label{subsec:ordonnancement_des_sujets}
    % Quel est le but de l'ordonnancement ?
    % Comment est-il effectué ?
    L'ordonnancement des sujets a pour objectif de trouver quels sont ceux qui
    sont les plus important dans le document analysé. À l'instar de
    \newcite{mihalcea2004textrank}, nous décidons de déterminer l'importance des
    sujets en modélisant le document sous la forme d'un graphe de sujets, puis
    en applicant l'algorithme d'ordonnancement de TextRank.

    % Comment le graphe est-il construit ?
    Les sujets du document analysé composent les n\oe{}uds ($V$) du graphe
    complet $G = (V, E)$, $E$ étant l'ensemble des liens entre les
    n\oe{}uds\footnote{$E = \{(v_1, v_2)\ |\ \forall{v_1, v_2 \in V}, v_1 \neq v_2\}$,
    car $G$ est un graphe complet.}. Le graphe utilisé étant un graphe complet,
    la pondération de ses arêtes est l'étape la plus importante pour rendre
    possible un ordonnancement efficace des sujets. Pour celle-ci, nous 
    choisissons d'utiliser la force du liens sémantique entre les sujets.
    Contrairement à ce qui est fait dans les autres
    travaux~\cite{wan2008expandrank,tsatsaronis2010semanticrank,liu2010topicalpagerank},
    nous ne représentons pas cette force avec le nombre de co-occurrences
    calculées dans une fenêtre de mots définie manuellement, mais nous utilisons
    la distance entre les mots de sujets~:
    \begin{align}
      \text{poids}(s_i, s_j) &= \sum_{c_i \in s_i}\ \sum_{c_j \in s_j} \text{dist}(c_i, c_j) \label{math:ponderation}\\
      \text{dist}(c_i, c_j) &= \sum_{p_i \in \text{pos}(c_i)}\ \sum_{p_j \in \text{pos}(c_j)} \frac{1}{|p_i - p_j|} \label{math:distance}
    \end{align}
    où $\text{poids}(s_i, s_j)$ est le poids de l'arête entre les sujets $s_i$
    et $s_j$, et où $\text{dist}(c_i, c_j)$ représente la force sémantique entre
    les candidats $c_i$ et $c_j$, calculée à partir de leurs positions
    respectives, $\text{pos}(c_i)$ et $\text{pos}(c_j)$, dans le document.

    % Comment le graphe est-il utilisé pour ordonner les sujets ?
    % Quelle est l'intuition de PageRank/TextRank ?
    Une fois la construction du graphe, l'algorithme d'ordonnancement de
    TextRank est utilisé. Celui-ci se fonde sur le principe de \og vote~\fg,
    c'est à dire qu'un sujet fortement connecté à un autre sujet est fortement
    recommendé par le dernier et gagne donc de l'importance. De ce fait, un
    sujet connecté à un autre sujet très important gagne aussi plus
    d'importance~:
    \begin{align}
      \text{importance}(s_i) = (1 - \lambda) + \lambda \times \sum_{s_j \in V_i} \frac{\text{poids}(s_i, s_j) \times \text{importance}(s_j)}{\sum_{s_k \in V_j} \text{poids}(s_j, s_k)} \label{math:textrank}
    \end{align}
    où $V_i$ est l'ensemble des sujets connectés au sujet
    $s_i$\footnote{$V_i = \{v_i\ |\ \forall{v_j in V}, v_j \neq v_i\}$,
    car $G$ est un graphe complet.} et où $\lambda$ est un facteur d'aténuation
    définit par défaut à $0,85$ par \newcite{brin1998pagerank}.

  \subsection{Sélection des termes-clés}
  \label{subsec:selection_des_termes_cles}
    % De quoi s'agit-il ?
    % Quel en est le but ?
    % Quelles sont les différentes stratégies envisageable ?

  % Que donne l'extraction ? (exemple)

