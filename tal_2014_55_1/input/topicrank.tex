\section{Extraction de termes-clés avec TopicRank}
\label{sec:extraction_de_termes_cles_avec_topicrank}
  % Qu'est-ce que TopicRank ?
  TopicRank est une méthode non-supervisée qui extrait les termes-clés d'un
  document à partir de sa représentation sous la forme d'un graphe de sujets.
  % Quels problèmes résout-il ?
  Elle se différencie des autres méthodes à base de graphe, car, plutôt que de
  chercher les mots importants du document, elle cherche ses sujets importants
  quelqu'en soient leurs vecteurs (termes-clés candidats). Ce nouveau procédé
  présente l'intérêt de rassembler des informations complémentaires véhiculées
  par des candidats différents, mais appartenant au même sujet.
  % Quel en est le fonctionnement général ?
  Dans un premier temps, les termes-clés candidats sont groupés par sujets, puis
  les sujets sont ordonnés et enfin, les candidats les plus représentatifs des
  sujets les plus importants sont extraits comme termes-clés.

  \subsection{Identification des sujets}
  \label{subsec:identification_des_sujets}
    % Que nous faut-il pour identifier les sujets ?
    La première étape de l'identification des sujets consiste à trouver les
    unités textuelles qui les véhiculent. Parce que les termes-clés représentent
    les sujets les plus importants d'un documents, nous choisisons les
    termes-clés candidats comme étant ces unités textuelles.
    % Quels candidats composent les sujets ?
    À l'instar des travaux de \newcite{wan2008expandrank} et de
    \newcite{hassan2010conundrums}, TopicRank extrait les plus longues séquences
    de noms (noms propres inclus) et d'adjectifs. Cette méthode d'extraction
    présente les avantages d'extraire des candidats dont la syntaxe est
    controllée et ne nécessite pas d'adaptation conséquente pour une quelconque
    langue. La syntaxe controllée permet d'éviter l'extraction d'un grand nombre
    d'unités textuelles non pertinentes afin d'améliorer la qualité de
    l'identification des sujets lors de la seconde étape.

    % Comment détectons nous deux candidats appartenant au même sujet ?
    La seconde étape de l'identification des sujets consiste à grouper les
    termes-clés candidats lorsqu'ils appartiennent au même sujet.
    Dans le soucis de proposer une méthode ne faisant pas l'usage de données
    supplémentaires, nous optons pour un groupement quelque peu naïf des
    candidats. Ceux-ci sont groupés en fonction d'une similatité de Jaccard
    (voir l'équation~\ref{equa:jaccard}) dans laquelle ils sont considérés comme
    des sacs de mots. En addition, les mots sont tronqués selon la méthode de
    \newcite{porter1980suffixstripping} afin de considérer identiques ceux qui
    ont le même radical. Cette mesure est naïve dans le sens où l'ordre des
    mots, leur ambiguïté et les liens de synonymie ne sont pas pris en compte.
    \begin{align}
      \text{sim}(c_1, c_2) &= \frac{\|c_1 \cap c_2\|}{\|c_1 \cup c_2\|} \label{equa:jaccard}
    \end{align}

    % Comment groupons nous les candidats d'un même sujet ?
    Une fois la similarité connue entre tous les candidats deux à deux, nous
    appliquons l'algorithme de groupement hiérarchique agglomératif
    (\textit{Hierarchical Agglomerative Clustering -- HAC}). Initialement,
    chaque candidat représente un groupe. À chaque itération de l'algorithme,
    les deux groupes ayant la plus forte similarité sont unis pour ne former
    qu'un seul groupe. Afin de ne pas fixer le nombre de sujets à créer comme
    condition d'arrêt de l'algorithme, nous définissons un seuil de similarité
    $\zeta$ entre les groupes deux à deux. Cette similarité entre deux groupes
    est calculée en fonction de la similarité entre les candidats de chaque
    groupe. Il existe trois stratégies pour calculer cette similarité~:
    \begin{itemize}
      \item{simple~: la plus grande valeur de similarité entre les candidats
            des deux groupes sert de similarité entre eux~;}
      \item{complète~: la plus petite valeur de similarité entre les candidats
            des deux groupes sert de similarité entre eux.}
      \item{moyenne~: la moyenne de toutes les similarités entre les
            candidats des deux groupes sert de similarité entre eux (compromis
            entre les stratégies simple et complète)~;}
    \end{itemize}
    L'une ou l'autre de ces stratégies est à privilégier en fonction du type des
    candidats qui sont extraits. Pour des candidats tels que les n-grammes pour
    $n \in 1..m$, il semble par exemple plus pertinent d'utiliser la stratégie
    complète, qui, contrairement aux deux autres, a moins tendance à grouper de
    nombreux candidats ayant quelques mots en commun (voir un seul). Dans le cas
    de TopicRank, les termes-clés candidats étant de meilleur qualité que les
    n-grammes, la stratégie moyenne est une meilleure alternative.

  \subsection{Ordonnancement des sujets}
  \label{subsec:ordonnancement_des_sujets}
    % Quel est le but de l'ordonnancement ?
    % Comment est-il effectué ?
    L'ordonnancement des sujets a pour objectif de trouver quels sont ceux qui
    ont le plus d'importance dans le document analysé. À l'instar de
    \newcite{mihalcea2004textrank}, l'importance des sujets est déterminée à
    partir d'un graphe.

    % Comment le graphe est-il construit ?
    Les sujets du document analysé composent les n\oe{}uds $V$ du graphe complet
    $G = (V, E)$, $E$ étant l'ensemble des liens entre les
    n\oe{}uds\footnote{$E = \{(v_1, v_2)\ |\ \forall{v_1, v_2 \in V}, v_1 \neq v_2\}$,
    car $G$ est un graphe complet.}. Le graphe utilisé étant un graphe complet,
    la pondération de ses arêtes est l'étape la plus importante pour rendre
    possible un ordonnancement efficace des sujets. Pour celle-ci, nous 
    choisissons d'utiliser la force du liens sémantique entre les sujets.
    Contrairement à ce qui est fait dans les autres
    travaux~\cite{wan2008expandrank,tsatsaronis2010semanticrank,liu2010topicalpagerank},
    nous ne représentons pas cette force avec le nombre de co-occurrences
    calculées dans une fenêtre de mots définie manuellement, mais nous utilisons
    la distance entre les mots des sujets dans le document~:
    \begin{align}
      \text{poids}(s_i, s_j) &= \sum_{c_i \in s_i}\ \sum_{c_j \in s_j} \text{dist}(c_i, c_j) \label{math:ponderation}\\
      \text{dist}(c_i, c_j) &= \sum_{p_i \in \text{pos}(c_i)}\ \sum_{p_j \in \text{pos}(c_j)} \frac{1}{|p_i - p_j|} \label{math:distance}
    \end{align}
    où $\text{poids}(s_i, s_j)$ est le poids de l'arête entre les sujets $s_i$
    et $s_j$, et où $\text{dist}(c_i, c_j)$ représente la force sémantique entre
    les candidats $c_i$ et $c_j$, calculée à partir de leurs positions
    respectives, $\text{pos}(c_i)$ et $\text{pos}(c_j)$, dans le document.

    % Comment le graphe est-il utilisé pour ordonner les sujets ?
    % Quelle est l'intuition de PageRank/TextRank ?
    Une fois le graphe construit, l'algorithme d'ordonnancement de TextRank est
    utilisé pour identifier quels sont les sujets les plus importants du
    document. Cette ordonnancement se fonde sur le principe de \og vote~\fg,
    c'est à dire qu'un sujet fortement connecté à un autre sujet est fortement
    recommandé par le dernier, il gagne donc de l'importance. De ce fait, un 
    sujet connecté à un autre sujet très important gagne aussi plus
    d'importance~:
    \begin{align}
      \text{importance}(s_i) = (1 - \lambda) + \lambda \times \sum_{s_j \in V_i} \frac{\text{poids}(s_i, s_j) \times \text{importance}(s_j)}{\sum_{s_k \in V_j} \text{poids}(s_j, s_k)} \label{math:textrank}
    \end{align}
    où $V_i$ est l'ensemble des sujets connectés au
    sujet\footnote{$V_i = \{v_i\ |\ \forall{v_j in V}, v_j \neq v_i\}$, car $G$
    est un graphe complet.} $s_i$ et où $\lambda$ est un facteur d'atténuation
    définit par défaut à 0,85 par \newcite{brin1998pagerank}.

  \subsection{Sélection des termes-clés}
  \label{subsec:selection_des_termes_cles}
    % De quoi s'agit-il ?
    La sélection des termes-clés est la dernière étape de TopicRank. Celle-ci
    consiste à chercher les termes-clés candidats qui représentent le mieux les
    sujets importants qui sont abordés dans le document. De plus, dans le but de
    ne pas extraire de termes-clés redondants, un seul candidats est sélectionné
    par sujet.
    % Quel en est le but ?
    Pour chacun des $k$ sujets les plus importants, ce principe de sélection
    donne lieu à $k$ termes-clés non redondant et couvrant exactement $k$
    sujets.

    % Quelles sont les différentes stratégies envisageable ?
    La difficulté de ce principe de sélection réside dans la capacité à trouver
    parmi plusieurs termes-clés candidats d'un même sujet celui qui le
    représente le mieux. Nous distinguons trois stratégies de sélection pouvant
    répondre à ce problème~:
    \begin{itemize}
      \item{la première position~: en supposant qu'un sujet est tout d'abord
            introduit dans sa forme la plus appropriée, le terme-clé candidat
            sélectionné pour un sujet est celui qui apparaît en premier dans le
            document analysé;}
      \item{la fréquence~: en supposant que la forme la plus représentative d'un
            sujet est sa forme la plus fréquente, le terme-clé candidat
            sélectionné pour un sujet est celui qui est le plus fréquent dans le
            document analysé;}
      \item{le centroïde~: le terme-clé candidat sélectionné pour un sujet est
            celui a la plus haute similarité avec les autres candidats du sujet
            (voir l'équation~\ref{equa:jaccard}).}
    \end{itemize}
    % Laquelle des trois stratégies semble être la mieux ?
    Parmi ces trois stratégies, celle qui semble la plus appropriée est la
    stratégie qui se fonde sur la première position des termes-clés candidats.
    En effet, sélectionner les candidats les plus fréquents risque de favoriser
    l'extraction de formes abrégées ou de concepts inhérents. Par exemple, dans
    la collection SemEval (voir la section~\ref{sec:evaluation}), le document
    \textit{C-17} parle de \og réseaux à commutation de
    paquets~\fg\ (\textit{packet-switched networks}), mais le candidat le plus
    fréquent dans le sujet correspondant est le concept inhérent
    \og réseau~\fg\ (\textit{network}). Extraire le centroïde de chaque groupe
    risque d'avoir un effet semblable, car \og réseau~\fg\ est le sous-composant
    de nombreux autres candidats du sujet:
    \og réseau étendu~\fg\ (\textit{wide area network}),
    \og réseaux locaux~\fg\ (\textit{local area networks}),-
    \og réseaux informatisés de communication~\fg\ (\textit{computer-communication networks}),
    etc.

  % Que donne l'extraction ? (exemple)
  La figure~\ref{fig:exemple_topicrank} donne un exemple d'extraction de
  termes-clés avec TopicRank. Nous observons un groupement correct de toutes les
  variantes d'\og alertes~\fg, mais aussi un groupement erroné de \og août
  2003~\fg\ avec \og août 2012~\fg. Dans ce dernier cas, TopicRank est tout de
  même capable d'extraire \og août 2012~\fg, grâce à la sélection du candidat
  apparaissant en premier. Globalement, l'extraction des termes-clés est
  correcte et huit termes-clés sur les dix extraits ont aussi été donnés par des
  humains.
  \begin{figure}
    \centering
    \begin{tikzpicture}
      \node[style={rectangle,
                   rounded corners,
                   draw=black,
                   thick,
                   inner sep=2pt}](document){
        \begin{varwidth}{.98\linewidth}
          \small
          \textbf{Météo du 19 août 2012~: alerte à la canicule sur la Belgique
                  et le Luxembourg}

          \indent{
            A l'exception de la province de Luxembourg, en alerte jaune,
            l'ensemble de la Belgique est en vigilance orange à la canicule. Le
            Luxembourg n'est pas épargné par la vague du chaleur : le nord du
            pays est en alerte orange, tandis que le sud a était placé en
            alerte rouge.
          }

          \indent{
            En Belgique, la température n'est pas descendue en dessous des
            23\degre{}C cette nuit, ce qui constitue la deuxième nuit la plus
            chaude jamais enregistrée dans le royaume. Il se pourrait que ce
            dimanche soit la journée la plus chaude de l'année. Les températures
            seront comprises entre 33 et 38\degre{}C. Une légère brise de
            côte pourra faiblement rafraichir l'atmosphère. Des orages de
            chaleur sont a prévoir dans la soirée et en début de nuit.
          }

          \indent{
            Au Luxembourg, le mercure devrait atteindre 32\degre{}C ce dimanche
            sur l'Oesling et jusqu'à 36\degre{}C sur le sud du pays, et 31 à
            32\degre{}C lundi. Une baisse devrait intervenir pour le reste de la
            semaine. Néanmoins, le record d'août 2003 (37,9\degre{}C) ne devrait
            pas être atteint.
          }
        \end{varwidth}
      };
    \end{tikzpicture}~\\
    \vspace{2em}
    \begin{overpic}[width=.8\linewidth]{include/44960_topicrank_empty.eps}
      \put(38,100.5){\small[soirée]}
      \put(22,96){\small[oesling]}
      \put(55,100){\small[nord]}
      \put(69,95){\small[belgique]}
      \put(0,86){\small[août 2003~; août 2012]}
      \put(83,83.5){\small[36\degre{}c]}
      \put(0,73){\small[37,9\degre{}c]}
      \put(89,70.5){\small[royaume]}
      \put(-5,57.5){\small[année]}
      \put(92.5,53.5){\small[dimanche]}
      \put(-11,41){\small[vigilance orange]}
      \put(88.5,38){\small[luxembourg]}
      \put(1.5,25){\small[début]}
      \put(77,23){\small[nuit~; deuxième nuit]}
      \put(11,13.5){\small[exception]}
      \put(72,10.5){\small[canicule]}
      \put(25,5){\small[journée]}
      \put(60,4.5){\small[côte]}
      \put(17,1){\small[alerte rouge~; alerte jaune~; alerte orange~; alerte]}
      \put(40,84){\small[ensemble]}
      \put(24,80){\small[légère brise]}
      \put(58,82.5){\small[record]}
      \put(15.5,68.5){\small[météo]}
      \put(69.5,72){\small[orages]}
      \put(11,53){\small[chaude]}
      \put(79,58){\small[baisse]}
      \put(10,37){\small[température]}
      \put(72,42){\small[atmosphère]}
      \put(26,26){\small[23\degre{}c]}
      \put(71,27){\small[sud]}
      \put(39,17){\small[lundi]}
      \put(55.5,20.5){\small[38\degre{}c]}
      \put(31,63){\small[chaleur]}
      \put(49.5,68){\small[reste]}
      \put(27,46){\small[province]}
      \put(62,56){\small[vague]}
      \put(41,34.5){\small[pays]}
      \put(56,39){\small[mercure]}
      \put(43,51){\small[semaine]}
    \end{overpic}~\\
    \vspace{1.5em}
    \begin{tikzpicture}
      \node[style={rectangle,
                   rounded corners,
                   draw=black,
                   thick,
                   inner sep=2pt}](termes_cles){
        \begin{varwidth}{\linewidth}
          \small
          \textbf{Termes-clés extraits par des humains~:}

          luxembourg~; alerte~; météo~; belgique~; août 2012~; chaleur~;
          température~; chaude~; canicule~; orange~; la plus chaude

          \textbf{Termes-clés extraits par TopicRank~:}

          luxembourg~; alerte~; nuit~; belgique~; août 2012~; chaleur~;
          température~; chaude~; canicule~; dimanche
        \end{varwidth}
      };
    \end{tikzpicture}

    \caption{Extraction des termes-clés du document \textit{44960} de WikiNews 
             (voir la section~\ref{sec:evaluation}), avec TopicRank.
             \label{fig:exemple_topicrank}}
  \end{figure}

