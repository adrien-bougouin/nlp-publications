\section{Linguistically Refined NPs (LR-NPs)}
\label{sec:proposed_candidate_selection_method}
    To get rid of unwanted modifiers within keyphrase candidates, we propose a set of heuristics drawn from the observations presented in Section~\ref{sec:keyphrase_properties}.
    
    First, we use more specific POS tag patterns to select candidates: \texttt{A?\,(Nc|Np)+} for English and \texttt{(Nc|Np)+\,A?} for French.
    These patterns ensure more consistency with Property~\ref{prop:informativity} and, in the case of French, filter out some attributive adjectives (left-side adjectives).
    
    Second, we further refine the selected keyphrase candidates by pruning modifiers according to the following rules.
    As relational adjectives often appear in reference keyphrases, we consider that they are always relevant modifiers and must be preserved in all candidates.
    We also accept every single-word compound modifier, such as an hyphenated modifier (e.g.~``graph-based'', ``data-driven'', ect.).
    Indeed, similar to relational adjectives, single-word compound modifiers introduce a relation with a noun.
    
    Other modifiers are in turn removed from the keyphrase candidates, except when they bring \textit{useful} information.
    To assess whether a modifier is useful or not, we simply compare the number of occurrences of a candidate with and without the modifier.
    If a keyphrase candidate occurs more often unmodified, we prune the candidate from the modifier (e.g.~``\textit{large storms}'' becomes ``\textit{storms}'').
    
    To detect relational adjectives, we use external resources: WordNet~\cite{miller1995wordnet} for English and WoNeF~\cite{pradet2013wonef} for French.
    %We consider an adjective to be relational if it has a pertainym in those resources, or if its suffix matches one of the known suffixes of relational adjectives\footnote{This simple detection technique requires few resources and is easy to adapt for new languages.}: ``al'', ``ant'', ``ary'', ``ic'', ``ous''  or ``ive'' for English~\cite{grabar2006terminologystructuring}; ``ain'', ``aire'', ``al'', ``el'', ``eux'', ``ien'', ``ier'', ``ique'' or ``ois'' for French~\cite{harastani2013relationaladjectivetranslation}.
    We consider an adjective to be relational if it has a pertainym in those resources, or if its suffix matches one of the known suffixes of relational adjectives: ``al'', ``ant'', ``ary'', ``ic'', ``ous''  or ``ive'' for English~\cite{grabar2006terminologystructuring}; ``ain'', ``aire'', ``al'', ``el'', ``eux'', ``ien'', ``ier'', ``ique'' or ``ois'' for French~\cite{harastani2013relationaladjectivetranslation}.
