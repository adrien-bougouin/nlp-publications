%
% File naaclhlt2015.tex
%

\documentclass[11pt,letterpaper]{article}
\usepackage{naaclhlt2015}
\usepackage{times}
\usepackage{latexsym}
%\setlength\titlebox{6.5cm} % Expanding the titlebox

\usepackage[utf8]{inputenc}
\usepackage{amsmath}
\usepackage{relsize}
\usepackage{xfrac}
\usepackage{booktabs}
\usepackage{multirow}
\usepackage{rotating}
\usepackage{lipsum}
\usepackage{color}
\usepackage{subfigure}
\usepackage{overpic}
\usepackage{fancybox}
\usepackage[ruled,vlined]{algorithm2e}
\usepackage{url}
\usepackage{graphicx}

\setlength{\textfloatsep}{0em}
\setlength{\floatsep}{0em}
\setlength{\dbltextfloatsep}{0em}
\setlength{\dblfloatsep}{0em}
\setlength{\abovecaptionskip}{.33em}
\setlength{\belowcaptionskip}{.44em}

\newtheorem{property}{Property}
\newcommand\TODO[1]{}%\textcolor{red}{[TODO #1]}}

\title{
    Linguistically Motivated Keyphrase Candidate Selection
    %Selecting Keyphrase Candidates using Relational Adjectives
    % \Thanks{
    %     The authors would like to thank the anonymous reviewers for their useful
    %     advice and comments. This work was supported by the French National Research
    %     Agency (TermITH project -- ANR-12-CORD-0029).
    % }
}

% \author{First Author \\
%   Affiliation \\
%   {\tt email@domain} \\\And
%   Second Author \\
%   Affiliation \\
%   {\tt email@domain} \\}

% \author{
%     Adrien Bougouin \and Florian Boudin \and Béatrice Daille\\
%     Université de Nantes, LINA, France\\
%     \normalsize\texttt{\{adrien.bougouin,florian.boudin,beatrice.daille\}@univ-nantes.fr
% }

\date{}

\begin{document}
    \maketitle
    
    \begin{abstract}
        % This paper analyses human-assigned keyphrases and identifies their linguistic properties.
        % We use these properties to propose a new method for selecting keyphrase candidates.
        % Through both intrinsic and extrinsic evaluations on three datasets, we show that a linguistically motivated selection of keyphrase candidates improves the keyphrase extraction performance.
        Keyphrase candidate selection is the task of producing a set of potential keyphrases.
        This work analyses human-assigned keyphrases and identifies their linguistic properties.
        We use these properties to select keyphrase candidates.
        Through both intrinsic and extrinsic comparisons with current candidate selection methods, we show that a
        linguistically motivated selection of keyphrase candidates improves the keyphrase extraction performance.
    \end{abstract}
    
    
    \section{Introduction}
\label{sec:section}
    Keyphrases are words or phrases that represent the main content of a document.
    Similar to an abstract, keyphrases give a synoptic picture of what is important in the document.
    Disimilar to an abstract, keyphrases are small grain units and are useful resources for many Natural Language Processing tasks: document clustering~\cite{han2007webdocumentclustering}, information retrieval~\cite{medelyan2008smalltrainingset}, document summarization~\cite{litvak2008graphbased}, etc.
    However, documents do not always contain keyphrases.
    As the daily flow of new documents grows, manually annotating documents has become impractical.
    Hence automatic keyphrase extraction recently attracts a lot of attention and many different methods are proposed~\cite{hasan2014state_of_the_art}.

    Automatic keyphrase extraction is the task of detecting important words or phrases within a document.
    Generally speaking, we divide keyphrase extraction methods into two categories: supervised and unsupervised.
    Supervised methods treat keyphrase extraction as a binary classification task, e.g.~\cite{witten1999kea}.
    Conversely, unsupervised methods usually rank keyphrase candidates by importance and select the top-ranked ones as keyphrases, e.g.~\cite{mihalcea2004textrank}.

    Although they tackle the keyphrase extraction problem differently, both supervised and unsupervised methods rely on a candidate selection step.
    Keyphrase candidate selection identifies words or phrases consistent with human-assigned keyphrase properties.
    %Although keyphrase candidate selection starts to draw attention~\cite{wang2014keyphraseextractionpreprocessing}, keyphrase extraction methods use simple heuristics: selection of n-grams, sequences of nouns and adjectives, etc.
    However, current selection methods use simple heuristics~\cite{wang2014keyphraseextractionpreprocessing}: candidates are n-grams or sequences of nouns and adjectives.
    %This work infers linguistic properties from human-assigned keyphrases and demonstrates their applicability on keyphrase candidate selection.
    This work proposes rules based on a comprehensive analysis of modifiers within human-assigned keyphrases.
    We demonstrate their applicability on keyphrase candidate selection.
    
    This paper is organized as follows.
    Section~\ref{sec:keyphrase_properties} presents an analysis of human-assigned keyphrases.
    Section~\ref{sec:candidate_selection} describes common keyphrase candidate selection methods followed by a description of our method in Section~\ref{sec:proposed_candidate_selection_method}. Finally, Section~\ref{sec:experiments} presents the expriments and Section~\ref{sec:conclusion} concludes our work.

    \section{Keyphrase Properties}
\label{sec:keyphrase_properties}
    In this section, we infer keyphrase properties from human-assigned keyphrases.
    We use three standard datasets: DUC~\cite{wan2008expandrank}, SemEval~\cite{kim2010semeval} and DEFT~\cite{paroubek2012deft}.
    These datasets differ in terms of language, nature and document size (see Table~\ref{tab:datasets}), which makes our inferred properties more general.
    We avoid evaluation bias and only infer properties from train documents, while test documents are used for evaluation purpose.
    \begin{table}[!h]
        \centering
        \begin{tabular}{@{}r@{~}|@{~}c@{~~}c@{~~}c@{}}
            \toprule
            & \textbf{DUC} & \textbf{SemEval} & \textbf{DEFT}\\
            \hline
            Language & English & English & French\\
            Nature & Journalistic & Scientific & Scientific\\
            Train documents & 208 & 144 & 141\\
            Test documents & 100 & 100 & ~~93\\
            \bottomrule
        \end{tabular}
        \caption{Caracteristics of DUC, SemEval and DEFT
                 \label{tab:datasets}}
    \end{table}

    Table~\ref{tab:train_dataset_statistics} shows statistics about the train documents.
    %Reference keyphrases were automatically Part-of-Speech (POS) tagged\footnote{We use the Stanford POS tagger~\cite{toutanova2003stanfordpostagger} for English and MElt~\cite{denis2009melt} for French.} and manually reviewed for consistency.
    Reference keyphrases were automatically Part-of-Speech (POS) tagged and manually reviewed for consistency.
    The bottom part of the table presents the percentage of multi-word keyphrases that contain a certain POS.
    We do not show single-word keyphrase statistics as they are mostly nouns.
    \vspace{-.25em}
    \begin{table}[!h]
        \centering
            \begin{tabular}{@{}r@{~}|@{~}c@{~~}c@{~~}c@{}}
                \toprule
                ~ & \textbf{DUC} & \textbf{SemEval} & \textbf{DEFT}\\
                \hline
                \multicolumn{1}{@{}l@{~}|@{~}}{\textbf{Documents:}}\\
                Tokens/doc. & 912.0 & 5134.6 & 7276.7\\
                Keyphrases/doc. & 8.1 & 15.4 & 5.4\\
                Maximum recall (\%) & 96.1 & 86.5 & 81.8\\
                \hline
                \multicolumn{1}{@{}l@{~}|@{~}}{\textbf{Keyphrases:}}\\
                Unigrams (\%) & 17.1 & 20.2 & 60.2\\
                Bigrams (\%) & 60.8 & 53.4 & 24.5\\
                Trigrams (\%) & 17.8 & 21.3 & ~~8.8\\
                \hline
                \multicolumn{1}{@{}l@{~}|@{~}}{\textbf{Multi-word keyphrases}}\\
                \multicolumn{1}{@{}l@{~}|@{~}}{\textbf{with:}\hfill{}Noun (\%)} & 94.5 & 98.7 & 93.1\\
                Proper noun (\%) & 17.1 & ~~4.3 & ~~6.9\\
                Attributive adj. (\%) & 24.2 & 29.1 & ~~8.6\\
                Relational adj. (\%) & 28.9 & 24.1 & 57.6\\
                Prep. (\%) & ~~0.3 & ~~1.5 & 31.2\\
                Det. (\%) & ~~0.0 & ~~0.0 & 20.4\\
                \bottomrule
            \end{tabular}
        \caption{Statistics of the train documents.
                 The maximum recall represents the percentage of keyphrases that can be extracted from the documents.
                 \label{tab:train_dataset_statistics}}
    \end{table}
    \vspace{-1.5em}
    
    First, we observe, as noted in previous work, that most keyphrases are small-sized textual units.
    \begin{property}\label{prop:informativity}
      Keyphrases are small-sized textual units, usually containing up to three words (e.g.~``storms'', ``hurricane expert'' or ``annual hurricane forecast'').
    \end{property}

    Second, we observe that most keyphrases contain a noun and more than half of them are modified by an adjective.
    Most importantly, among these adjectives, there is a larger number of relational adjectives.
    This is also confirmed by the presence of relational adjectives in the most frequent POS tag patterns of keyphrases, as shown in Table~\ref{tab:best_patterns}.
    \begin{property}\label{prop:noun_phrases}
      Keyphrases are noun sequences (e.g.~``storms'') modified or not, most likely by a relational adjective (e.g.~``annual hurricane forecast'').
    \end{property}
    \vspace{-.5em}
    \begin{table}[!h]
        \centering
        \begin{tabular}{@{}r@{~}|@{~}l@{~}l@{~}l@{~}ll@{}}
            \toprule
            \multicolumn{1}{r}{} & \multicolumn{4}{@{}l}{\textbf{Pattern}} & \textbf{Example}\\
            \midrule
            \multirow{3}{*}{\begin{sideways}\textbf{English}\end{sideways}}
            & \texttt{Nc} & \texttt{Nc} & & & \textit{``hurricane expert''}\\ % AP880409-0015
            & \texttt{Nc} & & & & \textit{``storms''}\\ % AP880409-0015
            & \texttt{rA} & \texttt{Nc} & & & \textit{``Chinese earthquake''}\\ % AP890228-0019
            \hline
            \multirow{3}{*}{\begin{sideways}\textbf{French}\end{sideways}}
            & \texttt{Nc} & & & & \textit{``patrimoine'' (``cultural heritage'')}\\ % as_2002_007048ar
            & \texttt{Nc} & \texttt{rA} & & & \textit{``tradition orale'' (``oral tradition'')}\\ % as_2002_007048ar
            & \texttt{Np} & & & & \textit{``Indonésie'' (``Indonesia'')}\\ % as_2001_000235ar
            \bottomrule
      \end{tabular}
      \caption{Most frequent patterns -- Multex format~\cite{ide1994multext}.
               \texttt{rA} stands for \textit{relational adjective}.
               \label{tab:best_patterns}}
    \end{table}
    \vspace{-.5em}
    
    Unlike most attributive adjectives, relational adjectives express a relation with a noun (e.g.~``cultural'' is derived from ``culture'').
    Due to this denominal property, relational adjectives are highly used as taxonomic classes (e.g. the Wikipedia category \textit{cultural heritage}), which can be seen as higher-level keyphrases~\cite{fatima2011automaticdocumentannotation}.
    \TODO{They are also used in other tasks such as term extraction~\cite{daille2001relationaladjectives}.}
    
    Although Property~\ref{prop:informativity} is well-known in automatic keyphrase extraction, Property~\ref{prop:noun_phrases} has only been partially covered in previous work.
    Indeed, keyphrases are known to be mostly noun phrases.
    However, no comprehensive analysis of the keyphrase modifiers has been conducted so far, showing how relational and attributive adjectives are used in keyphrases.
    We use these findings to devise a new method that filters out irrelevant modifiers from keyphrase candidates.

    \section{Candidate Selection}
\label{sec:candidate_selection}
    Previous work commonly selects candidates of either one of the following types:
    \begin{itemize}
        \item{\textbf{N-gram:} Ordered sequence of $n$ word(s), with $n$ usually set to one up to three (see Property~\ref{prop:informativity}).
              To avoid selecting an irrelevant keyphrase candidate, an n-gram containing a stopword is pruned~\cite{witten1999kea}.}
          
        \item{\textbf{POS sequence:} Word or phrase matching a given POS tag pattern. A simple, yet efficient POS pattern represents a longest sequence of nouns and adjectives~\cite{bougouin2013topicrank}.}
        \item{\textbf{NP-Chunk:} Non-recursive (minimal) noun-phrase (NP). An
              NP-chunk can be detected by pattern matching. In our experiments, we use the following POS patterns for English and French, respectively:}
            \begin{itemize}
                \item{\texttt{Np+|(A+~Nc)|Nc+}}
                \item{\texttt{Np+|(A?~Nc~A+)|(A~Nc)|Nc+}}
            \end{itemize}
    \end{itemize}
  
    These candidates are not fully consistent with properties inferred in Section~\ref{sec:keyphrase_properties}.
    Selecting n-grams respects Property~\ref{prop:informativity}, but produces candidates that contain verbs, adverbs and other words inconsistent with Property~\ref{prop:noun_phrases}.
    Conversely, selecting the longest sequences of nouns and adjectives (longest NPs) satisfies Property~\ref{prop:noun_phrases} but not Property~\ref{prop:informativity} as the length of the selected candidates is not controlled.
    Selecting NP-chunks is consistent with both properties, but it does not consider the nature of the candidate modifiers.
    We thus propose to refine this latter selection method by filtering out irrelevant modifiers.
  
    \section{Linguistically Refined NPs (LR-NPs)}
\label{sec:proposed_candidate_selection_method}
    To get rid of unwanted modifiers within keyphrase candidates, we propose a set of heuristics drawn from the observations presented in Section~\ref{sec:keyphrase_properties}.
    
    First, we use more specific POS tag patterns to select candidates: \texttt{A?\,(Nc|Np)+} for English and \texttt{(Nc|Np)+\,A?} for French.
    These patterns ensure more consistency with Property~\ref{prop:informativity} and, in the case of French, filter out some attributive adjectives (left-side adjectives).
    
    Second, we further refine the selected keyphrase candidates by pruning modifiers according to the following rules.
    As relational adjectives often appear in reference keyphrases, we consider that they are always relevant modifiers and must be preserved in all candidates.
    We also accept every single-word compound modifier, such as an hyphenated modifier (e.g.~``graph-based'', ``data-driven'', ect.).
    Indeed, similar to relational adjectives, single-word compound modifiers introduce a relation with a noun.
    
    Other modifiers are in turn removed from the keyphrase candidates, except when they bring \textit{useful} information.
    To assess whether a modifier is useful or not, we simply compare the number of occurrences of a candidate with and without the modifier.
    If a keyphrase candidate occurs more often unmodified, we prune the candidate from the modifier (e.g.~``\textit{large storms}'' becomes ``\textit{storms}'').
    
    To detect relational adjectives, we use external resources: WordNet~\cite{miller1995wordnet} for English and WoNeF~\cite{pradet2013wonef} for French.
    %We consider an adjective to be relational if it has a pertainym in those resources, or if its suffix matches one of the known suffixes of relational adjectives\footnote{This simple detection technique requires few resources and is easy to adapt for new languages.}: ``al'', ``ant'', ``ary'', ``ic'', ``ous''  or ``ive'' for English~\cite{grabar2006terminologystructuring}; ``ain'', ``aire'', ``al'', ``el'', ``eux'', ``ien'', ``ier'', ``ique'' or ``ois'' for French~\cite{harastani2013relationaladjectivetranslation}.
    We consider an adjective to be relational if it has a pertainym in those resources, or if its suffix matches one of the known suffixes of relational adjectives: ``al'', ``ant'', ``ary'', ``ic'', ``ous''  or ``ive'' for English~\cite{grabar2006terminologystructuring}; ``ain'', ``aire'', ``al'', ``el'', ``eux'', ``ien'', ``ier'', ``ique'' or ``ois'' for French~\cite{harastani2013relationaladjectivetranslation}.

    \section{Experiments}
\label{sec:experiments}
    We validate the effectiveness of our proposed candidate selection method by using two series of experiments.
    First, we provide a qualitative evaluation of the keyphrase candidates produced by our method and perform a comparison with the other methods.
    Second, we conduct an end-to-end evaluation by exploiting two keyphrase extraction systems.
    
    \subsection{Experimental settings}
    \label{subsec:experimental_settings}
        To quantify the capacity of the candidate selection methods to provide suitable candidates and avoid irrelevant ones, we compute the number of selected candidates (Cand./Doc.) and confront it with the best possible performance (maximum recall~--~R$_{\text{max}}$).
        To do so, we compute a quality ratio (QR):
        \begin{align}
            \text{QR} &= \frac{\text{R$_{\text{max}}$}}{\text{Cand./Doc.}} \times 100
        \end{align}
        The higher is QR, the better.

        \begin{table*}
            \centering
            \begin{tabular}{r|ccc|ccc|ccc}
                \toprule
                \multirow{2}{*}[-2pt]{\textbf{Method}} & \multicolumn{3}{c|}{\textbf{DUC} (\textit{English})} & \multicolumn{3}{c|}{\textbf{SemEval} (\textit{English})} & \multicolumn{3}{c}{\textbf{DEFT} (\textit{French})}\\
                \cline{2-10}
                & Cand./Doc. & R$_{\text{max}}$ & QR & Cand./Doc. & R$_{\text{max}}$ & QR & Cand./Doc. & R$_{\text{max}}$ & QR\\
                \hline
                \{1..3\}-grams & $~~~$596.2 & \textbf{90.8} & 15.2 & 2580.5 & \textbf{72.2} & $~~$2.8 & 4070.2 & \textbf{74.1} & $~~~$1.8\\
                Longest NPs & $~~~$155.6 & 88.7 & 57.0 & $~~~$646.5 & 62.4 & $~~$9.7 & $~~~$914.5 & 61.1 & $~~$6.7\\
                NP-chunks & $~~~$149.9 & 76.0 & 50.7 & $~~~$598.4 & 56.6 & $~~$9.5 & $~~~$812.3 & 63.0 & $~~$7.8\\
                LR-NPs & \textbf{$~~~$143.8} & 85.3 & \textbf{59.3} & \textbf{$~~~$538.2} & 59.4 & \textbf{11.0} & \textbf{$~~~$738.2} & 60.1 & \textbf{$~~$8.1}\\
                \bottomrule
            \end{tabular}
            \caption{Qualitative comparison of the keyphrase candidate selection methods
                     \label{tab:candidate_extraction_statistics}}
        \end{table*}

        \begin{table*}
            \centering
            \resizebox{\linewidth}{!}{
            \begin{tabular}{r@{~}|c@{~~}c@{~~}c@{~}|@{~}c@{~~}c@{~~}c@{~}|@{~}c@{~~}c@{~~}c@{~}|@{~}c@{~~}c@{~~}c@{~}|@{~}c@{~~}c@{~~}c@{~}|@{~}c@{~~}c@{~~}c}
                \toprule
                \multirow{2}{*}[-2pt]{\textbf{Method}} & \multicolumn{6}{c@{~}|@{~}}{\textbf{DUC} (\textit{English})} & \multicolumn{6}{c@{~}|@{~}}{\textbf{SemEval} (\textit{English})} & \multicolumn{6}{c}{\textbf{DEFT} (\textit{French})}\\
                \cline{2-19}
                & \multicolumn{3}{c@{~}|@{~}}{TF-IDF} & \multicolumn{3}{c@{~}|@{~}}{KEA} & \multicolumn{3}{c@{~}|@{~}}{TF-IDF} & \multicolumn{3}{c@{~}|@{~}}{KEA} & \multicolumn{3}{c@{~}|@{~}}{TF-IDF} & \multicolumn{3}{c}{KEA}\\
                \cline{2-19}
                & P & R & F & P & R & F & P & R & F & P & R & F & P & R & F & P & R & F\\
                \hline
                \{1..3\}-grams & 14.3 & 19.0 & 16.1$~~$ & 12.0 & 16.6 & 13.7$~~$ & $~~$9.0 & $~~$6.6 & $~~$7.2$~~$ & 19.4 & 13.7 & 15.9 & $~~$6.7 & 12.5 & $~~$8.6 & 13.4 & 25.3 & 17.3\\
                Longest NPs & 24.2 & 31.7 & 27.0$~~$ & \textbf{14.5} & 19.9 & 16.5$~~$ & 11.7 & $~~$7.9 & $~~$9.3$~~$ & 19.6 & 13.7 & 16.0 & $~~$9.5 & 17.6 & 12.1 & 14.1 & 26.3  &18.1\\
                NP-chunks & 21.1 & 28.1 & 23.8$~~$ & 13.5 & 18.6 & 15.4$~~$ & 11.9 & $~~$8.0 & $~~$9.5$~~$ & 19.5 & 13.7 & 16.0 & $~~$9.6 & 17.9 & 12.3 & 14.3 & 26.8 & 18.4\\
                LR-NPs & \textbf{24.3} & \textbf{32.0} & \textbf{27.2$^\dagger$} & \textbf{14.5} & \textbf{20.0} & \textbf{16.6$^\ddagger$} & \textbf{12.4} & \textbf{$~~$8.4} & \textbf{$~~$9.9$^\ddagger$} & \textbf{20.4} & \textbf{14.4} & \textbf{16.7}& \textbf{10.1} & \textbf{18.5} & \textbf{12.9} & \textbf{14.4} & \textbf{27.0} & \textbf{18.6}\\
                \bottomrule
            \end{tabular}
            }
            \caption{Comparison of TF-IDF and KEA applied on top of different candidate selection methods.
            $\ddagger$ indicates a significant improvement overall candidate sets and $\dagger$ indicates a significant improvement overall candidate sets but the longest NPs at 0.001 level using Student's t-test.
                     \label{tab:keyphrase_extraction_results}}
        \end{table*}
        
        We measure the impact of each candidate selection method on the keyphrase extraction task using the two following unsupervised and supervised keyphrase extraction methods:
        \begin{itemize}
            \item{\textbf{TF-IDF~\cite{jones1972tfidf}:} Word significancy weighting scheme.
                  Words are weighted based on their frequency in the document and the inverse number of documents in which they appear (specificity);
                  Candidates are weighted using the sum of  their words' score.}
            \item{\textbf{KEA~\cite{witten1999kea}:} Naive Bayes classifier trained on two features: the TF-IDF\footnote{KEA computes a TF-IDF based on candidate frequency, whereas our TF-IDF baseline relies on word frequency. KEA's TF-IDF is more efficient on larger documents than smaller ones.} and the first position of each candidate selected within train documents.}
        \end{itemize}
        We report the performance of TF-IDF and KEA in terms of precision~(P), 
        recall~(R) and F1-measure (F) at the top 10 keyphrases.
        Candidate and reference keyphrases are stemmed to reduce the number 
        of mismatches.
    
    \subsection{Candidate selection evaluation}
    \label{subsec:candidate_extraction_evaluation}
        Table~\ref{tab:candidate_extraction_statistics} presents the results of the intrinsic evaluation of the candidate selection methods.
        %Unsurprisingly, the best maximum recall is achieved by the $\{1..3\}$-grams selection method, but at the cost of a huge number of unrelevant candidates as indicated by the low quality ratio.
        Unsurprisingly, the best maximum recall is achieved when selecting $\{1..3\}$-grams, but at the cost of a huge number of irrelevant candidates as indicated by the low QR.
        Among the other selection methods, our method shows a competitive maximum recall while reducing the number of candidates.
        As a consequence, the LR-NPs quality outperforms other selected candidates quality, which is crucial as it directly affects the performance and time complexity of keyphrase extraction methods~\cite{wang2014keyphraseextractionpreprocessing}.
        
        \TODO{examples of true and false positives}
    
    \subsection{Keyphrase extraction evaluation}
    \label{subsec:keyphrase_extraction_evaluation}
        Table~\ref{tab:keyphrase_extraction_results} shows the results of the extrinsic evaluation of the candidate selection methods.
        Overall, we observe that the performance of TF-IDF and KEA is closely correlated with the quality of the set of selected candidates.
        Best results are then obtained when TF-IDF and KEA are applied on LR-NPs (half of them are significantly better).
        Thus, although our proposed selection method does not achieve the best maximum recall, it still outperforms the other candidate selection  methods.
        Comparing longest NPs, NP-chunks and LR-NPs demonstrates that it is efficient to use heuristic based on linguistic properties.

    \section{Conclusion et perspectives}
\label{sec:conclusion_et_perspectives}
  Dans ce travail, nous proposons une méthode à base de graphe pour l'extraction
  non supervisée de termes-clés. Cette méthode groupe les termes-clés candidats
  en sujets, détermine quels sont ceux les plus importants, puis extrait le
  terme-clé candidat qui représente le mieux chacun des sujets les plus
  importants. Cette nouvelle méthode offre plusieurs avantages vis-à-vis des
  précédentes à base de graphe. Le groupement des termes-clés potentiels en
  sujets distincts permet de rassembler des indices utiles auparavant éparpillés
  et le choix d'un seul terme-clé pour représenter un sujet important permet
  d'extraire un ensemble de termes-clés non redondants ( pour $k$ termes-clés
  extraits, exactement $k$ sujets sont couverts). Enfin, le graphe est complet
  et ne requiert plus le paramétrage d'une fenêtre de cooccurrences,
  contrairement aux autres méthodes à base de graphe.

  Les bons résultats de notre méthode montrent la pertinence d'un groupement en
  sujets des candidats pour ensuite les ordonner. Les expériences
  supplémentaires montrent aussi qu'il est encore possible d'améliorer notre
  méthode en proposant une nouvelle stratégie de sélection du terme-clé candidat
  le plus représentatif d'un sujet (pour un gain maximum allant de 4,2 à 15
  points de f-score).

  Nous avons aussi effectué une analyse d'erreurs à partir de laquelle trois
  perspectives de travaux futurs émergent~:

  Nous avons pour objectif d'améliorer la sélection des termes-clés candidats.
  Aussi, des méthodes empruntées à d'autres domaines du TAL peuvent être
  appliquées. Il semble, par exemple, pertinent d'évaluer l'apport des méthodes
  d'extraction terminologiques~\cite{castellvi2001automatictermdetection} pour
  la sélection des termes-clés candidats.
  
  Nous envisageons également d'améliorer le groupement en sujets,
  car celui-ci est très naïf et ne tient compte ni de la synonymie, ni de
  l'ambiguïté des mots. De plus, l'usage du
  radical~\cite{porter1980suffixstripping} des mots n'est pas sans introduire du
  bruit lié à certains faux positifs (p.~ex. \og{}\underline{empir}e\fg{} et
  \og{}\underline{empir}ique\fg{}). L'ajout de connaissances concernant les
  synonymes permettrait de créer des sujets plus complets et une étape de
  désambiguïsation éviterait un groupement systématique des termes-clés
  candidats ayant un ou plusieurs mots en commun. Nous envisageons aussi de
  remplacer la racinisation de \newcite{porter1980suffixstripping} par une
  méthode de lemmatisation. D'un point de vue plus technique, il faudrait
  explorer différentes méthodes de groupement, dont le groupement spectral
  (\textit{spectral clustering}) qui, dans d'autres travaux portant sur
  l'extraction automatique de termes-clés~\cite{liu2009keycluster}, montre de
  meilleures performances que le groupement hiérarchique agglomératif.

  Enfin, une étude détaillée des caractéristiques des termes-clés pourrait
  orienter notre travail vers des critères plus efficaces pour la définition
  d'une stratégie \og{}optimale\fg{} de sélection du terme-clé le plus
  représentatif d'un sujet. Un apprentissage supervisé à partir de certains
  critères est aussi envisagé, au même titre que l'usage de méthodes
  d'optimisation, telles que celle utilisée par
  \newcite{ding2011binaryintegerprogramming} dans leur méthode d'extraction
  automatique de termes-clés.


    
    % \section*{Acknowledgements}
    %     The authors would like to thank the anonymous reviewers for their useful
    %     advice and comments. This work was supported by the French National Research
    %     Agency (TermITH project -- ANR-12-CORD-0029).
    
    \bibliographystyle{naaclhlt2015}
    \bibliography{../../biblio}
\end{document}
