  \chapter{Contexte et problématique}
    \section{Contexte}
      Avec l'essort du numérique, le Web occupe aujourd'hui une place importante
      dans notre société. Il contient toutes sortes d'informations (culturelles,
      historiques, scientifiques, etc.) et les rend disponibles pour tous.
      Cependant, le Web est en constante expansion et le nombre croissant
      d'informations disponibles compliquent leur accès. Pour résoudre ce
      problème, il nous faut nous donner les moyens de représenter et
      d'organiser efficacement les documents numériques. Dans le contexte de la
      recherche scientifique, il est d'autant plus important de résoudre ce
      problème que la recherche a un impact sur le développement des pays.
      Favoriser l'accès aux productions scientifiques, que ce soit au niveau
      mondial ou national, favorise les avancées scientifiques et contribu donc
      au développement des pays. C'est dans cette objectif que travail la
      Bibliothèque Scientifique Numérique (BSN).

    \section{Problématique}

