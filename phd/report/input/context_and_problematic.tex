  \chapter{Contexte et problématique}
    \section{Contexte}
      %% Expression du besoin
      Avec l'essor du numérique, le Web occupe aujourd'hui une place importante
      dans notre société. Celui-ci contient tous types d'informations
      (culturelles, historiques, scientifiques, etc.) qu'il rend disponibles
      pour tous. Cependant, le Web est en constante expansion et le nombre
      croissant d'informations disponibles compliquent leur accès, leur
      recherche. Pour résoudre ce problème, il faut représenter et organiser
      efficacement les documents numériques. Dans le contexte de la recherche
      scientifique, résoudre ce problème est d'autant plus important qu'elle a
      un impact sur le développement des pays. Favoriser l'accès aux productions
      scientifiques, que ce soit au niveau national ou international, favorise
      les avancées scientifiques et contribue donc au développement des pays.
      C'est dans ce but que sont créées les bibliothèques numériques telles que
      la Bibliothèque Scientifique Numérique (\textsc{Bsn}) fondée en 2009 par
      le ministère de l'enseignement supérieur et de la recherche français.

      %% Réponse à ce besoin (introduction de la notion de termes-clés)
      Afin de mieux comprendre comment l'accès aux informations peut être
      facilité, prenons l'exemple de l'Institut de l'Information Scientifique et
      Technique (Inist) dont les activités s'organisent aujourd'hui dans le
      cadre de la \textsc{Bsn}. Créé en 1988, l'Inist possède l'une des plus
      importantes collections de publications scientifiques d'Europe et fournit
      plusieurs services pour la recherche d'information, dont le maintient de
      bases de données bibliographiques. Ces bases de données, organisées par
      thématiques (p.~ex.~la base Francis pour les domaines des lettres et des
      sciences humaines et sociales~---~\textsc{Shs}), contiennent des notices
      bibliographiques qui décrivent le contenu de chaque production
      scientifique. Ce sont les descriptions de ces notices bibliographiques qui
      facilitent la recherche d'information, en permettant d'établir une
      correspondance entre un besoin d'un utilisateur (exprimé sous la forme
      d'une requête textuelle rédigée dans le langage naturel) et certains
      documents. Elles incluent des connaissances explicitées dans les
      documents, telles que le titre et les auteurs, mais aussi des
      connaissances qui ne le sont pas toujours~: le résumé et les mots-clés
      (concepts qui caractérisent un document), que nous décidons d'appeler
      termes-clés afin d'éviter toute ambiguïté sur leur nature~---~monolexicale
      ou polylexicale. Lorsque les résumés ne sont pas fournis par les auteurs
      des documents, ce sont des ingénieurs documentalistes (indexeurs
      professionnels) qui les rédigent. Quand aux termes-clés, qu'ils soient
      fournis ou non par les auteurs, les indexeurs professionnels en
      fournissent toujours de nouveaux, cela afin que les notices soient
      homogènes (le même terme-clé pour représenter le même concept) et aussi
      afin d'assurer que le vocabulaire utilisé soit, autant que possible, issu
      du vocabulaire de la discipline des documents.

      %% Besoin d'aller vers une solution automatique (surcharge => travail
      %% bâclé)
      L'extraction manuelle des termes-clés des documents est une tâche coûteuse
      et chronophage. Pour reprendre l'exemple de l'Inist, le flux d'entrée des
      notices est parfois trop important et, en réponse à cette surcharge, soit
      certaines notices ne sont pas indexées avec des termes-clés d'un ingénieur
      documentaliste, soit le temps accordé pour chaque notice est réduit, au
      risque de dégrader la qualité du travail d'indexation (p. ex. l'indexeur
      se concentre sur le titre et/ou le résumé et néglige le reste du
      document). Conscients de ce problème, que ce soit pour des articles
      scientifiques ou des documents d'autres natures, de nombreux chercheurs
      s'intéressent à la tâche d'extraction automatique de termes-clés, en
      témoigne le nombre grandissant d'articles scientifiques à ce
      sujet~\citep{hasan2014state_of_the_art} ainsi que l'émergence de campagnes
      d'évaluation des méthodes d'extraction automatique de
      termes-clés~\citep{kim2010semeval,paroubek2012deft}.

      %% Définition et types d'indexations
      L'extraction automatique de termes-clés consiste à extraire du contenu
      d'un document les concepts qui y sont importants, c'est-à-dire qui le
      caractérisent le mieux. Dans un article d'archéologie, par exemple, des
      termes-clés valides peuvent concerner le type de travail (fouille, mise en
      valeur d'artefacts, etc.), des données géographiques (pays, régions,
      etc.), des données chronologiques (années, périodes, sous-périodes, etc.)
      ou encore des données religieuses (dieux, cultes, etc.). Il existe deux
      types d'indexation pouvant être réalisés par la tâche d'extraction de
      termes-clés~: l'indexation libre et l'indexation
      contrôlée~\citep{paroubek2012deft}. La première consiste à assigner des
      termes-clés sans aucune contrainte, alors que la seconde consiste à
      assigner des termes-clés contraints par un vocabulaire (une terminologie)
      spécifique au domaine des documents qui sont traités. Pour ces deux
      indexations, nous observons aussi le phénomène d'indexation silencieuse.
      Cette indexation, difficile à reproduire automatiquement, fait apparaître
      des termes-clés qui ne sont pas présents dans le document auquel ils sont
      assignés. Il peut s'agir de reformulations d'expressions utilisées dans le
      document ou de concepts plus généraux décrivant une catégorie à laquelle
      appartient le document.

      %% Fonctionnement des méthodes d'extraction automatique de termes-clés
      Dans la littérature, nous observons un comportement commun à toutes les
      méthodes~: (1) prétraitement du document (segmentation en phrases,
      segmentation en mots, étiquetage morphosyntaxique des mots, etc.), (2)
      sélection des termes-clés candidats à partir du document prétraité, (3)
      analyse des termes-clés candidats (ordonnancement ou classification) et
      (4) sélection de $k$ candidats en tant que termes-clés, selon le résultat
      de leur analyse. Nous observons aussi deux catégories de méthodes
      d'extraction automatique de termes-clés~: les méthodes supervisées et les
      méthodes non-supervisées. Les premières réduisent la tâche d'extraction de
      termes-clés à une tâche de classification binaire~\citep{witten1999kea}.
      Entraînées à partir de collections de documents annotés en termes-clés,
      celles-ci classent les termes-clés candidats en tant que
      \textit{terme-clé} ou \textit{non terme-clé}. Quant aux secondes, elles
      ordonnent généralement les termes-clés candidats selon leur importance
      dans le document~\citep{wan2008expandrank}. Les méthodes supervisées sont
      en général plus performantes que les méthodes non-supervisées. Cependant,
      la faible quantité de données d'apprentissage disponibles, couplée à la
      forte dépendance des méthodes supervisées vis-à-vis du domaine des
      documents d'apprentissages, poussent les chercheurs à s'intéresser de plus
      en plus aux méthodes non-supervisées~\citep{hassan2010conundrums}.

    \section{Problématique}
      %% Cadre de la thèse
      Dans le cadre du projet \textsc{Anr} Termith\footnote{Terminologie et
      Indexation de Textes en sciences Humaines~:
      \url{http://www.atilf.fr/ressources/termith/}.}, en partenariat avec
      l'Atilf\footnote{Analyse et Traitement Informatique de la Langue
      Française.}, l'Inist, le Lidilem\footnote{Linguistique et Didactique des
      Langues Étrangères et Maternelles.} et l'Inria\footnote{Institut National
      de Recherche en Informatique et en Automatique.} (Nancy et Saclay), nous
      nous intéressons à la problématique de l'extraction automatique de
      termes-clés. Notre but est d'automatiser le processus d'indexation des
      notices bibliographiques tel qu'il est réalisé par les ingénieurs
      documentalistes de l'Inist. À l'Inist, l'indexation est en priorité
      contrôlée, mais, dans le cas où des concepts importants d'un document
      n'ont pas de correspondance dans la terminologie utilisée (p. ex. un
      nouveau concept), l'usage de termes-clés libres est autorisé. Enfin, dans
      le but d'homogénéiser l'indexation des notices et de faciliter leur
      classification, l'indexation peut être silencieuse.

      %% Moyens
      Afin de répondre aux attentes de l'Inist, nous devons réaliser une méthode
      d'extraction automatique de termes-clés qui, dans un premier temps, se
      fonde sur les mots et expressions contenus dans le titre et le résumé des
      notices\footnote{\textcolor{red}{Et à terme dans le texte intégral des
      articles. Cependant, ceux-ci ne sont pas encore disponibles.}}, puis, dans
      un second temps, reformule certains termes-clés extraits (contraintes
      d'homogénéité et de respect du vocabulaire contrôlé) et en ajoute de
      nouveaux (déclenchés par la présence de certains termes-clés). Pour cela,
      nous disposons des ressources terminologiques de
      l'Inist\footnote{\textcolor{red}{Ce n'est pas encore le cas.}}
      (vocabulaires contrôlés) et de cinq collections de notices
      bibliographiques (indexées) correspondant à cinq disciplines scientifiques
      (archéologie, chimie, linguistique, psychologie et sciences de
      l'information).
      %Les collections sont divisées en un ensemble d'évaluation et un ensemble
      %d'apprentissage.

