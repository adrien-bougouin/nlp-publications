\section{Conclusion et perspectives}
  \begin{frame}{Conclusion}
    Extraction de termes-clés dans le contexte général
    \begin{itemize}
      \item{Ordonnancement des sujets du document}
      \item{Sans ressources}
      \item{Pas de redondance}
    \end{itemize}

    \vspace{1em}

    Ajout d'assignement de termes-clés en domaines de spécialité
    \begin{itemize}
      \item{Ordonnancement conjoint des sujets du document et termes-clés du
            domaine}
      \item{100~\% de couverture théorique}
    \end{itemize}
  \end{frame}

  \begin{frame}{Perspectives}
    TopicRank~:
    \begin{itemize}
      \item{Groupement en sujets}
      \begin{itemize}
        \item{Exploiter des ressources lexicales}
        \item{Utiliser le contexte sémantique des candidats}
      \end{itemize}
      \item{Choix du terme-clé d'un sujet}
      \begin{itemize}
        \item{S'inspirer des méthodes de classification}
        \item{Appliquer une méthode de titrage automatique}
      \end{itemize}
    \end{itemize}
  %\end{frame}

  \vspace{1em}

  %\begin{frame}{Perspectives}
    TopicCoRank~:
    \begin{itemize}
      \item{Choix des termes-clés du domaine}
      \begin{itemize}
        \item{Se restreindre à des documents similaire}
        \item{Utiliser un thésaurus distributionnel}
      \end{itemize}
      \item{Équilibrage des deux graphes}
      \begin{itemize}
        \item{Paramétrage différent de l'ordonnancement dans chaque graphe}
      \end{itemize}
    \end{itemize}
  \end{frame}

