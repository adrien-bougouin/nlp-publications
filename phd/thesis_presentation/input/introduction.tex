\section*{Introduction}
  \subsection*{Contexte}
    \begin{frame}{Introduction}\framesubtitle{Contexte}
  \visible<+->{
    Accès à l'information numérique en domaines de spécialité.
  }

  \begin{block}<+->{Information numérique}
    \begin{itemize}
      \item{Savoir~/~Connaissance}
      \begin{itemize}
        \item{Intellectuel}
        \item{Social}
      \end{itemize}
      \item{Stockée virtuellement}
      \begin{itemize}
        \item{document électronique $\neq$ document papier (physique)}
        \item{base de données électronique $\neq$ bibliothèque (physique)}
      \end{itemize}
    \end{itemize}
  \end{block}

  \begin{block}<+->{Domaine de spécialité}
    \begin{itemize}
      \item{Champ de connaissance particulier}
      \item{Possède un vocabulaire spécifique}
    \end{itemize}
  \end{block}
\end{frame}

\begin{frame}{Introduction}\framesubtitle{Contexte}
  Valorisation de l'information numérique à l'Inist.

  \begin{block}{Inist (Institut de l'information scientifique et technique)}
  \end{block}

  \begin{block}{Base de données bibliographiques}
  \end{block}
\end{frame}



  \subsection*{Problématique}
    \begin{frame}[t]{Introduction}\framesubtitle{Problématique}
  \begin{alertblock}{Limites}
    \vspace{-.5em}
    \begin{itemize}
      \item{Les termes-clés ne sont pas toujours fournis}
      \item{Les termes-clés ne sont pas toujours adaptés}
      \alt<2->{
        \begin{itemize}
          \item{Inist~:}
          \vspace{-1.5em}
          \begin{multicols}{3}
            \begin{itemize}
              \item{Mailhac}
              \item{Aude}
              \item{Mourrel-Ferrat}
              \item{Olonza}
              \item{Hérault}
              \item{céramique}
              \item{\underline{typologie}}
              \item{\underline{décor}}
              \item{chronologie}
              \item{diffusion}
              \item{production}
              \item{commerce}
              \item{répartition}
              \item{oppidum}
              \item{analyse}
              \item{fouille ancienne}
              \item{le Cayla}
              \item{micassé}
              \item{\underline{céramique non-} \underline{tournée}}
              \item{echange}
              \item{\underline{age du Fer}}
              \item{La Tène}
              \item{Europe}
              \item{France}
              \item{celtes}
              \item{distribution}
              \item{cartographie}
              \item{habitat}
              \item{site fortifié}
              \item{identification}
              \item{étude du matériel}
            \end{itemize}
          \end{multicols}
          \vspace{-.66em}
          \item{Revues.org~:}
          \vspace{-1.5em}
          \begin{multicols}{3}
            \begin{itemize}
              \item{\underline{typologie}}
              \item{\underline{décors}}
              \item{\underline{céramique non} \underline{tournée}}
              \item{\underline{âge du Fer}}
              \item{Protohistoire}
              \item{deuxième âge du Fer}
              \item{commerce et échanges}
              \item{Languedoc occidental}
            \end{itemize}
          \end{multicols}
          \vspace{-1em}
        \end{itemize}
      }{
      }
    \end{itemize}
  \end{alertblock}
\end{frame}

\begin{frame}{Introduction}\framesubtitle{Problématique}
  Comment identifier automatiquement les termes-clés d'un document~?
  \begin{enumerate}
    \item{Dans le contexte général}
    \item{En domaines de spécialité}
  \end{enumerate}

  \vspace{1em}

  \large\textbf{$\Rightarrow$ Indexation (automatique) par termes-clés}
\end{frame}



  \begin{frame}{Introduction}\framesubtitle{Exemple 1}
    \begin{exampleblock}{\small
      Météo du 19 \textbf{août 2012}~: \textbf{alerte} à la
      \textbf{canicule} sur la \textbf{Belgique} et le
      \textbf{Luxembourg}
    }\justifying\small
      ~~~À l'exception de la province de \textbf{Luxembourg}, en
      \textbf{alerte} jaune, l'ensemble de la \textbf{Belgique} est en
      vigilance \textbf{orange} à la \textbf{canicule}. Le
      \textbf{Luxembourg} n'est pas épargné par la vague du \textbf{chaleur}
      : le nord du pays est en \textbf{alerte} \textbf{orange}, tandis que
      le sud a était placé en \textbf{alerte} rouge.

      ~~~En \textbf{Belgique}, la \textbf{température} n'est pas descendue
      en dessous des 23\degre{}C cette nuit, ce qui constitue la deuxième nuit
      \textbf{la plus chaude} jamais enregistrée dans le royaume. Il se
      pourrait que ce dimanche soit la journée \textbf{la plus chaude} de
      l'année. Les \textbf{températures} seront comprises entre 33 et
      38\degre{}C. Une légère brise de côte pourra faiblement rafraichir
      l'atmosphère. Des orages de \textbf{chaleur} sont a prévoir dans la
      soirée et en début de nuit.

      ~~~Au \textbf{Luxembourg}, le mercure devrait atteindre 32\degre{}C ce
      dimanche sur l'Oesling et jusqu'à 36\degre{}C sur le sud du pays, et 31 à
      32\degre{}C lundi. Une baisse devrait intervenir pour le reste de la
      semaine. Néanmoins, le record d'août 2003 (37,9\degre{}C) ne devrait pas
      être atteint.

      \begin{exampleblock}{\small Termes-clés de référence}\justifying\small
        \textbf{Août 2012}~; \textbf{canicule}~;
        \textbf{Belgique}~; \textbf{Luxembourg}~; \textbf{alerte}~;
        \textbf{orange}~; \textbf{chaleur}~; \textbf{chaude}~;
        \textbf{température}~; \textbf{la plus chaude}
      \end{exampleblock}
    \end{exampleblock}
  \end{frame}

  \begin{frame}{Introduction}\framesubtitle{Exemple 2}
    \vspace{-.33em}
    \begin{exampleblock}{\small
      Étude préliminaire de la \textbf{céramique non tournée}
      \textbf{micacée} du bas Languedoc occidental~: \textbf{typologie},
      \textbf{chronologie} et aire de \textbf{diffusion}
    }\justifying\small
      ~~~L'étude présente une variété de \textbf{céramique non tournée} dont la
      \textbf{typologie} et l'analyse des \textbf{décors} permettent de
      l'identifier facilement. La nature de l'argile enrichie de mica donne un
      aspect pailleté à la pâte sur laquelle le \textbf{décor} effectué selon
      la méthode du brunissoir apparaît en traits brillant sur fond mat. Cette
      première approche se fonde sur deux séries issues de \textbf{fouilles
      anciennes} menées sur les \textbf{oppidums} \textbf{du Cayla} à
      \textbf{Mailhac} (\textbf{Aude}) et de \textbf{Mourrel-Ferrat} à
      \textbf{Olonzac} (\textbf{Hérault}). La carte de
      \textbf{répartition} fait état d'\textbf{échanges} ou de
      \textbf{commerce} à l'échelon macrorégional rarement mis en évidence pour
      de la \textbf{céramique non tournée}. S'il est difficile de statuer sur
      l'origine des \textbf{décors}, il semble que la \textbf{production}
      s'insère dans une ambiance celtisante. La \textbf{chronologie} de cette
      \textbf{production} se situe dans le deuxième \textbf{âge du Fer}. La
      fourchette proposée entre la fin du IV$^\text{e}$ et la fin du II$^\text{e}$
      s. av. J.-C. reste encore à préciser.

      \begin{exampleblock}{\small Termes-clés de référence}\justifying\small
        \textbf{Mailhac}~; \textbf{Aude}~; \textbf{Mourrel-Ferrat}~;
        \textbf{Olonzac}~; \textbf{Hérault}~; \textbf{céramique}~;
        \textbf{typologie}~; \textbf{décor}~; \textbf{chronologie}~;
        \textbf{diffusion}~; \textbf{production}~; \textbf{commerce}~;
        \textbf{répartition}~; \textbf{oppidum}~; \textbf{analyse}~;
        \textbf{fouille ancienne}~; \textbf{le Cayla}~;
        \textbf{micassé}~; \textbf{céramique non-tournée}~;
        \textbf{echange}~; \textbf{age du} \textbf{Fer}~; La Tène~;
        Europe~; France~; celtes~; distribution~; cartographie~; habitat~; site
        fortifié~; identification~; étude du matériel
      \end{exampleblock}
    \end{exampleblock}
  \end{frame}

