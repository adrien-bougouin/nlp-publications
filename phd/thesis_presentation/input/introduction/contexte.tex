\begin{frame}{Introduction}\framesubtitle{Contexte}
  \visible<+->{
    Accès à l'information numérique en domaines de spécialité.
  }

  \begin{block}<+->{Information numérique}
    \begin{itemize}
      \item{Savoir~/~Connaissance}
      \begin{itemize}
        \item{Intellectuel}
        \item{Social}
      \end{itemize}
      \item{Stockée virtuellement}
      \begin{itemize}
        \item{document électronique}
        \item{base de données électronique}
      \end{itemize}
    \end{itemize}
  \end{block}

  \begin{block}<+->{Domaine de spécialité}
    \begin{itemize}
      \item{Champ de connaissance particulier}
      \item{Possède un vocabulaire spécifique}
    \end{itemize}
  \end{block}
\end{frame}

\begin{frame}{Introduction}\framesubtitle{Contexte}
  \visible<+->{
    Valorisation de l'information numérique à l'Inist.
  }

  \begin{block}<+->{Inist (Institut de l'information scientifique et technique)}
    \begin{itemize}
      \item{Savoir faire documentaire}
      \item{Bases de données bibliographiques}
      \begin{itemize}
        \item{\textsc{Francis}~: Sciences humaines et sociales}
        \item{\textsc{Pascal}~: Sciences exactes}
      \end{itemize}
    \end{itemize}
  \end{block}

  \begin{block}<+->{Base de données bibliographiques}
    Collection de notices bibliographiques~:
    \begin{itemize}
      \item{Titre}
      \item{Auteur(s)}
      \item{Résumé}
      \item{Descripteurs~/~Termes-clés (mots-clés)}
    \end{itemize}
  \end{block}
\end{frame}

\begin{frame}{Introduction}\framesubtitle{Contexte}
  \begin{block}<+->{Termes-clés}
    \begin{itemize}
      \item{Unités textuelles (mots et expressions)}
      \item{Décrivent le contenu d'un document}
      \item{Utiles pour la Recherche d'information (\textsc{Ri})~:}
      \begin{itemize}
        \item{Indexation}
        \item{Expansion de requête}
      \end{itemize}
    \end{itemize}
  \end{block}

  \begin{exampleblock}<+->{TODO}
  \end{exampleblock}
\end{frame}

