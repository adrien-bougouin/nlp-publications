\begin{frame}{Introduction}\framesubtitle{Contexte}
  Accès à l'information numérique en domaines de spécialité.

  \begin{block}{Information numérique}
    \begin{itemize}
      \item{Savoir~/~Connaissance}
      \begin{itemize}
        \item{Social}
        \item{Intellectuel}
      \end{itemize}
      \item{Stocké virtuellement}
      \begin{itemize}
        \item{document électronique}
        \item{base de données électronique}
      \end{itemize}
    \end{itemize}
  \end{block}

  \begin{block}{Domaine de spécialité}
    \begin{itemize}
      \item{Champ de connaissance particulier}
      \item{Possède un vocabulaire spécifique}
    \end{itemize}
  \end{block}
\end{frame}

\begin{frame}{Introduction}\framesubtitle{Contexte}
  Valorisation de l'information numérique à l'Inist.

  \begin{block}{Inist (Institut de l'information scientifique et technique)}
    \begin{itemize}
      \item{Savoir faire documentaire}
      \item{Bases de données bibliographiques~:}
      \begin{itemize}
        \item{\textsc{Francis}~: Sciences humaines et sociales}
        \item{\textsc{Pascal}~: Sciences exactes}
      \end{itemize}
    \end{itemize}
  \end{block}

  \begin{block}{Base de données bibliographiques}
    Collection de notices bibliographiques~:
    \begin{itemize}
      \item{Titre}
      \item{Auteur(s)}
      \item{Résumé}
      \item{Descripteurs~/~Termes-clés (mots-clés)}
    \end{itemize}
  \end{block}
\end{frame}

\begin{frame}{Introduction}\framesubtitle{Contexte}
  \begin{block}{Termes-clés}
    \begin{itemize}
      \item{Unités textuelles (mots et expressions)}
      \item{Décrivent le contenu principal d'un document}
      \item{Utiles pour la Recherche d'information (\textsc{Ri})~:}
      \begin{itemize}
        \item{Indexation}
        \item{Expansion de requête}
        \item{Résumé automatique}
      \end{itemize}
    \end{itemize}
  \end{block}
\end{frame}

\begin{frame}{Introduction}\framesubtitle{Contexte -- Exemple 1}
  \begin{exampleblock}{\small
    Météo du 19 août 2012~: alerte à la canicule sur la Belgique et le
    Luxembourg
  }\justifying\small
    ~~~À l'exception de la province de Luxembourg, en alerte jaune, l'ensemble
    de la Belgique est en vigilance orange à la canicule. Le Luxembourg n'est
    pas épargné par la vague du chaleur : le nord du pays est en alerte
    orange, tandis que le sud a était placé en alerte rouge.

    ~~~En Belgique, la température n'est pas descendue en dessous des
    23\degre{}C cette nuit, ce qui constitue la deuxième nuit la plus chaude
    jamais enregistrée dans le royaume. Il se pourrait que ce dimanche soit la
    journée la plus chaude de l'année. Les températures seront comprises entre
    33 et 38\degre{}C. Une légère brise de côte pourra faiblement rafraichir
    l'atmosphère. Des orages de chaleur sont a prévoir dans la soirée et en
    début de nuit.

    ~~~Au Luxembourg, le mercure devrait atteindre 32\degre{}C ce dimanche sur
    l'Oesling et jusqu'à 36\degre{}C sur le sud du pays, et 31 à 32\degre{}C
    lundi. Une baisse devrait intervenir pour le reste de la semaine.
    Néanmoins, le record d'août 2003 (37,9\degre{}C) ne devrait pas être
    atteint.

    \begin{exampleblock}{\small Termes-clés}\justifying\small
      \underline{Août 2012}~; \underline{canicule}~;
      \underline{Belgique}~; \underline{Luxembourg}~; \underline{alerte}~;
      \underline{orange}~; \underline{chaleur}~; \underline{chaude}~;
      \underline{température}~; \underline{la plus chaude}.
    \end{exampleblock}
  \end{exampleblock}
\end{frame}

\begin{frame}{Introduction}\framesubtitle{Contexte -- Exemple 2}
  \vspace{-.33em}
  \begin{exampleblock}{\small
    Étude préliminaire de la céramique non tournée micacée du bas Languedoc
    occidental~: typologie, chronologie et aire de diffusion
  }\justifying\small
    ~~~L'étude présente une variété de céramique non tournée dont la typologie
    et l'analyse des décors permettent de l'identifier facilement. La nature de
    l'argile enrichie de mica donne un aspect pailleté à la pâte sur laquelle le
    décor effectué selon la méthode du brunissoir apparaît en traits brillant
    sur fond mat. Cette première approche se fonde sur deux séries issues de
    fouilles anciennes menées sur les oppidums du Cayla à Mailhac (Aude) et de
    Mourrel-Ferrat à Olonzac (Hérault). La carte de répartition fait état
    d'échanges ou de commerce à l'échelon macrorégional rarement mis en évidence
    pour de la céramique non tournée. S'il est difficile de statuer sur
    l'origine des décors, il semble que la production s'insère dans une ambiance
    celtisante. La chronologie de cette production se situe dans le deuxième âge
    du Fer. La fourchette proposée entre la fin du IV$^\text{e}$ et la fin du
    II$^\text{e}$ s. av. J.-C. reste encore à préciser.

    \begin{exampleblock}{\small Termes-clés}\justifying\small
      \underline{Mailhac}~; \underline{Aude}~; \underline{Mourrel-Ferrat}~;
      \underline{Olonzac}~; \underline{Hérault}~; \underline{céramique}~;
      \underline{typologie}~; \underline{décor}~; \underline{chronologie}~;
      \underline{diffusion}~; \underline{production}~; \underline{commerce}~;
      \underline{répartition}~; \underline{oppidum}~; \underline{analyse}~;
      \underline{fouille ancienne}~; Europe~; France~; le Cayla~; La Tène~;
      celtes~; distribution~; micassé~; céramique non-tournée~; echange~;
      cartographie~; habitat~; site fortifié~; identification~; étude du
      matériel~; age du fer.
    \end{exampleblock}
  \end{exampleblock}
\end{frame}

