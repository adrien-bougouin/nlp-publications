\section{Indexation par termes-clés}
  \begin{frame}{Indexation par termes-clés}
    \begin{figure}
      \centering
      \begin{tikzpicture}[scale=.35, node distance=3cm]
        \node [component] (extraction) {Extraction de termes-clés};
        \node [component, right=of extraction] (assignement) {Assignement de termes-clés};

        \node [io, above=of extraction] (document) {document};
        \node [io, above=of assignement] (vocabulaire_controle) {vocabulaire contrôlé};

        \node [io, below=of extraction] (termes_cles_extraits) {termes-clés $\subseteq$ document};
        \node [io, below=of assignement] (termes_cles_assignes) {termes-clés $\subseteq$ vocabulaire contrôlé};

        \node [above=of document, yshift=-2.85cm] (general) {\small Contexte général};

        \draw [dashed] ($(extraction.north west)+(-.5cm, 5.5cm)$) rectangle ($(extraction.south east)+(.5cm, -5.5cm)$);
        \node [above=of vocabulaire_controle, yshift=-2.9cm] (domaine_de_specialite) {\small Domaines de spécialité};

        \draw [dashed] ($(assignement.north west)+(-1cm, 5.5cm)$) rectangle ($(assignement.south east)+(1cm, -5.5cm)$);

        \path [->] (document) edge (extraction);
        \path [->] (extraction) edge (termes_cles_extraits);
        \path [->] (document) edge (assignement);
        \path [->] (vocabulaire_controle) edge (assignement);
        \path [->] (assignement) edge (termes_cles_assignes);
      \end{tikzpicture}
    \end{figure}
  \end{frame}

  \subsection{Extraction de termes-clés}
    \begin{frame}{Indexation par termes-clés}
      \begin{figure}
        \centering
        \begin{tikzpicture}[scale=.35, node distance=3cm]
          \node [selectedcomponent] (extraction) {Extraction de termes-clés};
          \node [component, right=of extraction] (assignement) {Assignement de termes-clés};

          \node [io, above=of extraction] (document) {document};
          \node [io, above=of assignement] (vocabulaire_controle) {vocabulaire contrôlé};

          \node [io, below=of extraction] (termes_cles_extraits) {termes-clés $\subseteq$ document};
          \node [io, below=of assignement] (termes_cles_assignes) {termes-clés $\subseteq$ vocabulaire contrôlé};

          \node [above=of document, yshift=-2.85cm] (general) {\small Contexte général};

          \draw [dashed] ($(extraction.north west)+(-.5cm, 5.5cm)$) rectangle ($(extraction.south east)+(.5cm, -5.5cm)$);
          \node [above=of vocabulaire_controle, yshift=-2.9cm] (domaine_de_specialite) {\small Domaines de spécialité};

          \draw [dashed] ($(assignement.north west)+(-1cm, 5.5cm)$) rectangle ($(assignement.south east)+(1cm, -5.5cm)$);

          \path [->] (document) edge (extraction);
          \path [->] (extraction) edge (termes_cles_extraits);
          \path [->] (document) edge (assignement);
          \path [->] (vocabulaire_controle) edge (assignement);
          \path [->] (assignement) edge (termes_cles_assignes);
        \end{tikzpicture}
      \end{figure}
    \end{frame}
    \section{Extraction de termes-clés}
  \begin{frame}{Extraction de termes-clés}
    \framesubtitle{Chaîne de traitements}

    \tikzstyle{io}=[
      ellipse,
      minimum width=5cm,
      minimum height=2cm,
      fill=green!20,
      draw=green!33,
      transform shape,
      font={\Large\bfseries}
    ]
    \tikzstyle{component}=[
      text centered,
      thick,
      rectangle,
      minimum width=12.5cm,
      minimum height=2.5cm,
      fill=cyan!20,
      draw=cyan!33,
      transform shape,
      font={\Large\bfseries}
    ]

    \begin{center}
      \begin{tikzpicture}[thin,
                          align=center,
                          scale=.4,
                          node distance=2cm,
                          every node/.style={text centered, transform shape}]
        \node[io] (document) {document};
        \node[component] (preprocessing) [right=of document] {Prétraitement};
        \node[component] (candidate_extraction) [below=of preprocessing] {Sélection des candidats};
        \node[component] (candidate_classification_and_ranking) [below=of candidate_extraction] {
          \begin{tabular}{r|l}
            Classification & des candidats\\
            Ordonnancement & \\
          \end{tabular}
        };
        \node[component] (keyphrase_selection) [below=of candidate_classification_and_ranking] {Sélection de termes-clés};
        \node[io] (keyphrases) [right=of keyphrase_selection] {termes-clés};

        \path[->, thick] (document) edge (preprocessing);
        \path[->, thick] (preprocessing) edge (candidate_extraction);
        \path[->, thick] (candidate_extraction) edge (candidate_classification_and_ranking);
        \path[->, thick] (candidate_classification_and_ranking) edge (keyphrase_selection);
        \path[->, thick] (keyphrase_selection) edge (keyphrases);

        \visible<2->{
            \node[draw=red, dashed, yshift=-1cm, minimum width=13cm, minimum
          height=7.5cm, label={[xshift=-5.75cm]\Large\textbf{\textcolor{red}{c\oe{}ur}}}] (core) at (candidate_extraction.south) {};
          \visible<3->{
            \coordinate[xshift=3.5em, yshift=1em] (classification) at (candidate_classification_and_ranking.west);
            \coordinate[xshift=-6em, yshift=5em] (supervised_coordinates) at (candidate_classification_and_ranking.west);
            \coordinate[xshift=.5em, yshift=-1em] (ranking) at (candidate_classification_and_ranking.west);
            \coordinate[xshift=-6em, yshift=-5em] (unsupervised_coordinates) at (candidate_classification_and_ranking.west);

            \node (supervised) at (supervised_coordinates) {\Large\textbf{supervisé}};
            \node (unsupervised) at (unsupervised_coordinates) {\Large\textbf{non-supervisé}};

            \path[->] (supervised) edge (classification);
            \path[->] (unsupervised) edge (ranking);
          }
          \visible<4->{
            \node[draw, ellipse] (unsupervised) at (unsupervised_coordinates) {\Large\textbf{non-supervisé}};
          }
        }
      \end{tikzpicture}
    \end{center}
  \end{frame}

%  \begin{frame}{Extraction de termes-clés}
%    \framesubtitle{}
%
%    \tikzstyle{io}=[
%      ellipse,
%      minimum width=5cm,
%      minimum height=2cm,
%      fill=green!20,
%      draw=green!33,
%      transform shape,
%      font={\Large\bfseries}
%    ]
%    \tikzstyle{component}=[
%      text centered,
%      thick,
%      rectangle,
%      minimum width=12.5cm,
%      minimum height=2.5cm,
%      fill=cyan!20,
%      draw=cyan!33,
%      transform shape,
%      font={\Large\bfseries}
%    ]
%
%    \begin{center}
%      \begin{tikzpicture}[thin,
%                          align=center,
%                          scale=.4,
%                          node distance=2cm,
%                          every node/.style={text centered, transform shape}]
%        \node[io] (document) {document};
%        \node[component] (preprocessing) [right=of document] {Prétraitement};
%        \node[component] (candidate_extraction) [below=of preprocessing] {Sélection des candidats};
%        \node[component] (candidate_classification_and_ranking) [below=of candidate_extraction] {
%          \begin{tabular}{r|l}
%            Classification & des candidats\\
%            Ordonnancement & \\
%          \end{tabular}
%        };
%        \node[component] (keyphrase_selection) [below=of candidate_classification_and_ranking] {Sélection de termes-clés};
%        \node[io] (keyphrases) [right=of keyphrase_selection] {termes-clés};
%
%        \path[->, thick] (document) edge (preprocessing);
%        \path[->, thick] (preprocessing) edge (candidate_extraction);
%        \path[->, thick] (candidate_extraction) edge (candidate_classification_and_ranking);
%        \path[->, thick] (candidate_classification_and_ranking) edge (keyphrase_selection);
%        \path[->, thick] (keyphrase_selection) edge (keyphrases);
%
%        \node[draw, thick, minimum width=12.5cm, minimum height=2.5cm] (current) at (preprocessing) {\Large\textbf{Prétraitement}};
%      \end{tikzpicture}
%    \end{center}
%  \end{frame}
%
%  \begin{frame}{Extraction de termes-clés}
%    \framesubtitle{Prétraitement}
%
%    \begin{enumerate}
%      \item{Segmentation en phrases}
%      \item{Segmentation des phrases en mots}
%      \item{Étiquetage grammatical des mots}
%    \end{enumerate}
%  \end{frame}

  \begin{frame}{Extraction de termes-clés}
    \framesubtitle{}

    \tikzstyle{io}=[
      ellipse,
      minimum width=5cm,
      minimum height=2cm,
      fill=green!20,
      draw=green!33,
      transform shape,
      font={\Large\bfseries}
    ]
    \tikzstyle{component}=[
      text centered,
      thick,
      rectangle,
      minimum width=12.5cm,
      minimum height=2.5cm,
      fill=cyan!20,
      draw=cyan!33,
      transform shape,
      font={\Large\bfseries}
    ]

    \begin{center}
      \begin{tikzpicture}[thin,
                          align=center,
                          scale=.4,
                          node distance=2cm,
                          every node/.style={text centered, transform shape}]
        \node[io] (document) {document};
        \node[component] (preprocessing) [right=of document] {Prétraitement};
        \node[component] (candidate_extraction) [below=of preprocessing] {Sélection des candidats};
        \node[component] (candidate_classification_and_ranking) [below=of candidate_extraction] {
          \begin{tabular}{r|l}
            Classification & des candidats\\
            Ordonnancement & \\
          \end{tabular}
        };
        \node[component] (keyphrase_selection) [below=of candidate_classification_and_ranking] {Sélection de termes-clés};
        \node[io] (keyphrases) [right=of keyphrase_selection] {termes-clés};

        \path[->, thick] (document) edge (preprocessing);
        \path[->, thick] (preprocessing) edge (candidate_extraction);
        \path[->, thick] (candidate_extraction) edge (candidate_classification_and_ranking);
        \path[->, thick] (candidate_classification_and_ranking) edge (keyphrase_selection);
        \path[->, thick] (keyphrase_selection) edge (keyphrases);

        \node[draw, thick, minimum width=12.5cm, minimum height=2.5cm] (current) at (candidate_extraction) {\Large\textbf{Sélection des candidats}};
      \end{tikzpicture}
    \end{center}
  \end{frame}

  \begin{frame}{Extraction de termes-clés}
    \framesubtitle{Sélection des candidats}

    Deux approches classiques~:
    \begin{itemize}
      \item{Extraction des n-grammes}
        \begin{itemize}
          \item{$n \subseteq \{1..3\}$}
          \item{Filtrage avec un anti-dictionnaire}
          \item[\textcolor{red}{\scriptsize$\blacktriangleright$}]{Sursélection des candidats $\Rightarrow$ faible qualité}
        \end{itemize}
      \item{Reconnaissance de formes}
        \begin{itemize}
          \item{\texttt{(NOM | ADJ)+}}
        \end{itemize}
    \end{itemize}

    \uncover<2->{
      Une approche non explorée jusqu'alors~:
      \begin{itemize}
        \item{Extraction des candidats termes}
        \begin{itemize}
          \item{Utilisation de TTC Terme Suite}
          \item{Formes très précises~:}
          \begin{itemize}
            \item{\texttt{NOM à NOM}}
            \item{\texttt{NOM en NOM}}
            \item{\texttt{NOM à NOM ADJ}}
            \item{etc.}
          \end{itemize}
        \end{itemize}
      \end{itemize}
    }
  \end{frame}

  \begin{frame}{Extraction de termes-clés}
    \framesubtitle{Sélection des candidats --- Exemples}

    \resizebox{\linewidth}{!}{
      \begin{tabular}{l|l|l}
        \toprule
        \multicolumn{3}{c}{\textit{\og{}bassin moyen du Don\fg{}}}\\
        \midrule
        \multicolumn{1}{c|}{$\{1..3\}$-grammes} & \multicolumn{1}{c|}{\texttt{(NOM | ADJ)+}} & \multicolumn{1}{c}{Candidats termes} \\
        \hline
        \textit{\og{}bassin\fg{}} & \textit{\og{}bassin moyen\fg{}} & \textit{\og{}bassin\fg{}} \\
         \textit{\og{}moyen\fg{}} & \textit{\og{}Don\fg{}} & \textit{\og{}Don\fg{}} \\
          \textit{\og{}Don\fg{}} & & \textit{\og{}bassin moyen\fg{}} \\
        \textit{\og{}bassin moyen\fg{}} & & \textit{\og{}bassin moyen du Don\fg{}} \\
        \textit{\og{}moyen du Don\fg{}} & & \\
        \bottomrule
      \end{tabular}
    }
  \end{frame}

  \begin{frame}{Extraction de termes-clés}
    \framesubtitle{}

    \tikzstyle{io}=[
      ellipse,
      minimum width=5cm,
      minimum height=2cm,
      fill=green!20,
      draw=green!33,
      transform shape,
      font={\Large\bfseries}
    ]
    \tikzstyle{component}=[
      text centered,
      thick,
      rectangle,
      minimum width=12.5cm,
      minimum height=2.5cm,
      fill=cyan!20,
      draw=cyan!33,
      transform shape,
      font={\Large\bfseries}
    ]

    \begin{center}
      \begin{tikzpicture}[thin,
                          align=center,
                          scale=.4,
                          node distance=2cm,
                          every node/.style={text centered, transform shape}]
        \node[io] (document) {document};
        \node[component] (preprocessing) [right=of document] {Prétraitement};
        \node[component] (candidate_extraction) [below=of preprocessing] {Sélection des candidats};
        \node[component] (candidate_classification_and_ranking) [below=of candidate_extraction] {
          \begin{tabular}{r|l}
            Classification & des candidats\\
            Ordonnancement & \\
          \end{tabular}
        };
        \node[component] (keyphrase_selection) [below=of candidate_classification_and_ranking] {Sélection de termes-clés};
        \node[io] (keyphrases) [right=of keyphrase_selection] {termes-clés};

        \path[->, thick] (document) edge (preprocessing);
        \path[->, thick] (preprocessing) edge (candidate_extraction);
        \path[->, thick] (candidate_extraction) edge (candidate_classification_and_ranking);
        \path[->, thick] (candidate_classification_and_ranking) edge (keyphrase_selection);
        \path[->, thick] (keyphrase_selection) edge (keyphrases);

        \node[draw, thick, minimum width=12.5cm, minimum height=2.5cm] (current)
        at (candidate_classification_and_ranking) {\Large\bfseries
          \begin{tabular}{r|l}
            \sout{Classification} & des candidats\\
            Ordonnancement & \\
          \end{tabular}
        };
      \end{tikzpicture}
    \end{center}
  \end{frame}

  \begin{frame}{Extraction de termes-clés}
    \framesubtitle{Ordonnancement des candidats}

    \begin{itemize}
      \item{De nombreuses méthodes}
      \item{Diverses approches}
      \item{Parmi elles~:}
      \begin{itemize}
        \item{TF$\times$IDF}
        \item{TopicRank~\cite{bougouin2013topicrank}}
      \end{itemize}
    \end{itemize}
  \end{frame}

  \begin{frame}{Extraction de termes-clés}
    \framesubtitle{Ordonnancement des candidats --- TF$\times$IDF}

    \begin{block}{Hypothèse}
      Dans un document, un mot est d'autant plus important qu'il  y est fréquent
      (TF) et spécifique (IDF).
    \end{block}

    \begin{align*}
      \text{importance}(\text{candidat}) = \sum_{\text{mot} \in \text{candidat}} \text{TF$\times$IDF}(\text{mot})
    \end{align*}
  \end{frame}

  \begin{frame}{Extraction de termes-clés}
    \framesubtitle{Ordonnancement des candidats --- TopicRank}

    \begin{block}{Hypothèses}
      \begin{enumerate}
        \item{Plusieurs candidats désignent le même sujet (concept)}
        \item{Seul le candidat le plus représentatif du sujet doit être extrait}
        \item{Les sujets qui cooccurrent se recommandent mutuellement~:}
        \begin{itemize}
          \item{Plus un sujet cooccurre avec d'autres sujets, plus il est
                important}
          \item{Plus un sujet est important, plus les sujets avec lesquels il
                cooccurre sont important}
        \end{itemize}
      \end{enumerate}
    \end{block}
  \end{frame}



  \subsection{Assignement de termes-clés}
    \begin{frame}{Indexation par termes-clés}
      \begin{figure}
        \centering
        \begin{tikzpicture}[scale=.35, node distance=3cm]
          \node [component] (extraction) {
            Extraction de termes-clés\\
            \Large (ordonnancement ou classification)
          };
          \node [selectedcomponent, right=of extraction] (assignement) {Assignement de termes-clés};

          \node [io, above=of extraction] (document) {document};
          \node [io, above=of assignement] (vocabulaire_controle) {vocabulaire contrôlé};

          \node [io, below=of extraction] (termes_cles_extraits) {termes-clés $\subseteq$ document};
          \node [io, below=of assignement] (termes_cles_assignes) {termes-clés $\subseteq$ vocabulaire contrôlé};

          \node [above=of document, yshift=-2.85cm] (general) {\small Contexte général};

          \draw [dashed] ($(extraction.north west)+(-.5cm, 5.5cm)$) rectangle ($(extraction.south east)+(.5cm, -5.5cm)$);
          \node [above=of vocabulaire_controle, yshift=-2.9cm] (domaine_de_specialite) {\small Domaines de spécialité};

          \draw [dashed] ($(assignement.north west)+(-1cm, 5.5cm)$) rectangle ($(assignement.south east)+(1cm, -5.5cm)$);

          \path [->] (document) edge (extraction);
          \path [->] (extraction) edge (termes_cles_extraits);
          \path [->] (document) edge (assignement);
          \path [->] (vocabulaire_controle) edge (assignement);
          \path [->] (assignement) edge (termes_cles_assignes);
        \end{tikzpicture}
      \end{figure}
    \end{frame}
    \begin{frame}{Indexation par termes-clés}\framesubtitle{Assignement de termes-clés}
  \begin{block}{Objectif}
    Positionner le document dans son domaine.
  \end{block}
\end{frame}

\begin{frame}{Assignement de termes-clés}\framesubtitle{\textsc{Kea}++~\cite{medelyan2006kea++}}
  \begin{block}{Thésaurus}
    \begin{itemize}
      \item{Liste de termes (= vocabulaire contrôlé)}
      \item{Décrit les relations entre les termes}
      \begin{itemize}
        \item{Hiérarchie (générique, spécifique)}
        \item{Champ sémantique}
      \end{itemize}
    \end{itemize}
  \end{block}

  \begin{enumerate}
    \item{Projection du thésaurus dans le document}
    \item{Ajout des relations du thésaurus comme critère de classification}
    \begin{itemize}
      \item{TF-IDF}
      \item{Première position}
      \item{Nombre de relations avec d'autres candidats}
    \end{itemize}
  \end{enumerate}

  \begin{alertblock}{TODO}
    À refaire
  \end{alertblock}
\end{frame}

\begin{frame}{Assignement de termes-clés}\framesubtitle{Alternatives}
  \begin{alertblock}{TODO}
    \begin{itemize}
      \item{SMT Keyphrase extraction}
      \item{Brigitte Grau}
    \end{itemize}
  \end{alertblock}
\end{frame}

%\subsubsection{Bilan}
%  \begin{frame}{Assignement de termes-clés}\framesubtitle{Bilan}
%    \begin{itemize}
%      \item{Une seule méthodes}
%      \item{Avec ressources supplémentaires}
%    \end{itemize}
%
%    \vspace{1em}
%
%    \begin{block}{Avantages}
%      \begin{itemize}
%        \item{Adapté au domaine de spécialité traité}
%        \item{Assigne des termes-clés terminologiquement validés}
%        \item{Indexation par termes-clés cohérente}
%      \end{itemize}
%    \end{block}
%
%    \begin{alertblock}{Inconvénients}
%      \begin{itemize}
%        \item{Scénarii d'utilisation limités}
%        \item{Impossibilité de trouver tous les termes-clés}
%      \end{itemize}
%    \end{alertblock}
%  \end{frame}



    \begin{frame}{Indexation par termes-clés}\framesubtitle{Bilan}
    Deux catégories réalisées disjointement~:
    \begin{itemize}
      \item{Extraction de termes-clés}
      \begin{itemize}
        \item{Ordonnancement ou classification}
        \item{Applicable dans le contexte général}
      \end{itemize}
      \item{Assignement de termes-clés}
      \begin{itemize}
        \item{Focalisé sur un vocabulaire contrôlé}
        \item{Adapté aux domaines de spécialité}
      \end{itemize}
    \end{itemize}

%    Extraction Vs Assignement~:
%    \begin{itemize}
%      \item{Contexte général Vs domaines de spécialité}
%      \item{Limation au document Vs Limitation au vocabulaire contrôlé}
%    \end{itemize}

    \vspace{1em}

    \begin{alertblock}{Limite}
      Manque d'exhaustivité
      \begin{itemize}
        \item{Extraction limitée au contenu du document}
        \item{Assignement limitée au vocabulaire du domaine}
      \end{itemize}
    \end{alertblock}
  \end{frame}

  \begin{frame}{Indexation par termes-clés}\framesubtitle{Bilan}
    \vspace{-.33em}
    \begin{exampleblock}{\small
      Étude préliminaire de la \textbf{céramique non tournée}
      \textbf{micacée} du bas Languedoc occidental~: \textbf{typologie},
      \textbf{chronologie} et aire de \textbf{diffusion}
    }\justifying\small
      ~~~L'étude présente une variété de \textbf{céramique non tournée} dont la
      \textbf{typologie} et l'analyse des \textbf{décors} permettent de
      l'identifier facilement. La nature de l'argile enrichie de mica donne un
      aspect pailleté à la pâte sur laquelle le \textbf{décor} effectué selon
      la méthode du brunissoir apparaît en traits brillant sur fond mat. Cette
      première approche se fonde sur deux séries issues de \textbf{fouilles
      anciennes} menées sur les \textbf{oppidums} \textbf{du Cayla} à
      \textbf{Mailhac} (\textbf{Aude}) et de \textbf{Mourrel-Ferrat} à
      \textbf{Olonzac} (\textbf{Hérault}). La carte de
      \textbf{répartition} fait état d'\textbf{échanges} ou de
      \textbf{commerce} à l'échelon macrorégional rarement mis en évidence pour
      de la \textbf{céramique non tournée}. S'il est difficile de statuer sur
      l'origine des \textbf{décors}, il semble que la \textbf{production}
      s'insère dans une ambiance celtisante. La \textbf{chronologie} de cette
      \textbf{production} se situe dans le deuxième \textbf{âge du Fer}. La
      fourchette proposée entre la fin du IV$^\text{e}$ et la fin du II$^\text{e}$
      s. av. J.-C. reste encore à préciser.

      \begin{exampleblock}{\small Termes-clés de référence}\justifying\small
        \textbf{Mailhac}~; \textbf{Aude}~; \textbf{Mourrel-Ferrat}~;
        \textbf{Olonzac}~; \textbf{Hérault}~; \textbf{céramique}~;
        \textbf{typologie}~; \textbf{décor}~; \textbf{chronologie}~;
        \textbf{diffusion}~; \textbf{production}~; \textbf{commerce}~;
        \textbf{répartition}~; \textbf{oppidum}~; \textbf{analyse}~;
        \textbf{fouille ancienne}~; \textbf{le Cayla}~;
        \textbf{micassé}~; \textbf{céramique non-tournée}~;
        \textbf{echange}~; \textbf{age du} \textbf{Fer}~; La Tène~;
        Europe~; France~; celtes~; distribution~; cartographie~; habitat~; site
        fortifié~; identification~; étude du matériel
      \end{exampleblock}
    \end{exampleblock}
  \end{frame}
