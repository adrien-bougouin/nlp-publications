\chapter{Introduction}
\label{chap:main-introduction}
  \chaptercite{
    Rechercher des informations est une activité fréquente pour quiconque
    utilise quotidiennement un ordinateur. Alors qu'Internet est une source
    abondante d'information de tout genre, trouver les documents pertinents est
    encore difficile. [...] Les termes-clés aident à organiser et retrouver ces
    documents d'après leur contenu.
%    Seeking information is an important activity for everyone who uses computers
%    in daily life. While the Internet is a bountiful source of all types of
%    knowledge, locating relevant documents is still a great challenge [\dots]
%    Keyphrases help organise documents and retrieve them based on content.
  }{
    \newcite{medelyan2008smalltrainingset}
  }{.75\linewidth}{\justify}

  %-----------------------------------------------------------------------------

  \section{Contexte}
  \label{sec:main-introduction-context}
    La société contemporaine dans laquelle nous vivons se situe en pleine ère de
    l'information. Cette ère succède l'ère moderne, durant laquelle de
    nombreuses découvertes et avancées scientifiques ont été faites, durant
    laquelle des connaissances considérables ont été acquises. Elle  est aussi
    marquée par le début de la mondialisation, qui, sur le plan scientifique,
    favorise la dissémination et la production de nouvelles connaissances.
    Jusqu'alors rangées au format papier dans des bibliothèques, où des
    documentalistes les classent et aident ensuite scientifiques et particuliers
    à y accèder le plus efficacement possible, les connaissances sont devenues
    trop nombreuses et leur stockage physique
    inadapté~\cite{rider1946thegreatdilemmaofworldorganization}. L'ère de
    l'information débute vers la fin des années 1940 et apporte une solution à
    ce problème~: l'informatisation des données. Cette informatisation présente
    tout d'abord l'avantage de pouvoir stocker les connaissances sur des
    supports pérennes, de capacité de plus en plus grande (de quelques
    mégaoctets à plusieurs giga-octets) et de taille de plus en plus réduite (du
    disque dur au \textsc{Dvd}). Très vite, la communauté scientifique y voit
    aussi un moyen pour améliorer la recherche d'information, en indexant les
    documents qui contiennent les connaissances, en proposant des interfaces
    pour permettre à un utilisateur de formuler une requète et en cherchant les
    documents pertinents vis-à-vis de cette
    requète~\cite{salton1964automaticphrasematching}.

    Autre résultat du progrès informatique, le réseau informatique mondial,
    Internet, joue un rôle crucial dans l'ère de l'information. En effet, si
    l'informatisation des données facilite leur recheche, Internet facilite leur
    accès depuis les bases de données informatisées qui y sont connectées.
    Médium d'information mondial et accessible de tous\footnote{En 2012,
    l'Institut national de la statistique et des études économiques (Insee)
    estimait qu'environ 80~\% de français sont connectés à Internet.}, il
    favorise la transition depuis les bibliothèques traditionnelles vers des
    bibliothèques numériques, qui combinent le savoir faire des documentalistes
    avec les techniques du Traitement automatique des langues (\textsc{Tal}) et
    de la Recherche d'information (\textsc{Ri}) pour informatiser les données et
    faciliter leur accès.

    Cette thèse s'inscrit dans le cadre du projet \textsc{Anr} Termith
    (\textsc{Anr-12-Cord-0029}), qui s'intéresse à l'accès à l'information
    numérique en domaines de spécialités et qui s'articule lui même autour du
    travail de l'Institut de l'information scientifique et technique (Inist). Né
    en 1988 de la fusion du Centre de documentation scientifique et technique
    (\textsc{Cdst}) et du Centre de documentation sciences humaines
    (\textsc{Cdsh}), tout deux fondés en 1970 pendant les débuts de
    l'informatisation des données, l'Inist possède deux des plus importantes
    bases de données informatisées d'Europe~: \textsc{Pascal} en sciences
    exactes et \textsc{Francis} en sciences humaines. Aujourd'hui acteur de la
    Bibliothèque scientifique numérique (\textsc{Bsn}) fondée en 2009 par le
    ministère de l'enseignement supérieur et de la recherche français, l'une de
    ses missions est de faciliter l'accès à la recherche mondiale au travers de
    la production de notices bibliographiques associées à des mots-clés. Ces
    mots-clés, que nous appelons ici termes-clés, sont des mots ou des
    expressions qui représentent le contenu principal du document que représente
    la notice. Ils permettent de le résumer, de le catégoriser et de l'indexer.
    Nous parlons d'indexation par termes-clés.
    
%    Au sein de l'Inist, l'indexation par termes-clés est réalisée dans le
%    respect des bonnes pratiques
%    documentaires~\cite{guinchat1996techniquesdocumentaires}~: les indexeurs
%    professionnels (documentalistes) ne donnent pas leur opinion au travers des
%    termes-clés (impartialité), donnent tous les termes-clés nécessaires à la
%    caractérisation du contenu du document (exhaustivité et spécificité) et
%    utilisent un vocabulaire contrôlé et identique pour tous les documents du
%    même domaine (conformité et homogénéité). 

    L'indexation manuelle des documents par leurs termes-clés est une tâche
    coûteuse et chronophage. Soucieux de faciliter le travail d'indexation par
    termes-clés, que ce soit pour des productions scientifiques ou des documents
    d'autres natures (articles journalistiques, nouvelles, etc.), de nombreux
    chercheurs s'intéressent à son automatisation en témoignent le nombre
    grandissant d'articles scientifiques à ce
    sujet~\cite{hasan2014state_of_the_art} ainsi que l'émergence de campagnes
    d'évaluation~\cite{kim2010semeval,paroubek2012deft}.
    
    Dans cette thèse, nous
    proposons des méthodes pour l'indexation automatique par termes-clés.
    Travaillant d'abord d'un point de vue généraliste, nous nous focalisons
    ensuite sur l'indexation par termes-clés en domaines de spécialités.

  %-----------------------------------------------------------------------------

  \section{Problématique}
  \label{sec:main-introduction-problem_statement}
    Étant donné un document (complet ou résumé), l'indexation automatique par
    termes-clés consiste à trouver les unités textuelles qui décrivent son
    contenu principal. La difficulté de cette tâche réside dans l'identification
    des éléments importants vis-à-vis de son contenu, ainsi que leur
    représentation avec les unités textuelles appropriées. La première
    difficulté est d'ordre sémantique~: il faut réussir à comprendre le document
    pour en extraire l'essence~; la seconde est d'ordre linguistique et
    terminologique~: il faut déterminer les propriétés linguistiques des
    termes-clés et connaître le vocabulaire du domaine auquel appartient le
    document. Par ailleurs, la forme la plus appropriée pour un terme-clé n'est
    pas nécessairement présente dans le contenu du document (le terme-clé est
    implicite).

    Plutôt que de comprendre le document, les méthodes d'indexation par
    termes-clés de la littérature se fondent sur des statistiques et
    des modélisations particulières du document. Pour ce qui est de l'usage
    d'unités textuelles appropriées, elles se contentent le plus souvent des
    unités textuelles qui occurrent dans le document. De manière général, des
    termes-clés candidats sont sélectionnés dans le document d'après des
    critères prédéfinis (ce doit être des groupes nominaux, par exemple), ces
    candidats sont analysés et les termes-clés sont extraits d'entre eux en
    fonction du résultat de l'analyse.

    L'analyse des termes-clés candidats du document peut être réalisée avec deux
    approches~: supervisée ou non supervisée. En général, l'approche supervisée
    consiste à analyser les caractéristiques des termes-clés de données
    manuellement indexées pour apprendre à reconnaître les termes-clés. Elle
    consiste donc à chercher les candidats qui sont le plus vraissemblablement
    les termes-clés, tandis que l'approche non supervisée consiste à chercher
    les candidats les plus importants dans le contenu du document.

    Cette thèse s'inscrit dans le cadre du projet \textsc{Anr} Termith
    (\textsc{Anr-12-Cord-0029}), qui s'intéresse à l'accès à l'information
    numérique en domaines de spécialités. Notre objectif est de proposer une
    méthode d'indexation automatique par termes-clés en domaines de spécialités.
    Dans un premier temps, nous proposons une méthode non supervisée,
    généraliste et applicable dans tous les scénarii d'utilisation. Dans un
    second temps, nous proposons une méthode supervisée adaptée aux domaines de
    spécialités.

  %-----------------------------------------------------------------------------

  \section{Hypothèses}
  \label{sec:main-introduction-hypothesis}
    Notre première hypothèse concerne la sélection des candidats et leur impact
    sur la suite du processus d'indexation par termes-clés. Selon nous,
    l'indexation gagne en efficacité lorsque la qualité linguistique de
    l'ensemble de candidats sélectionnés augmente. \TODO{discuter}
    %
%    Cette qualité peut être quantifiée à partir
%    de deux critères~: le nombre de candidats sélectionnés et le nombre de
%    termes-clés qui s'y trouvent. Paradoxalement, le premier doit être minimisé,
%    car un espace de recherche trop grand augmente la difficulté de
%    l'indexation~\cite{hasan2014state_of_the_art}, et le second maximisé. Afin
%    de trouver le meilleur compromis entre ces deux conditions, nous proposons
%    d'analyser plus finement les propriétés linguistiques des termes-clés.
%    Identifier ces propriétés doit permettre d'affiner les critères de
%    sélection, donc de réduire le nombre de candidats sélectionnés, sans
%    éliminer les termes-clés du document.
    
    Notre seconde hypothèse concerne la détection des mots et expressions
    importants vis-à-vis d'un document. Selon nous ce n'est pas l'importance de
    ces mots et expressions qui doit être déterminé, mais l'importance de ce
    qu'ils représentes. \TODO{discuter}
    %
%    représentation du document utilisée par
%    une catégorie particulière de méthodes non supervisées. Cette catégorie de
%    méthodes modélise le document par un graphe de mots connectés entre eux
%    lorsqu'ils sont sémantiquement liés, analyse ce graphe afin d'en déterminer
%    les mots les plus importants (les mots-clés), puis les utilisent pour
%    générer les termes-clés. Selon nous, ce n'est pas l'importance des mots
%    qu'il faut déterminer, mais l'importance de ce qu'ils représentent. De plus,
%    si plusieurs mots ou expression véhiculent le même sujet, la même idée,
%    alors leurs relations sémantiques doivent être mutualisée afin d'améliorer
%    la précision de l'analyse.
    
    Enfin, notre troisième hypothèse concerne l'usage des données indexées
    manuellement pour l'indexation automatique d'un document du même domaine que
    celles-ci. Nous pensons qu'il est possible de tirer pofit de ces données
    pour (1) améliorer la précision de l'extraction des unités textuelles importantes en
    contextualisant le document dans son domaine et (2) assigner des termes-clés
    du domaine qui sont implicites au document, mais importants dans son
    contexte. \TODO{discuter}
    %
%    Alors que dans la littérature, ces données servent à apprendre
%    à identifier les termes-clés parmi les candidats selon diverses
%    caractéristiques discriminantes, nous pensons qu'elles peuvent aussi servir
%    à contextualiser le document à indexer dans son domaine. Selon nous, cette
%    contextualisation permet d'identifier plus précisément les éléments
%    importants du document s'ils sont importants dans son contexte global
%    (domaine), et même de faire émerger des termes-clés implicites. Si cette
%    hypothèse est juste, alors la contextualisation est d'autant plus importante
%    en domaines de spécialités, car elle favorise un indexation par termes-clés
%    respectueuse des pratiques
%    documentaires~\cite{guinchat1996techniquesdocumentaires}.

  %-----------------------------------------------------------------------------

  \section{Plan de thèse}
  \label{sec:main-introduction-outline}
    Cette thèse est organisée de la manière suivante. Tout d'abord, le
    chapitre~\ref{chap:main-state_of_the_art} présente l'état de l'art en
    indexation automatique par termes-clés, puis le
    chapitre~\ref{chap:main-data_description} introduit les données avec
    lesquelles nous travaillons. Nos contributions sont détaillées dans les
    chapitres~\ref{chap:main-domain_independent_keyphrase_extraction}
    et~\ref{chap:main-domain_specific_keyphrase_annotation}. Le
    chapitre~\ref{chap:main-domain_independent_keyphrase_extraction} présente
    nos travaux fondés sur les deux premières hypothèses et le
    chapitre~\ref{chap:main-domain_specific_keyphrase_annotation} sur la
    troisième hypothèse. Ce dernier ce concentre sur l'indexation en domaine de
    spécialité. Il présente tout d'abord l'indexation manuelle réalisée par les
    indexeurs professionnel, puis notre contribution et enfin, une campagne
    d'évaluation manuelle de nos travaux en domaines de spécialités. Pour
    terminer, le chapitre~\ref{chap:main-conclusion} dresse le bilan de notre
    travail et présente quelques perspectives.

