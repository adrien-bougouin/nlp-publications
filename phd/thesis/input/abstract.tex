A keyphrase is a word or a multi-word expression that represent one of the most
important aspects of a document. Similar to an abstract, a set of keyphrases
associated to a document prodvides a synthetical representation of it. As a
consequence, readers can easily and precisely get the big picture of a document
without reading it entirely. They are also used to index documents so that
anyone can easily find them. Although keyphrases are very usefull in such a
contemporary world where knwoledges are numerically stored and accessible from
anywhere, documents do not always have associated keyphrases. Therefore,
automatic keyphrase annotation is a challenging task. Studied since a few
decades, best reported performances are far below 100~\% precision. This thesis
addresses the automatic keyphrases annotation and presents three new methods.

First, we tackle the keyphrase candidate selection problem. Keyphrase candidates
are textual units that could be keyphrases based on shallow observations. Our
candidate selection method relies on a preliminary study of linguistic
properties of ground truth keyphrases from various datasets. This method shows
that filtering heuristics focused on adjective categories contribute to the
selections of better candidates. Second, we propose an unsupervised keyphrase
annotation method that ranks by importance the topics of the document using the
graph-based algorithm historically used by the famous search engine Google. This
method performs better performances than previous graph-based methods and more
efficiently deals with redundancy. Finally, we extend the previous method. We
leverage the domain knowledge and assign keyphrases that do not always occur
within the document.

