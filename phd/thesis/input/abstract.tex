Keyphrases are words or multi-word expressions that represent the content of a
document. Keyphrases give a synoptic view of a document and help to index it for
information retrieval. This Ph.D thesis focuses on domain-specific automatic
keyphrase annotation. Automatic keyphrase annotation is still a difficult task,
and current systems do not achieve satisfactory results. Our work is divided in
two steps. First, we propose a keyphrase candidate selection method that focuses
on the categories of adjectives relevant within keyphrases and propose a method
to rank them according to their importance within the document. This method,
TopicRank, is a graph-based method that clusters keyphrase candidates into
topics, ranks the topics and extracts one keyphrase per important topic. Our
experiments show that TopicRank significantly outperforms other graph-based
methods for automatic keyphrase annotation. Second, we focus on domain-specific
documents and adapt our previous work. We study the best practice of manual
keyphrase annotation by professional indexers and mimic it with a new method,
TopicCoRank. TopicCoRank adds a new graph representing the specific domain to
the topic graph of TopicRank. Leveraging this second graph, TopicCoRank
possesses the rare ability to provide keyphrases that do not occur within
documents. Applied on four corpora of four specific domains, TopicCoRank
significantly outperforms TopicRank. 

