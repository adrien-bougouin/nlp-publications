Les termes-clés, ou mots-clés, sont des mots ou des expressions qui représentent
le contenu d'un document. Ils en donnent une représentation synthétique et
permettent de l'indexer pour la recherche d'information. Cette thèse s'intéresse
à l'indexation automatique par termes-clés de documents en domaines de
spécialités. La tâche est difficile à réaliser et les méthodes actuelles peinent
encore à atteindre des résultats satisfaisant. Notre démarche s'organise en deux
temps. Dans un premier temps, nous nous intéressons à l'indexation par
termes-clés en général. Nous proposons une méthode pour sélectionner des
termes-clés candidats dans un document en nous focalisant sur la catégorie des
adjectifs qu'ils peuvent contenir, puis proposons une méthode pour les ordonner
par importance. Cette dernière, TopicRank, se situe en aval de la sélection des
candidats. C'est une méthode à base de graphe qui groupe les termes-clés
candidats véhiculant le même sujet, projettent les sujets dans un graphe et
extrait un terme-clé par sujet. Nos expériences montrent que TopicRank est
significativement meilleur que les précédentes méthodes à base de graphe. Dans
un second temps, nous adaptons notre travail à l'indexation par termes-clés en
domaines de spécialités. Nous étudions la méthodologie d'indexation manuelle de
documentalistes et la simulons à l'aide de TopicCoRank. TopicCoRank ajoute à
TopicRank un graphe qui représente le domaine de spécialités du document. Grâce
à ce second graphe, TopicCoRank possède la rare capacité à fournir des
termes-clés qui n'apparaissent pas dans les documents. Appliqué à quatre
domaines de spécialités, TopicCoRank améliore significativement TopicRank.

