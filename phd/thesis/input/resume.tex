Un terme-clé, couramment appelé mot-clé, est un mot ou une expression qui
représente un des aspects les plus importants d'un document. À l'instar d'un
résumé, les termes-clés attribués à un document donnent une représentation
synthétique de ce dernier et permettent à un lecteur de se faire une idée rapide
et précise de son contenu sans en lire l'intégralité. En plus de cela, ils
permettent d'indexer des documents pour la recherche d'information. Bien
qu'avantageux, les termes-clés ne sont pas toujours disponibles et il faut donc
les trouver automatiquement pour les documents. Étudiée depuis quelques
décennies, la tâche d'indexation automatique par termes-clés est encore
aujourd'hui une tâche pour laquelle nous peinons à atteindre le seuil
psychologique de 50~\% de précision. Dans cette thèse, nous nous intéressons à
l'indexation automatique par termes-clés et proposons trois nouvelles méthodes.
Notre démarche s'organise en deux temps.

Dans un premier temps, nous nous intéressons à l'extraction de termes-clés dans
un contexte généraliste et proposons une méthode pour sélectionner des
termes-clés candidats dans un document et une méthode pour ordonner par
importance les termes-clés candidats. Avant de proposer notre méthode de
sélection des candidats, nous effectuons une étude, sur deux langues (français
et anglais), des propriétés linguistiques des termes-clés et montrons que la
catégorie des adjectifs, en particulier s'il est dénominal, est un facteur
descriminant pour décider si un adjectif doit faire partie d'un terme-clé
candidat. Fondée sur cette analyse, notre méthode sélectionne les séquences de
noms et d'adjectifs comme termes-clés candidats, puis supprime de ces derniers
les adjectifs jugés inintéressants. La seconde méthode que nous proposons se
situe en aval de la sélection des candidats. Il s'agit du c\oe{}ur de
l'extraction de termes-clés, qui consiste à déterminer quels sont les
termes-clés candidats les plus important vis-à-vis du contenu du document. Notre
méthode, TopicRank, est dite \og{}à base de graphe\fg{}, c'est-à-dire qu'elle
projette des entités du document dans un graphe et utilise un algorithme qui
simule le concept du vote pour déterminer celles les plus importantes. Avec
TopicRank, le type d'entité sur lequel nous travaillons est le sujet, que nous
approximons en groupant les termes-clés candidats qui véhiculent la même
information sous une forme lexicale différente. Pour chacun des sujets jugés les
plus importants, un de ses termes-clés candidats est ensuite sélectionné comme
terme-clé. Nos expériences réalisées sur des collections de données de langue et
de nature différentes, montrent des performances significativement supérieures
aux précédentes méthodes \og{}à base de graphe\fg{}.

Dans un second temps, nous étendons notre travail et l'adaptons au contexte de
l'indexation par termes-clés en domaine de spécialité. Après une étude de la
méthodologie d'indexation par termes-clés réalisée manuellement par des
indexeurs professionels pour une bibliothèque numérique, nous proposons une
extension \og{}à base de graphe\fg{} de TopicRank pour simuler le comportement
d'un indexeur professionnel. Notre troisième méthode, TopicCoRank, ajoute à
TopicRank un graphe qui représente le domaine de spécialité du document,
modélisé par les termes-clés de référence dans le domaine. Grâce à ce second
graphe unifié au graphe de sujets initial, TopicCoRank possède la rare capacité
à fournir des termes-clés pertinents, même lorsqu'ils n'aparaissent pas dans le
document analysé. Appliqué à cinq collections de notices bibliographiques dans
cinq disciplines différentes, TopicCoRank montre une amélioration significative
vis-à-vis de TopicRank et de l'état de l'art.
