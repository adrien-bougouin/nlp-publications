\chapter{Introduction}
\label{chap:main-introduction}
  \chaptercite{
    Seeking information is an important activity for everyone who uses computers
    in daily life. While the Internet is a bountiful source of all types of
    knowledge, locating relevant documents is still a great challenge [\dots]
    Keyphrases help organise documents and retrieve them based on content.
  }{
    \newcite{medelyan2008smalltrainingset}
  }

  %-----------------------------------------------------------------------------

  \section{Contexte}
  \label{sec:main-introduction-context}
    Avec l'essor du numérique, le Web occupe aujourd'hui une place importante
    dans notre société. Celui-ci contient tout type d'information (culturelle,
    historique, scientifique, etc.) qu'il rend disponible pour tous. Cependant,
    le Web est en constante expansion et le nombre croissant d'informations
    disponibles complique leur accès, leur recherche. Pour résoudre ce problème,
    il faut représenter et organiser efficacement les documents numériques. Dans
    le contexte de la recherche scientifique (en domaines de spécialités) il est
    important de résoudre ce problème, car favoriser l'accès aux productions
    scientifiques favorise les avancées scientifiques. C'est pourquoi sont
    créées des bibliothèques numériques telles que la Bibliothèque Scientifique
    Numérique (\textsc{Bsn}) fondée en 2009 par le ministère de l'enseignement
    supérieur et de la recherche français.

    Afin de mieux comprendre comment l'accès aux informations peut être
    facilité, prenons l'exemple de l'institut de l'information scientifique et
    technique (Inist), avec lequel nous collaborons et dont les activités
    s'organisent autour de la \textsc{Bsn}. Créé en 1988, l'Inist possède l'une
    des plus importantes collections de publications scientifiques d'Europe et
    fournit plusieurs services pour la recherche d'information (\textsc{Ri}),
    dont le maintient de bases de données bibliographiques. Ces dernières sont
    composées de notices bibliographiques décrivant les documents
    scientifiques~: titre, auteur(s), résumé et termes-clés. Parmi ces
    métadonnées, les termes-clés font partie des plus importantes pour la
    \textsc{Ri}. Ce sont des mots ou des expressions qui représentent le contenu
    principal d'un document\footnote{Un terme-clé est plus communément appelé
    mot-clé. Cependant, un mot-clé n'étant pas uniquement monolexical, nous
    utilisons la notion de terme-clé pour lever toute ambiguïté. Lorsque dans la
    suite nous parlons de mots-clés, cela ne concerne donc que les
    monolexicaux.}. Ils permettent donc de le résumer, de le catégoriser et de
    définir les éléments discriminants pour déterminer s'il correspond à la
    requête d'une personne cherchant une information particulière. Lorsque
    réalisée par les auteurs des documents ou par un quelconque lecteur,
    l'attribution des termes-clés aux documents est une tâche subjective. Dans
    un soucis d'homogénéité de l'indexation de tous les documents, de conformité
    du vocabulaire utilisé, d'exhaustivité et d'impartialité, les organismes
    tels que l'Inist emploient des indexeurs professionnels (ingénieurs
    documentalistes) qui attribuent les termes-clés dans le respect des
    pratiques documentaires et du langage du domaine de spécialité des
    documents~\cite{guinchat1996techniquesdocumentaires}.

    L'indexation manuelle des documents par leurs termes-clés est une tâche
    coûteuse et chronophage. Soucieux de faciliter le travail d'indexation par
    termes-clés, que ce soit pour des productions scientifiques ou des documents
    d'autres natures, de nombreux chercheurs s'intéressent à son automatisation,
    en témoignent le nombre grandissant d'articles scientifiques à ce
    sujet~\cite{hasan2014state_of_the_art} ainsi que l'émergence de campagnes
    d'évaluation automatique~\cite{kim2010semeval,paroubek2012deft}. Dans cette
    thèse, nous proposons des méthodes pour l'indexation par termes-clés.
    Travaillant d'abord d'un point de vu généraliste, nous nous focalisons
    ensuite sur l'indexation par termes-clés en domaines de spécialités.

  %-----------------------------------------------------------------------------

  \section{Problématique}
  \label{sec:main-introduction-problem_statement}
    La tâche d'indexation automatique par termes-clés consiste à analyser chaque
    document un à un pour en déterminer le contenu principal. Nous distinguons
    deux catégories d'indexation par termes-clés~: l'une libre extrait parmi les
    unités textuelles du document celles qui représentent le mieux sont contenu,
    l'autre contrôlée assigne au document les entrées d'une terminologie qui le
    décrivent le mieux. Indexation libre et contrôlée possèdent toutes les deux
    un avantage et un inconvénient. L'indexation libre, ou extraction de
    termes-clés, a l'avantage d'extraire des termes-clés qui représentent des
    concepts nouveaux, alors que l'indexation contrôlée, ou assignement de
    termes-clés, ne peut le faire que dès lors que la terminologie qu'elle
    utilise a été mise à jour manuellement. Par ailleurs, l'indexation libre
    extrait les termes-clés tels qu'ils apparaissent dans les documents,
    c'est-à-dire dans leur forme qui n'est pas toujours canonique et donc
    imparfaite, alors que l'indexation contrôlée assigne toujours des
    termes-clés dans leur forme canonique (appropriée).
    
    Que les termes-clés possibles (candidats) soient les unités textuelles du
    document ou les entrées d'une terminologie, la difficulté de la tâche
    d'indexation par termes-clés réside dans l'identification des relations
    entre les candidats dans le document afin de déterminer lesquels sont les
    plus importants (les termes-clés), ou dans l'identification de leurs
    caractéristiques discriminantes afin de déterminer lesquels sont les
    termes-clés et lesquels ne le sont pas. L'identification des relations entre
    les termes-clés candidats n'a été utilisée que pour l'extraction de
    termes-clés (indexation libre) et l'identification des caractéristiques
    discriminantes a été utilisée indifféremment pour l'extraction de
    termes-clés et l'assignement de termes-clés (indexation contrôlée).

    L'extraction de termes-clés, lorsqu'elle s'intéresse aux relations entre les
    termes-clés candidats peut être réalisée de manière non supervisée,
    c'est-à-dire qu'aucune intervention humaine n'est nécessaire pour aider un
    système à trouver les termes-clés. Lorsqu'il s'agit d'identifier les
    caractéristiques discriminantes d'un terme-clé, extraction et assignement de
    termes-clés sont réalisés de manière supervisée, c'est-à-dire que des
    données d'entraînement dont nous connaissons les termes-clés sont données au
    système pour qu'il s'y adapte. L'approche non supervisée est intéressante,
    car elle n'est dépendante d'aucune ressource et peut donc être appliquée
    sans coût à n'importe quel document, alors que l'approche supervisée
    nécessite la disponibilité de ressources déjà indexées ou de ressources
    humaines pour les produire. Néanmoins, ce sont actuellement les méthodes
    supervisées les plus performantes pour l'indexation automatique par
    termes-clés.

  %-----------------------------------------------------------------------------

  \section{Hypothèses}
  \label{sec:main-introduction-hypothesis}
    Nous faisons le choix de travailler sur trois aspects de l'indexation par
    termes-clés~: la sélection des termes-clés candidats pour l'extraction de
    termes-clés, l'extraction non supervisée de termes-clés et l'indexation de
    termes-clés en domaines de spécialités. Chaque aspect donne lieu à une
    contribution qui, une fois réunie aux deux autres, permet d'obtenir une
    unique méthode capable d'indexer des documents par des termes-clés de
    qualité linguistique, spécifiques vis-à-vis de chaque document et respectant
    autant que possible les pratiques documentaires.

    La sélection des termes-clés candidats détermine quelles unités textuelles
    du document (mots et expressions) sont potentiellement des termes-clés.
    Après une étude des termes-clés déjà attribués à des documents, nous nous
    focalisons sur la catégorie des adjectifs qui composent les termes-clés et
    proposons de sélectionner les groupes nominaux et de filtrer de ces derniers
    certains adjectifs jugés superflus.

    L'extraction non supervisée de termes-clés a récemment été étudiée à partir
    d'une représentation du document sous la forme d'un graphe. Les mots du
    document forment les n\oe{}uds du graphe et les arêtes entre les mots
    indiquent qu'ils cooccurrent dans le document. Un algorithme qui mesure la
    centralité est appliqué à ce graphe afin de déterminer les mots centraux
    (importants) du document, et les plus longues séquences de mots centraux
    sont extraites comme termes-clés du document. Convaincus que le graphe doit
    servir à ordonner les termes-clés candidats, plutôt que les mots, et que
    certains n\oe{}uds du graphe sont sémantiquement équivalents, nous proposons
    de grouper les termes-clés candidats qui véhiculent le même sujet,
    d'utiliser ces sujets comme n\oe{}uds du graphe et d'extraire un terme-clé
    par sujet.
    
    L'indexation par termes-clés est toujours réalisée par extraction de
    termes-clés ou assignement de termes-clés séparément. Cependant, nous
    étudions la méthodologie d'indexation manuelle en domaines de spécialités et
    montrons qu'extraction et assignement doivent être effectués conjointement~:
    l'extraction pour assurer l'exhaustivité des termes-clés et l'assignement
    pour assurer homogénéité et conformité. Nous proposons d'utiliser des
    données d'entraînement pour créer une représentation d'un domaine sous forme
    de graphe et d'unifier ce graphe à celui qui représente le document afin de
    réaliser conjointement extraction et assignement de termes-clés.

  %-----------------------------------------------------------------------------

  \section{Plan de thèse}
  \label{sec:main-introduction-outline}
    Cette thèse est organisée de la manière suivante. Tout d'abord, le
    chapitre~\ref{chap:main-state_of_the_art} présente l'état de l'art en
    indexation automatique par termes-clés, puis le
    chapitre~\ref{chap:main-data_description} introduit les données avec
    lesquelles nous travaillons. Nos contributions sont détaillées dans les
    chapitres~\ref{chap:main-domain_independent_keyphrase_extraction}
    et~\ref{chap:main-domain_specific_keyphrase_annotation}. Le
    chapitre~\ref{chap:main-domain_independent_keyphrase_extraction} s'intéresse
    au deux contributions à l'extraction de termes-clés et le
    chapitre~\ref{chap:main-domain_specific_keyphrase_annotation} s'intéresse à
    notre contribution à l'indexation par termes-clés en domaines de
    spécialités. Ce dernier présente tout d'abord la méthodologie d'indexation
    manuelle en domaine de spécialité, il présente ensuite la méthode que nous
    proposons pour s'en approcher et il termine par la description et l'analyse
    des résultats d'une campagne d'évaluation manuelle que nous avons réalisé
    pour évaluer notre travail. Finalement, le
    chapitre~\ref{chap:main-conclusion} dresse le bilan de notre travail et
    présente quelques perspectives.

