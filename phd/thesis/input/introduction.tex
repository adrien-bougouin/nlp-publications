\chapter{Introduction}
\label{chap:main-introduction}
  \chaptercite{
    Seeking information is an important activity for everyone who uses computers
    in daily life. While the Internet is a bountiful source of all types of
    knowledge, locating relevant documents is still a great challenge [\dots]
    Keyphrases help organise documents and retrieve them based on content.
  }{
    \newcite{medelyan2008smalltrainingset}
  }

  %-----------------------------------------------------------------------------

  \section{Contexte}
  \label{sec:main-introduction-context}
    Avec l'essor du numérique, le Web occupe aujourd'hui une place importante
    dans notre société. Celui-ci contient tous types d'informations
    (culturelles, historiques, scientifiques, etc.) qu'il rend disponibles pour
    tous. Cependant, le Web est en constante expansion et le nombre croissant
    d'informations disponibles complique leur accès, leur recherche. Pour
    résoudre ce problème, il faut représenter et organiser efficacement les
    documents numériques. Dans le contexte de la recherche scientifique, en
    domaines de spécialités, il est important de résoudre ce problème, car
    favoriser l'accès aux productions scientifiques favorise les avancées
    scientifiques. C'est pourquoi sont créées des bibliothèques numériques
    telles que la Bibliothèque Scientifique Numérique (\textsc{Bsn}) fondée en
    2009 par le ministère de l'enseignement supérieur et de la recherche
    français.

    Afin de mieux comprendre comment l'accès aux informations peut être
    facilité, prenons l'exemple de l'institut de l'information scientifique et
    technique (Inist), avec lequel nous collaborons et dont les activités
    s'organisent autour de la \textsc{Bsn}. Créé en 1988, l'Inist possède l'une
    des plus importantes collections de publications scientifiques d'Europe et
    fournit plusieurs services pour la Recherche d'Information (\textsc{Ri}),
    dont le maintient de bases de données bibliographiques. Ces dernières sont
    composées de notices bibliographiques décrivant des documents
    scientifiques~: titre, auteur(s), résumé et termes-clés. Parmi ces
    métadonnées, l'ensemble des termes-clés est l'une des plus importantes pour
    la recherche d'information. Les termes-clés sont des mots ou des expressions
    qui représentent le contenu principal d'un document\footnote{Un terme-clé
    est plus communément appelé mot-clé. Cependant, un mot-clé n'étant pas
    uniquement monolexical, nous utilisons la notion de \textit{terme-clé} pour
    lever toute ambiguïté. Lorsque dans la suite nous parlons de
    \textit{mots-clés}, cela ne concerne donc que les monolexicaux.}. Ils
    permettent de résumer les documents, de les catégoriser et de définir les
    éléments discriminants pour retrouver ceux qui répondent à une requête d'un
    utilisateur. Les termes-clés sont parfois fournis par les auteurs, mais leur
    identification est subjective et peut varier selon les individus. Dans un
    soucis d'homogénéité et de respect des pratiques
    documentaires~\cite{guinchat1996techniquesdocumentaires}, les organismes
    tels que l'Inist font appel à des indexeurs professionnels (ingénieurs
    documentalistes) qui identifient des termes-clés en assurant, entre autres,
    le respect du vocabulaire spécifique au domaine de spécialité du document
    (vocabulaire contrôlé).

    L'indexation manuelle des documents par leurs termes-clés est une tâche
    coûteuse et chronophage. Soucieux de faciliter le travail d'indexation par
    termes-clés, que ce soit pour des articles scientifiques ou des documents
    d'autres natures, de nombreux chercheurs s'intéressent à son automatisation,
    en témoignent le nombre grandissant d'articles scientifiques à ce
    sujet~\cite{hasan2014state_of_the_art} ainsi que l'émergence de campagnes
    d'évaluation~\cite{kim2010semeval,paroubek2012deft}. Dans cette thèse, nous
    proposons des méthodes pour l'indexation par termes-clés. Travaillant d'un
    point de vu généraliste dans un premier temps, nous nous focalisons dans un
    second temps sur les domaines de spécialités (des \textsc{Bsn}).

  %-----------------------------------------------------------------------------

  \section{Problématique}
  \label{sec:main-introduction-problem_statement}
    \begin{itemize}
      \item{Définition de la tâche d'indexation automatique par termes-clés}
      \item{Extraction Vs. assignement}
      \item{Non supervisé Vs. supervisé}
      \item{État actuel de la tâche~: point sur les performances}
    \end{itemize}

  %-----------------------------------------------------------------------------

  \section{Hypothèses}
  \label{sec:main-introduction-hypothesis}

  %-----------------------------------------------------------------------------

  \section{Plan de thèse}
  \label{sec:main-introduction-outline}

