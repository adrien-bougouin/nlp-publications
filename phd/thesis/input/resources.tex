\chapter{Ressources}
\label{chap:main-data_description}
  \chaptercite{
    [\dots] to fully understand the strengths and weaknesses of a keyphrase
    extractor, it is essential to evaluate it on multiple datasets.
  }{
    \newcite{hassan2010conundrums}
  }

  \section{Introduction}
  \label{sec:main-data_description-introduction}
    Pour les travaux de recherche en indexation automatique par termes-clés, des
    collections de données sont nécessaires à l'évaluation et à la comparaison
    des nouveaux travaux aux précédents. De nombreuses collections sont
    accessibles publiquement\footnote{Un grand nombre de collections de données
    est accessible depuis le dépôt GitHub de Su Nam Kim (\textit{snkim})~:
    \url{https://github.com/snkim/AutomaticKeyphraseExtraction}.}, elles
    couvrent différentes langues (français, anglais, etc.), des documents de
    différentes natures (résumés, articles scientifiques, articles
    journalistiques, etc.) et différents domaines (météorologie, sciences
    humaines et sociales, informatique, etc.). Cette diversité est essentielle à
    la compréhenssion des points forts et des points faibles d'une
    méthode~\cite{hassan2010conundrums}. En effet, différents facteurs peuvent
    influencer les performances des méthodes d'indexation par termes-clés.
    \newcite{hasan2014state_of_the_art} en énoncent quatre~:
    \begin{itemize}
      \item{si un document est long, alors le nombre de termes-clés candidats
            pour celui-ci est important et l'indexation par termes-clés est plus
            difficile que pour un document court~;}
      \item{si le contenu d'un document est structuré (par exemple un article
            scientifique réparti en sections), alors une méthode tenant compte
            de cette répartition est avantagée~;}
      \item{si des changements de sujets surviennent dans un document, une
            méthode qui utilise la position de la première occurrence des
            candidats risque d'avoir plus de difficultés~;}
      \item{si des sujets sans relation sont abordés dans un même document,
            alors une méthode qui tisse des liens sémantiques entre les
            termes-clés candidats est pénalisée.}
    \end{itemize}
    \TODO{Le type d'annotation est un autre facteur\dots}
    \TODO{La conversion en plein text est un autre facteur\dots}

    Dans nos travaux de recherche, nous utilisons cinq collections de données~:
    Termith, \textsc{Deft}, Wikinews, SemEval et \textsc{Duc}. En accord avec la
    vision de \newcite{hassan2010conundrums}, celles-ci couvrent deux langues
    (français et anglais) trois natures de documents (résumé, articles
    scientifiques et articles journalistiques) et un large évantail de domaines
    (sciences humaines et sociales, informatique, météorologie, catastrophes
    naturelles, etc.). Dans la suite, nous présentons ces six resources.

  %-----------------------------------------------------------------------------

  \section{Termith}
  \label{sec:main-data_description-termith_data}
    \TODO{penser a parler des referentiels}
    %% Présentation des collections 
    Dans le cadre du projet Termith, nous disposons de cinq collections de
    notices bibliographiques fournies par l'Inist. Ces cinq collections
    représentent cinq disciplines~: l'archéologie, les sciences de
    l'information, la linguistique, la psychologie et la chimie. Le corpus
    d'archéologie est composé de 718 notices représentant des articles français
    parus entre 2001 et 2012 dans 22 revues (\textit{Paléo}, \textit{Le bulletin
    de la Société préhistorique française}, etc.)~; Le corpus des sciences de
    l'information contient 706 notices d'articles français publiés entre 2001 et
    2012 dans six revues (\textit{Documentaliste -- Sciences de l'information},
    \textit{Document numérique}, etc.)~; Le corpus de linguistique est constitué
    de 715 notices d'articles français parus entre 2000 à 2012 dans 12 revues
    (\textit{Linx~---~Revue des linguistes de l'Université Paris Ouest Nanterre
    La Défense}, \textit{Travaux de linguistique}, etc.)~; Le corpus de
    psychologie contient 720 notices d'articles français publiés entre 2001 et
    2012 dans sept revues (\textit{Enfance}, \textit{Revue internationale de
    psychologie et de gestion des comportements organisationnels}, etc.)~; Le
    corpus de chimie est composé de 782 notices d'articles français publiés
    entre 1983 et 2012 dans quatre revues (\textit{Comptes Rendus de l'Académie
    des Sciences}, \textit{Comptes Rendus Chimie}, etc.).
    
    %% Explication du procesus decréation des notices
    Chaque notice contient le titre, le résumé et les termes-clés associés au
    document qu'elle représente. Au total, il peut y avoir quatre ensembles de
    termes-clés différents~: les termes-clés des auteurs et les termes-clés des
    indexeurs de l'Inist en français, en anglais et en espagnole. Parmi ces
    quatre ensembles, nous prenons pour référence les termes-clés français
    fournis par les indexeurs de l'Inist, car notre objectif est d'automatiser
    l'indexation telle qu'elle est effectuée à l'Inist. Le processus
    d'indexation de l'Inist se déroule en deux étapes~: la reconnaissance des
    concepts contenant l'information dans les documents à indexer, puis la
    représentation de ces concepts dans le language documentaire (le vocabulaire
    contrôlé doit être utilisé en priorité). Ce processus fourni donc aussi bien
    des termes-clés présents dans les documents que des termes-clés qui n'y sont
    pas.

    Le tableau~\ref{tab:statistiques_des_corpus} présente les caractéristiques
    principales des collections de notices présentées ci-dessus. Elles sont de
    petite taille et sont rédigées différemment selon les disciplines
    (cf.~figure~\ref{fig:exemple_notice_inist}). Les notices d'archéologie, par
    exemple, font l'objet d'un effort de présentation du contexte historique lié
    aux travaux présentés, tandis que les notices de chimie, principalement des
    comptes rendus d'expériences, décrivent sommairement les expériences
    réalisées (noms des expériences, éléments chimiques impliqués, etc.). Les
    termes-clés associés aux documents varient en nombre et en complexité. En
    archéologie, par exemple, nous observons qu'un grand nombre de termes-clés
    sont des entités nommées principalement composées d'un seul mot
    (p. ex.~\og{}Paléolithique\fg{}, \og{}Europe\fg{}, etc.), tandis qu'en
    chimie, nous observons un usage fréquent de notions générales (dans la
    discipline) nécessitant une spécialisation presque systématique
    (p. ex..~\og{}\underline{réaction} topotactique\fg{},
    \og{}\underline{réaction} sonochimique\fg{}, \og{}\underline{réaction}
    électrochimique\fg{}, etc.). Enfin, il est important de noter la faible
    proportion de termes-clés apparaissant dans les notices -- rappel maximum
    pouvant être obtenu. Par exemple, dans le corpus des sciences de
    l'information, uniquement 2,8 termes-clés peuvent être extraits des notices
    parmi les 8,7 associés aux notices, en moyenne. Il est donc important de ne
    pas utiliser que le contenu du résumé des notices pour extraire les
    termes-clés.

    \begin{table}[!h]
      \centering
      \resizebox{\linewidth}{!}{
        \begin{tabular}{l|c@{~~}c@{~~}c@{~~}c|c@{~~}c@{~~}c@{~~}c}
          \toprule
          \multirow{2}{*}{\textbf{Corpus}} & \multicolumn{4}{c|}{\textbf{Documents}} & \multicolumn{4}{c}{\textbf{Termes-clés}}\\
          \cline{2-9}
          & Langue & Nature & Quantité & Mots moy. & Annotateur & Quantité moy. & \og{}À assigner\fg{} & Mots moy.\\
          \hline
          Linguistique & & & & & & &\\
          \hfill{}Appr. & Français & Résumé & 515 & 160,5 & Professionnel & 8,6 & 60,6~\% & 1,7\\
           \hfill{}Test & \ditto & \ditto & 200 & 147,0 & \ditto & 8,9 & 62,8~\% & 1,8\\
          \hline
          Sciences de l'infor- & & & & & & &\\
          mation\hfill{}Appr. & Français & Résumé &  & & & &\\
          \hfill{}Test & \ditto & \ditto &  & & & &\\
          \hline
          Psychologie & & & & & & &\\
          \hfill{}Appr. & Français & Résumé &  & & & &\\
          \hfill{}Test & \ditto & \ditto &  & & & &\\
          \hline
          Archéologie & & & & & & &\\
          \hfill{}Appr. & Français & Résumé &  & & & &\\
          \hfill{}Test & \ditto & \ditto &  & & & &\\
          \hline
          Chimie & & & & & & &\\
          \hfill{}Appr. & Français & Résumé &  & & & &\\
          \hfill{}Test & \ditto & \ditto &  & & & &\\
          \bottomrule
        \end{tabular}
      }

      \caption{Corpus Termith
               \label{tab:termith}}
    \end{table}

% FIXME linguistique pas a jour
%    \begin{table}
%      \centering
%      \begin{tabular}{@{~}r|c@{~~}c@{~~}c@{~~}c@{~~}c@{~}}
%        \toprule
%        & & \textbf{Sciences} & & &\\ \textbf{Statistique} & \textbf{Archéologie} & \textbf{de} & \textbf{Linguistique} & \textbf{Psychologie} & \textbf{Chimie}\\ & & \textbf{l'Information} & & &\\
%        \hline
%        Documents & 718 & 706 & 715 & 720 & 782\\
%        Mots/doc. & 219,1 & 119,7 & 156,7 & 185,7 & 105,2\\
%        Termes-clés/doc. & 16,6 & 8,5 & 8,7 & 11,6 & 12,8\\
%        Mots/terme-clé & 1,3 & 1,7 & 1,8 & 1,6 & 2,2\\
%        Rappel max. & 62,9~\% & 32,4~\% & 38,8~\% & 27,1~\% & 23,7~\%\\
%        \bottomrule
%      \end{tabular}
%      \caption{Caractéristiques des cinq corpus disciplinaires. La ligne
%               \textit{Rappel max.} indique le pourcentage de termes-clés
%               pouvant être extraits à partir du titre ou du résumé des notices.
%               Cela donne un aperçu des performances maximales que peuvent
%               atteindre des méthodes tenant uniquement compte du contenu des
%               documents traités.
%               \label{tab:statistiques_des_corpus}}
%    \end{table}

    \begin{figure}
%      \framebox[\linewidth]{ % archeologie_09-0054907
%        \parbox{.99\linewidth}{\small
%          \textbf{Variabilité du Gravettien de Kostienki (bassin moyen du Don)
%          et des ter-}
%          ~\hfill\underline{\textit{Archéologie}}\\
%          \textbf{ritoires associés}\\
%
%          Dans la région de Kostienki-Borschevo, on observe l'expression, à ce
%          jour, la plus orientale du modèle européen de l'évolution du
%          Paléolithique supérieur. Elle est différente à la fois du modèle
%          Sibérien et du modèle de l'Asie centrale. Comme ailleurs en Europe, le
%          Gravettien apparaît à Kostienki vers 28 ka (Kostienki 8 /II/). Par la
%          suite, entre 24-20 ka, les techno-complexes gravettiens sont
%          représentés au moins par quatre faciès dont deux, ceux de Kostienki
%          21/III/ et Kostienki 4 /II/, ressemblent au Gravettien occidental et
%          deux autres, Kostienki-Avdeevo et Kostienki 11/II/, sont des faciès
%          propres à l'Europe de l'Est, sans analogie à l'Ouest.\\
%
%          \textbf{Termes-clés~:} \underline{Europe}, Kostienko,
%          \underline{Borschevo}, variation, typologie, industrie osseuse,
%          industrie lithique, Europe centrale, \underline{Avdeevo},
%          \underline{Paléolithique supérieur}, \underline{Gravettien}.
%        }
%      }
%      ~\\~\\
      \framebox[\linewidth]{ % linguistique_08-0265302
        \parbox{.99\linewidth}{\small
          \textbf{Termes techniques et marqueurs d'argumentation : pour
          débusquer}
          \hfill\underline{\textit{Linguistique}}\\
          \textbf{l'argumentation cachée dans les articles de recherche}\\

          Les articles de recherche présentent les résultats d'une expérience
          qui modifie l'état de la connaissance dans le domaine concerné. Le
          lecteur néophyte a tendance à considérer qu'il s'agit d'une simple
          description et à passer à côté de l'argumentation au cours de laquelle
          le scientifique cherche à convaincre ses pairs de l'innovation et de
          l'originalité présentées dans l'article et du bien-fondé de sa
          démarche tout en respectant la tradition scientifique dans laquelle il
          s'insère. Ces propriétés spécifiques du discours scientifique peuvent
          s'avérer un obstacle supplémentaire à la compréhension, surtout
          lorsqu'il s'agit d'un article en langue étrangère. C'est pourquoi il
          peut être utile d'incorporer dans l'enseignement   des langues de
          spécialité une sensibilisation aux marqueurs linguistiques
          (terminologiques et argumentatifs), qui permettent de dépister le
          développement de cette rhétorique. Les auteurs s'appuient sur deux
          articles dans le domaine de la microbiologie.\\

          \textbf{Termes-clés~:} Langue scientifique, \underline{argumentation},
          \underline{rhétorique}, \underline{langue de spécialité},
          \underline{enseignement des langues}, linguistique appliquée,
          \underline{discours scientifique}, \underline{article de recherche}. 
        }
      }
%      ~\\~\\
%      \framebox[\linewidth]{ % chimie_90-0137940
%        \parbox{.99\linewidth}{\small
%          \textbf{Etude d'un condensat acide
%          isocyanurique-urée-formaldéhyde}
%          \hfill\underline{\textit{Chimie}}\\
%
%          La synthèse d'un condensat acide isocyanurique-urée-formaldéhyde
%          utilisant la pyridine en tant que solvant a été effectuée par réaction
%          sonochimique.\\
%
%          \textbf{Termes-clés~:} \underline{Réaction sonochimique}, hétérocycle
%          azote, cycle 6 chaînons, ether.
%        }
%      }
      \caption[Exemple de notice Inist]{
        Exemple de notice Inist. Les termes-clés soulignés sont ceux qui peuvent
        être extraits.
        \label{fig:exemple_notice_inist}
      }
    \end{figure}

  %-----------------------------------------------------------------------------

  \section[\textsc{Deft}]{\textsc{Deft}~\textnormal{\large\cite{paroubek2012deft}}}
  \label{sec:main-data_description-deft_data}
    \textsc{Deft} est une campagne d'évaluation francophone qui chaque année
    s'intéresse à un domaine particulier du \textsc{Tal}. Le corpus éponyme que
    nous utilisons dans nos travaux est la collection de documents construite
    dans le cadre de l'édition 2012 de \textsc{Deft}, édition axée sur
    l'extraction de termes-clés, d'une part, et sur l'assignement de
    termes-clés, d'une autre part. Le corpus \textsc{Deft} est composé de 234
    articles français publiés entre 2003 et 2008 dans quatre revues des Sciences
    Humaines et Sociales~: \textit{Anthropologie et Sociétés},
    \textit{\textsc{Meta}: Research in Hermeneutics, Phenomenology, and
    Practical Philosophy}, \textit{Revue des Sciences de l'Éducation} et
    \textit{\textsc{Ttr}~: traduction, terminologie, rédaction}. Il s'agit du
    corpus utilisé pour la tâche d'extraction de termes-clés.
    
    Le tableau~\ref{tab:deft} présente les différentes caractéristiques du
    corpus \textsc{Deft}. Celui-ci est divisé en deux sous-ensembles, un
    ensemble d'apprentissage (appr.) composé de 141 articles et un ensemble de
    test contenant 93 articles. \TODO{revenir sur les données du tableau + liens
    avec les facteurs énoncés précédemments} \TODO{répartition équitable}
    \begin{table}[!h]
      \centering
      \resizebox{\linewidth}{!}{
        \begin{tabular}{l|c@{~~}c@{~~}c@{~~}c|c@{~~}c@{~~}c@{~~}c}
          \toprule
          \multirow{2}{*}{\textbf{Corpus}} & \multicolumn{4}{c|}{\textbf{Documents}} & \multicolumn{4}{c}{\textbf{Termes-clés}}\\
          \cline{2-9}
          & Langue & Nature & Quantité & Mots moy. & Annotateur & Quantité moy. & \og{}À assigner\fg{} & Mots moy.\\
          \hline
          \hfill{}Appr. & Français & Scientifique & 141 & 7~276,7 & Auteur & 5,4 & 18,2~\% & 1,7\\
          \hfill{}Test & \ditto & \ditto & ~~93 & 6~839,4 & \ditto & 5,2 & 21,1~\% & 1,6\\
          \bottomrule
        \end{tabular}
      }
      \caption{Corpus \textsc{Deft}
               \label{tab:deft}}
    \end{table}

    Le corpus \textsc{Deft} est ausssi un corpus au contenu bruité, inparfait.
    Les articles, originalement au format \textsc{Pdf} (\textit{Portable
    Document Format}), sont convertis au format \textsc{Text}. Cependant, la
    conversion n'est pas parfaite~: des caractères spéciaux ne sont pas reconnus
    et la segmentation en paragraphes a parfois lieu en milieu de phrase. Les
    figures~\ref{fig:example_deft_ko}~et~\ref{fig:example_deft_ok} montrent deux
    exemples d'un document bruité et d'un document non bruité, respectivement.
    \begin{figure}
      \framebox[\linewidth]{ % meta_2005_019828ar
        \parbox{.99\linewidth}{\small

          ~~~~Considérée comme une \og{}problem solving activity\fg{} (Guilford
          1975), la créativité, démystifiée, fait partie du quotidien du
          traducteur. Victimes d'idées préconçues et erronées sur la notion de
          \og{}fidélité\fg{}, beaucoup de traducteurs sont insécurisés face à
          leur créativité. Ils peuvent alors, comme en témoigne un de nos
          exemples, manquer de courage et jouer la carte de la stratégie du
          \og{}playing it safe\fg{}, ou bien, lorsque, comme dans un autre cas,
          leur statut social et professionnel leur donne une certaine assurance,
          garder leurs solutions créatives et revendiquer leur
          \og{}trahison\fg{}, toutefois sans pour autant essayer de trouver des
          légitimations à leurs solutions. Légitimations qui restent la plupart
          du temps au stade de \og{}mécanismes de justification\fg{} ponctuels.
          Une analyse des besoins nous permet de montrer comment ces
          justifications hétéroclites et éparses peuvent venir s'intégrer dans
          un édifice théorique cohérent, s'appuyant notamment sur des fondements
          cognitivistes, susceptible de donner au traducteur le courage de sa
          créativité.

          ~~~~Pour pouvoir déterminer l.utilité d.un quelconque apport théorique
          à la pratique du traducteur, il

          ~~~~faut commencer par examiner s.il existe un besoin en la matière et
          quelle en est la nature. Nous le

          ~~~~ferons à l.aide de deux corpus qui se complètent. Le premier est
          la transcription du débat mené par

          ~~~~un groupe de quatre
          \og{}\&amp;\#x00A0;semi-professionnels\&amp;\#x00A0;\fg{} de
          l.Institut de traducteurs et interprètes de
          
          ~~~~[\dots]\\

          \textbf{Termes-clés~:} \underline{créativité}, didactique de la
          traduction, \underline{cognitivisme}, analyse conversationnelle,
          théorie de la traduction. 
        }
      }
      \caption[Exemple de document de \textsc{Deft}]{
        Exemple de document de \textsc{Deft}. Les termes-clés soulignés sont
        ceux qui peuvent être extraits.
        \label{fig:example_deft_ko}
      }
    \end{figure}
    \begin{figure}
      \framebox[\linewidth]{ % as_2004_008571ar
        \parbox{.99\linewidth}{\small
          ~~~~Bien qu'un grand nombre de travaux ethnographiques novateurs aient
          été suscités par l'\og{}espace interculturel\fg{} que se partagent
          Australiens autochtones et non autochtones, notamment dans le domaine
          des arts visuels, les chercheurs ont accordé moins d'attention aux
          représentations rituelles publiques auxquelles les Aborigènes ont
          donné un nouvel essor en tant qu'instruments politiques. On a encore
          moins écrit sur la (re)construction interne de l'identité sociale
          autochtone et sa projection dans la production de rituels publics sur
          la scène néocoloniale australienne contemporaine. Tout en effectuant
          une remise à jour des recherches précédentes sur la question, le
          présent article montre comment, au cours des dix dernières années, les
          leaders rituelles aînées d'une petite localité d'Australie centrale
          ont inauguré une phase entièrement nouvelle de représentations
          rituelles - une phase qui diffère substantiellement des formes
          antérieures d'expérience cérémonielle, qui étaient étroitement liées à
          la négociation et à l'échange des matériaux rituels.

          ~~~~Pour M. Nampijinpa L.

          ~~~~Depuis que l'anthropologie \og{}a découvert\fg{} la religion
          australienne – à partir du milieu du XIXe siècle avec les ouvrages de
          Spencer et Gillen dont les travaux de terrain ont alimenté les
          recherches de Durkheim, ethnologue en chambre, et de ses héritiers -
          on s'est beaucoup intéressé aux manifestations rituelles de la
          cosmologie aborigène connue sous le nom de \og{}Dreaming\fg{},
          c'est-à-dire \og{}Rêve\fg{} ou \og{}Récit du Rêve\fg{}. Et bien que la
          fréquence de ce type de représentations cérémonielles ait diminué chez
          les Aborigènes, cette diminution quantitative n'affecte en rien les
          résultats analytiques issus de l'étude des usages contemporains du
          champ rituel. De fait, les modifications fonctionnelles apportées aux
          cérémonies aborigènes, en raison de l'évolution spectaculaire de leurs
          objectifs et de leur structure, offrent au chercheur une compréhension
          inédite de la construction dynamique de l'identité sociale autochtone
          dans un contexte où la pression coloniale perdure.
          
          ~~~~[\dots]\\

          \textbf{Termes-clés~:} \underline{dussart}, \underline{aborigènes},
          \underline{femmes}, \underline{identité}, \underline{rituel},
          \underline{warlpiri}, \underline{australie}.
        }
      }
      \caption[Autre exemple de document de \textsc{Deft}]{
        Autre exemple de document de \textsc{Deft}. Les termes-clés soulignés
        sont ceux qui peuvent être extraits.
        \label{fig:example_deft_ok}
      }
    \end{figure}

    \newcite{paroubek2012deft} établissent un point de comparaison à l'aide
    d'étudiants de master en ingénierie multilingue. Le
    tableau~\ref{tab:deft_human_tests} montre les résultats de l'indexation par
    termes-clés obtenus pour chacun de ces étudiants. Ceux-ci ont jugé la tâche
    difficile, comme en témoigne les faibles résultats obtenus (f-mesure moyenne
    de 21,6~\%). L'indexation par terme-clé est subjective et ils éprouvent des
    difficultés dans le cas ou une expression dans le texte est reformulée. Ils
    donnent l'exemple de \og{}traduction française et allemande\fg{} qui est
    représenté par le terme-clé \og{}traduction allemande et traduction
    française\fg{}. Ils notent aussi la présence de termes-clés thématiquement
    redondants et soulignent la contre-intuitivité de ce cas de figure. Ils
    donnent l'exemple des termes-clés \og{}interprète\fg{} et
    \og{}interprétation\fg{}. Ces conclusions ne sont pas encourageantes pour
    l'indexation automatique par termes-clés. Il est possible que les problèmes
    rencontrés lors des tests humains soient dus à la nature des annotations. La
    redondance qui semble contre-intuitive en est un exemple. Nous pouvons en
    effet supposer qu'un auteur à recours à ce genre de procédé pour être
    certains d'attirer tout lecteur potentiel, par exemple l'un effectuant une
    recherche par mot-clé avec \og{}interprète\fg{} et l'autre avec
    \og{}interprétation\fg{}.
    \begin{table}[!h]
      \centering
      \begin{tabular}{l|ccccccc}
        \toprule
          \textbf{Mesure} & \textbf{P1} & \textbf{P2} & \textbf{P3} & \textbf{P4} & \textbf{P5} & \textbf{P6} & \textbf{P7}\\
        \hline
        Précision~\hfill(\%) & 25,0 & 20,0 & 16,7 & 11,8 & 29,2 & 29,2 & 20,8\\
        Rappel~\hfill(\%) & 25,0 & 20,8 & 16,7 & ~~8,3 & 29,2 & 29,2 & 20,8\\
        F-mesure~\hfill(\%) & 25,0 & 20,4 & 16,7 & ~~9,8 & 29,2 & 29,2 & 20,8\\
        \bottomrule
      \end{tabular}
      \caption[Résultats de tests humains sur le corpus \textsc{Deft}]{
        Résultats de tests humains (sept personnes --- P1$..$P7) sur le corpus
        \textsc{Deft}
        \label{tab:deft_human_tests}
      }
    \end{table}

  %-----------------------------------------------------------------------------

  \section{Wikinews}
  \label{sec:main-data_description-wikinews_data}
    Wikinews est une collection de 100 articles journalistiques en français
    que nous avons collecté sur site web d'information collabotatif
    WikiNews\footnote{\url{http://fr.wikinews.org/}} entre les mois de mai et
    décembre 2012\footnote{Les documents et les annotations du corpus Wikinews
    sont disponibles sur le dépôt GitHub suivant~:
    \url{https://github.com/adrien-bougouin/WikinewsKeyphraseCorpus}}. Le
    tableau~\ref{tab:wikinews} donne les détails de ce corpus. \TODO{Chaque
    document a été annoté par au moins trois étudiants. Les termes-clés des
    différents étudiants ont été groupés et les redondances lexicales ont été
    automatiquement supprimées.}

    \begin{table}[!h]
      \centering
      \resizebox{\linewidth}{!}{
        \begin{tabular}{l|c@{~~}c@{~~}c@{~~}c|c@{~~}c@{~~}c@{~~}c}
          \toprule
          \multirow{2}{*}{\textbf{Corpus}} & \multicolumn{4}{c|}{\textbf{Documents}} & \multicolumn{4}{c}{\textbf{Termes-clés}}\\
          \cline{2-9}
          & Langue & Nature & Quantité & Mots moy. & Annotateur & Quantité moy. & \og{}À assigner\fg{} & Mots moy.\\
          \hline
          \hfill{}Test & Français & Journalistique & 100 & 308,5 & Lecteur & 9,6 & 7,6~\% & 1,7\\
          \bottomrule
        \end{tabular}
      }

      \caption{Corpus Wikinews
               \label{tab:wikinews}}
    \end{table}

    \TODO{exemple}

  %-----------------------------------------------------------------------------

    \section[SemEval]{SemEval~\textnormal{\large\cite{kim2010semeval}}}
  \label{sec:main-data_description-semeval_data}
    À l'instar de \textsc{Deft}, SemEval est une campagne d'évaluation
    internationale. Le corpus éponyme dont nous disposons est la collection de
    documents construite pour la tâche 5 de l'édition 2010 de SemEval, tâche
    consacrée à l'extraction de termes-clés à partir d'articles scientifiques.
    Le corpus SemEval est constitué de 244 articles scientifiques en anglais
    issus de la bibliothèque numérique de l'\textsc{Acm} (\textit{Association
    for Computing Machinery}). Cette bibliothèque regroupe les articles d'une
    multitude de disciplines de l'informatique. Les documents du corpus SemEval
    concernent les catégories C2.4 (\textit{Distributed Systems} --- Systèmes
    distribués), H3.3 (\textit{Information Search and Retrieval} --- Recherche
    d'information), I2.11 (\textit{Distributed Artificial Intelligence --
    Multiagent Systems} --- Intelligence artificielle distribué -- Systèmes
    multi-agents) et J4 (\textit{Social and Behavioral Sciences -- Economics}
    --- Sciences sociales et comportementales -- Économie) de la classification
    \textsc{Acm} de 1998.
    
    Le tableau~\ref{tab:semeval} présente les caractéristiques de SemEval. La
    collection est répartie en deux sous-ensembles, un ensemble de 144 documents
    d'apprentissage et un ensemble de 100 documents de test. \TODO{revenir sur
    les données du tableau + liens avec les facteurs énoncés précédemments}
    \TODO{répartition équitable}

    \begin{table}[!h]
      \centering
      \resizebox{\linewidth}{!}{
        \begin{tabular}{l|c@{~~}c@{~~}c@{~~}c|c@{~~}c@{~~}c@{~~}c}
          \toprule
          \multirow{2}{*}{\textbf{Corpus}} & \multicolumn{4}{c|}{\textbf{Documents}} & \multicolumn{4}{c}{\textbf{Termes-clés}}\\
          \cline{2-9}
          & Langue & Nature & Quantité & Mots moy. & Annotateur & Quantité moy. & \og{}À assigner\fg{} & Mots moy.\\
          \hline
          \hfill{}Appr. & Anglais & Scientifique & 144 & 5~134,6 & Auteur~/~Lecteur & 15,4 & 13,5~\% & 2,1\\
          \hfill{}Test & \ditto & \ditto & 100 & 5~177,7 & \ditto & 14,7 & 22,1~\% & \ditto\\
          \bottomrule
        \end{tabular}
      }

      \caption{Corpus SemEval
               \label{tab:semeval}}
    \end{table}

    Le tableau~\ref{tab:semeval_annotators} montre la quantité de termes-clés
    annotés par les auteurs, des lecteurs ou les deux. Ces chiffres, donnés par
    \newcite{kim2010semeval} montrent qu'il y a effectivement une différence de
    stratégie entre les auteurs et des lecteurs. Les auteurs donnent peu de
    termes-clés (nous le voyons aussi avec \textsc{Deft}) comparés aux lecteurs
    et l'intersection des deux annotations ne couvre qu'un tier, seulement, des
    termes-clés des auteurs. \TODO{approfondir le lien avec \textsc{Deft}}

    \begin{table}[!h]
      \centering
      \begin{tabular}{l|ccc}
        \toprule
        \multirow{2}{*}{\textbf{Corpus}} & \multicolumn{3}{c}{\textbf{Annotateur}}\\
        \cline{2-4}
        & Auteur & Lecteur & Combinaison\\
        \hline
        \hfill{}Appr. & 559 & 1824 & 2223\\
        \hfill{}Test & 387 & 1217 & 1482\\
        \bottomrule
      \end{tabular}

      \caption{Nombre de termes-clés annotés dans SemEval, en fonction des
               annotateurs
               \label{tab:semeval_annotators}}
    \end{table}

  %-----------------------------------------------------------------------------

  \section[\textsc{Duc}]{\textsc{Duc}~\textnormal{\large\cite{wan2008expandrank}}}
  \label{sec:main-data_description-duc_data}
    \textsc{Duc} est une campagne d'évaluation internationale axée sur le résumé
    automatique. Notre collection de documents \textsc{Duc} est issue du corpus
    construit dans le cadre de l'édition 2001 de la campagne~\cite{over2001duc}.
    Dans leurs travaux en extraction automatique de termes-clés,
    \newcite{wan2008expandrank} annotent en termes-clés les 308 documents de
    tests du corpus de la campagne. Il s'agit de 308 articles journalistiques
    publiés par six média d'information différents~: \textit{Associated Press
    Newswire}, \textit{Foreign Broadcast Information Service}, \textit{Financial
    Times}, \textit{Los Angeles Times}, \textit{San Jose Mercury News} et
    \textit{Wall Street Journal}. Ceux-ci couvrent 30 sujets d'actualités
    (tornades, contrôle des armes à feu, etc.).
    
    Le tableau~\ref{tab:duc} donne les détails du corpus \textsc{Duc}.
    \TODO{revenir sur les données du tableau + liens avec les facteurs énoncés
    précédemments}

    \begin{table}[!h]
      \centering
      \resizebox{\linewidth}{!}{
        \begin{tabular}{l|c@{~~}c@{~~}c@{~~}c|c@{~~}c@{~~}c@{~~}c}
          \toprule
          \multirow{2}{*}{\textbf{Corpus}} & \multicolumn{4}{c|}{\textbf{Documents}} & \multicolumn{4}{c}{\textbf{Termes-clés}}\\
          \cline{2-9}
          & Langue & Nature & Quantité & Mots moy. & Annotateur & Quantité moy. & \og{}À assigner\fg{} & Mots moy.\\
          \hline
          \hfill{}Test & Anlgais & Journalistique & 308 & 900,7 & Lecteur & 8,1 & 3,5~\% & 2,1\\
          \bottomrule
        \end{tabular}
      }

      \caption{Corpus \textsc{Duc}
               \label{tab:duc}}
    \end{table}

  %-----------------------------------------------------------------------------

  \section{Conclusion}
  \label{sec:main-data_description-conclusion}
    Dans ce chapitre, nous présentons les ressources que nous utilisons dans nos
    travaux de recherche. Pour analyser nos travaux, les évaluer et les
    comparer à d'autres, nous disposons de cinq collections de données~:
    Termith, \textsc{Deft}, Wikinews, SemEval et \textsc{Duc}. Celles-ci ne
    partagent pas toutes la même langue, des documents de même nature et le même
    domaine. \TODO{Étendre avec les différentes annotations des termes-clés}.
    Cette diversité est importante pour mieux comprendre les points forts et les
    points faibles de nos travaux.

