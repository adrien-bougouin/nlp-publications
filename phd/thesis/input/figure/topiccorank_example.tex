\begin{figure} % 44960
  \centering
  \framebox[\linewidth]{ % archeologie_525-02-11060
    \parbox{.99\linewidth}{\footnotesize
      \textbf{Étude préliminaire de la céramique non tournée micacée du bas
      Langue-}
      ~\hfill\underline{\textit{Archéologie}}\\
      \textbf{doc occidental : typologie, chronologie et aire de diffusion}\\

      L'étude présente une variété de céramique non tournée dont la
      typologie et l'analyse des décors permettent de l'identifier
      facilement. La nature de l'argile enrichie de mica donne un aspect
      pailleté à la pâte sur laquelle le décor effectué selon la méthode du
      brunissoir apparaît en traits brillant sur fond mat. Cette première
      approche se fonde sur deux séries issues de fouilles anciennes menées
      sur les oppidums du Cayla à Mailhac (Aude) et de Mourrel-Ferrat à
      Olonzac (Hérault). La carte de répartition fait état d'échanges ou de
      commerce à l'échelon macrorégional rarement mis en évidence pour de la
      céramique non tournée. S'il est difficile de statuer sur l'origine des
      décors, il semble que la production s'insère dans une ambiance
      celtisante. La chronologie de cette production se situe dans le
      deuxième âge du Fer. La fourchette proposée entre la fin du
      IV$^\text{e}$ et la fin du II$^\text{e}$ s. av. J.-C. reste encore à
      préciser.\\

      \textbf{Termes-clés de référence~:} distribution~; mourrel-ferrat~;
      olonzac~; le cayla~; mailhac~; micassé~; céramique non-tournée~; celtes~;
      production~; echange~; commerce~; cartographie~; habitat~; oppidum~; site
      fortifié~; fouille ancienne~; identification~; décor~; analyse~;
      répartition~; diffusion~; chronologie~; typologie~; céramique~; etude du
      matériel~; hérault~; aude~; france~; europe~; la tène~; age du fer.
    }
  }~\\

  \vspace{1.5em}

  \TODO{}~\\

  \vspace{1em}

  \framebox[\linewidth]{
    \parbox{.99\linewidth}{\footnotesize
      \textbf{Sortie de TopicCoRank~:} \underline{céramique}~;
      \underline{décors}~; \underline{typologie}~; \underline{chronologue}~;
      \underline{production}~; étude préliminaire~; \underline{diffusion}~;
      \underline{analyse}~; \underline{france}~; \underline{répartition}.
    }
  }~\\

  \vspace{1em}

  \framebox[\linewidth]{
    \parbox{.99\linewidth}{\footnotesize
      \textbf{Sortie de TopicRank~:} \underline{décors}~;
      \underline{céramique}~; \underline{chronologie}~; \underline{typologie}~;
      \underline{production}~; fin~; étude préliminaire~; fer~; deuxième âge~;
      aire.
    }
  }

  \caption[
    Exemple d'extraction de termes-clés avec TopicCoRank sur \TODO{}
  ]{
    Exemple d'extraction de termes-clés avec TopicCoRank sur \TODO{}.
    Les termes-clés soulignés sont les termes-clés correctement extraits.
    \label{fig:exemple_topicrank}
  }
\end{figure}

