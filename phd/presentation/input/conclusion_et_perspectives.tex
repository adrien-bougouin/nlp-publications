\section{Conclusion et perspectives}
%  \begin{frame}{Conclusion}
%    Extraction de termes-clés
%    \begin{itemize}
%      \item{Générique}
%      \item{Ordonnancement des sujets du document}
%      \item{Amélioration de l'état de l'art}
%    \end{itemize}
%
%    \vspace{1em}
%
%    Ajout de l'assignement de termes-clés en domaines de spécialité
%    \begin{itemize}
%      \item{Extraction et assignement simultanés}
%      \item{Collaboration entre le document et le domaine}
%      \item{100~\% de couverture théorique}
%    \end{itemize}
%  \end{frame}

  \begin{frame}{Conclusion}
    Indexation automatique par termes-clés
    \begin{enumerate}
      \item{Dans le contexte général}
      \begin{itemize}
        \item{Extraction de termes-clés à base de graphe}
        \item{Ordonnancement par importance des sujets du document}
        \item{Amélioration de l'existant}
      \end{itemize}
      \item{En domaines de spécialité}
      \begin{itemize}
        \item{Extraction et assignement simultanés}
        \item{Usage de la connaissance du domaine}
        \item{100~\% de couverture théorique}
      \end{itemize}
    \end{enumerate}

    \vspace{1em}

    \begin{alertblock}{Limites}
      \begin{itemize}
        \item{TopicCoRank focalisé sur les domaines de spécialité}
        \item{Évaluation du point de vue des termes-clés uniquement}
      \end{itemize}
    \end{alertblock}    
  \end{frame}

%  \begin{frame}{Perspectives}
%    \begin{itemize}
%      \item{Varier l'influence de chaque graphe dans TopicCoRank}
%      \begin{itemize}
%        \item{Quelle influence du document sur le domaine~?}
%        \item{Quelle influence du domaine sur le document~?}
%        \item{Faut-il tenir compte de la couverture du domaine par les données
%              d'apprentissage~?}
%      \end{itemize}
%      \item{TopicCoRank Vs Classification multi-étiquette multi-classe}
%      \item{Mesurer l'impact de TopicRank et TopicCoRank pour la RI}
%      \begin{itemize}
%        \item{Évaluation intrinsèque Vs évaluation extrinsèque}
%      \end{itemize}
%      \item{Autres usages pour TopicRank}
%      \begin{itemize}
%        \item{Faciliter la lecture pour les
%              dyslexiques~\cite{rello2014dislexia}}
%        \item{Faciliter l'apprentissage des langues
%              secondes~\cite{pressley1982mnemonickeywordmethod}}
%      \end{itemize}
%    \end{itemize}
%  \end{frame}

  \begin{frame}{Perspectives}
    TopicCoRank dans le contexte général~?
    \begin{itemize}
      \item{Parallèle entre genre et domaine~?}
      \item{Quid de l'homogénéité~?}
      \item{Quelle influence de chaque graphe sur l'autre~?}
    \end{itemize}
  \end{frame}

  \begin{frame}{Perspectives}
    Apports de TopicRank~:
    \begin{itemize}
      \item{Recherche d'information~\cite{jones1999phrasier}}
      \item{Lecture pour les dislexiques~\cite{rello2014dislexia}}
      \item{Apprentissage des langues secondes~\cite{pressley1982mnemonickeywordmethod}}
    \end{itemize}

    \vspace{1em}

    Apports de TopicCoRank~:
    \begin{itemize}
      \item{Recherche d'information~\cite{jones1999phrasier}}
      \item{Veille terminologique}
      \begin{itemize}
        \item{Faut-il sélectionner les candidats avec un extracteur
              terminologique~?}
      \end{itemize}
    \end{itemize}
  \end{frame}

