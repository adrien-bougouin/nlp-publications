%%%%%%%%%%%%%%%%%%%%%%%%%%%%%%%%%%%%%%%%%%%%%%%%%%%%%%%%%%%%%%%%%%%%%%%%%%%%%%%%
% File acl2015.tex
%
% Contact: car@ir.hit.edu.cn, gdzhou@suda.edu.cn
%
% Based on the style files for ACL-2014, which were, in turn,
% Based on the style files for ACL-2013, which were, in turn,
% Based on the style files for ACL-2012, which were, in turn,
% based on the style files for ACL-2011, which were, in turn, 
% based on the style files for ACL-2010, which were, in turn, 
% based on the style files for ACL-IJCNLP-2009, which were, in turn,
% based on the style files for EACL-2009 and IJCNLP-2008...

% Based on the style files for EACL 2006 by 
% e.agirre@ehu.es or Sergi.Balari@uab.es
% and that of ACL 08 by Joakim Nivre and Noah Smith

\documentclass[11pt]{article}

\usepackage{acl2015}
\usepackage[utf8]{inputenc}
\usepackage{times}
\usepackage{url}
\usepackage{latexsym}
\usepackage{color}
\usepackage{tikz}
\usepackage{pgfplots}
\usepackage{lipsum}
\usepackage{units}
\usepackage{amsmath}
\usepackage{relsize}
\usepackage{booktabs}
\usepackage{multirow}
\usepackage{subfigure}
\usepackage[inline]{enumitem}
\usepackage{varwidth}

%\setlength\titlebox{5cm}

\usetikzlibrary{arrows, backgrounds, decorations.markings, positioning, shapes}

\newcommand\TODO[1]{\textcolor{red}{[TODO #1]}}
\newcommand\FILL[1]{\textcolor{red}{\lipsum[#1]}}

%% tikz pictures %%%%%%%%%%%%%%%%%%%%%%%%%%%%%%%%%%%%%%%%%%%%%%%%%%%%%%%%%%%%%%%
\tikzstyle{arrow}=[
  decoration={markings, mark=at position 1 with {\arrow[scale=2.5]{>}}},
  postaction={decorate}
]

%-- database -------------------------------------------------------------------
% http://tex.stackexchange.com/questions/123854/display-database-instance-relationship-with-tikz
\tikzstyle{database}=[
  draw,
  color=black,
  align=center,
  minimum width=5.5em,
  minimum height=6.5em,
  cylinder,
  cylinder uses custom fill,
  cylinder body fill=White,
  cylinder end fill=White,
  shape border rotate=90,
  aspect=0.5
]
\newcommand\drawdatabase[1]{
  \begin{tikzpicture}
    \node[database] (db) {#1};
  \end{tikzpicture}
}

%-- document -------------------------------------------------------------------
% http://tex.stackexchange.com/questions/103688/folded-paper-shape-tikz
\makeatletter
\pgfdeclareshape{doc}{
  \inheritsavedanchors[from=rectangle] % this is nearly a rectangle
  \inheritanchorborder[from=rectangle]
  \inheritanchor[from=rectangle]{center}
  \inheritanchor[from=rectangle]{north}
  \inheritanchor[from=rectangle]{south}
  \inheritanchor[from=rectangle]{west}
  \inheritanchor[from=rectangle]{east}
  % ... and possibly more
  \backgroundpath{% this is new
    % store lower right in xa/ya and upper right in xb/yb
    \southwest \pgf@xa=\pgf@x \pgf@ya=\pgf@y
    \northeast \pgf@xb=\pgf@x \pgf@yb=\pgf@y
    % compute corner of ‘‘flipped page’’
    \pgf@xc=\pgf@xb \advance\pgf@xc by-10pt % this should be a parameter
    \pgf@yc=\pgf@yb \advance\pgf@yc by-10pt
    % construct main path
    \pgfpathmoveto{\pgfpoint{\pgf@xa}{\pgf@ya}}
    \pgfpathlineto{\pgfpoint{\pgf@xa}{\pgf@yb}}
    \pgfpathlineto{\pgfpoint{\pgf@xc}{\pgf@yb}}
    \pgfpathlineto{\pgfpoint{\pgf@xb}{\pgf@yc}}
    \pgfpathlineto{\pgfpoint{\pgf@xb}{\pgf@ya}}
    \pgfpathclose
    % add little corner
    \pgfpathmoveto{\pgfpoint{\pgf@xc}{\pgf@yb}}
    \pgfpathlineto{\pgfpoint{\pgf@xc}{\pgf@yc}}
    \pgfpathlineto{\pgfpoint{\pgf@xb}{\pgf@yc}}
    \pgfpathlineto{\pgfpoint{\pgf@xc}{\pgf@yc}}
  }
}
\makeatother
\tikzstyle{document}=[
  draw,
  align=center,
  color=black,
  fill=white,
  minimum width=5.5em,
  minimum height=6.5em,
  shape=doc,
  inner sep=2ex
]
\newcommand\drawdocument[2]{
  \begin{tikzpicture}
    \node[document, scale=#2] (doc) {#1};
  \end{tikzpicture}
}

%-- corpus ---------------------------------------------------------------------
\newcommand\drawcorpus[2]{
  \begin{tikzpicture}
    \node[document, scale=#2] (background) {#1};
    \node[document, scale=#2, anchor=west] at ([xshift=.25em, yshift=-.25em]background.west) (middle) {#1};
    \node[document, scale=#2, anchor=west] at ([xshift=.25em, yshift=-.25em]middle.west) (foreground) {#1};
  \end{tikzpicture}
}

%-- component ------------------------------------------------------------------
\tikzstyle{component}=[
  draw,
  fill=white,
  align=center,
  rectangle,
  minimum width=12em,
  minimum height=3em,
  transform shape
]
\newcommand\drawcomponent[2]{
  \begin{tikzpicture}
    \node[component, scale=#2] (cmp) {#1};
  \end{tikzpicture}
}
%%%%%%%%%%%%%%%%%%%%%%%%%%%%%%%%%%%%%%%%%%%%%%%%%%%%%%%%%%%%%%%%%%%%%%%%%%%%%%%%

\title{Keyphrase Annotation with Graph Co-Ranking}

\author{
    Adrien Bougouin \and Florian Boudin \and Béatrice Daille\\
    Université de Nantes, LINA, France\\
    \normalsize\texttt{\{adrien.bougouin,florian.boudin,beatrice.daille\}@univ-nantes.fr}
}

\date{}

\begin{document}
  \maketitle

  \begin{abstract}
  Keyphrases are words or phrases that synthesize the content of a document.
  In previous work, automatic keyphrase annotation is either carried out 
  by extracting the most important phrases from a document (keyphrase 
  extraction), or by assigning entries from a controlled vocabulary 
  (keyphrase assignment).
  On the one hand, assignment methods provide better-formed keyphrases, 
  as well as keyphrases that do not occur in the document.
  On the other hand, extraction methods do not depend on manually built 
  resources and are able to provide new (not already assigned) keyphrases.
  This paper proposes a co-ranking approach that performs keyphrase 
  extraction and assignment simultaneously in a mutually reinforcing manner.
  Experimental results show the effectiveness of the proposed approach 
  for the two tasks, both individually and combined.
  
  

%    Keyphrases are words or phrases that %represent the most important
%    reflects the content of a document. 
    %soit phrase soit multi-word expressions mais pas mulit-word
%    Keyphrase extraction deals with the selection of keyphrases that occur 
%    in the document. %consists in annotating a document with the important concepts verbatim in the document. 
%    Keyphrase assignment deals with the assignation of keyphrases that are term entries of a thesausus to the document.  These keyphrases may not occur in the document as such.
    %consists in annotating a document with terminological entries,
    %not necessarily verbatim in the document, that describe the important
    %concepts in the document. 
    %The former approach 
%    Keyphrase extraction is able to discover new terms that are not yet in a reference or a thesaurus, whereas keyphrase assignment provides better-formed keyphrases. 
%    This paper presents the first method, TopicCoRank,  %to simultaneously j'aime pas ce mot
%    that performs keyphrase extraction and keyphrase assignment in parallel. 
    %ou the first integrated method that performs both keyphrase extraction and keyphrase assignment
    %Our method 
%    TopicCoRank uses a co-occurrence graph 
    %of the document and its domain, and
%    built from the document and from prior terminological knowledge 
    %unified them 
%    to rank document and terminological keyphrase candidates.
    % tu peux choisir entre domain et terminological (terminological inclut la notion de domaine)
    % Une phrase avec les résutats genre We demonstrate on several datasets and languages that TopiCoRank outperforms keyphrase extraction and keyphrase assignment.
    
  \end{abstract}

  \section{Introduction}
\label{sec:section}
    Keyphrases are words or phrases that represent the main content of a document.
    Similar to an abstract, keyphrases give a synoptic picture of what is important in the document.
    Disimilar to an abstract, keyphrases are small grain units and are useful resources for many Natural Language Processing tasks: document clustering~\cite{han2007webdocumentclustering}, information retrieval~\cite{medelyan2008smalltrainingset}, document summarization~\cite{litvak2008graphbased}, etc.
    However, documents do not always contain keyphrases.
    As the daily flow of new documents grows, manually annotating documents has become impractical.
    Hence automatic keyphrase extraction recently attracts a lot of attention and many different methods are proposed~\cite{hasan2014state_of_the_art}.

    Automatic keyphrase extraction is the task of detecting important words or phrases within a document.
    Generally speaking, we divide keyphrase extraction methods into two categories: supervised and unsupervised.
    Supervised methods treat keyphrase extraction as a binary classification task, e.g.~\cite{witten1999kea}.
    Conversely, unsupervised methods usually rank keyphrase candidates by importance and select the top-ranked ones as keyphrases, e.g.~\cite{mihalcea2004textrank}.

    Although they tackle the keyphrase extraction problem differently, both supervised and unsupervised methods rely on a candidate selection step.
    Keyphrase candidate selection identifies words or phrases consistent with human-assigned keyphrase properties.
    %Although keyphrase candidate selection starts to draw attention~\cite{wang2014keyphraseextractionpreprocessing}, keyphrase extraction methods use simple heuristics: selection of n-grams, sequences of nouns and adjectives, etc.
    However, current selection methods use simple heuristics~\cite{wang2014keyphraseextractionpreprocessing}: candidates are n-grams or sequences of nouns and adjectives.
    %This work infers linguistic properties from human-assigned keyphrases and demonstrates their applicability on keyphrase candidate selection.
    This work proposes rules based on a comprehensive analysis of modifiers within human-assigned keyphrases.
    We demonstrate their applicability on keyphrase candidate selection.
    
    This paper is organized as follows.
    Section~\ref{sec:keyphrase_properties} presents an analysis of human-assigned keyphrases.
    Section~\ref{sec:candidate_selection} describes common keyphrase candidate selection methods followed by a description of our method in Section~\ref{sec:proposed_candidate_selection_method}. Finally, Section~\ref{sec:experiments} presents the expriments and Section~\ref{sec:conclusion} concludes our work.

  \section{Related Work}
\label{sec:related_work}
  \TODO{Introduction to the standard pipeline}

  \subsection{Automatic Keyphrase Extraction}
  \label{subsec:ake}
    \begin{itemize}
      \item{TF-IDF, Likey, etc.}
      \item{Graph-based methods (focus on TopicRank)}
      \item{GenEX, KEA, HUMB, etc. (focus on KEA)}
    \end{itemize}

  \subsection{Automatic Keyphrase Assignment}
  \label{subsec:aka}
    \begin{itemize}
      \item{KEA++ (Automatic Keyphrase Indexing)}
      \item{WAM (SMT AKA method)}
    \end{itemize}

  \subsection{Graph co-ranking for NLP}
  \label{subsec:graph_co_ranking_for_nlp}
    \begin{itemize}
      \item{Xiaojun Wan: Cross-language document summarization}
      \item{Rui Yan: Tweet recommendation}
      \item{Kang Liu: Opinion mining (check) from online reviews}
    \end{itemize}


  \section{TopicRank++}
\label{sec:topicrankpp}
  This section presents TopicRank++, a supervised extension of TopicRank that
  adds AKA. Topic\-Rank works in three steps:
  \begin{enumerate}
    \item{Clustering candidate keyphrases that belongs to the same topic.
          \newcite{bougouin2013topicrank} assumes that candidate of the same
          topic must share as many words as possible. They uses a Hierarchical
          Agglomerative Clustering (HAC) with a ``naive'' stem overlap
          similarity: at the beginning, each candidate is a single cluster and
          candidates sharing an average of $\unitfrac{1}{4}$ stemmed words with
          the candidates of a given cluster are added to this cluster.}
    \item{Building a graph of topics and ranking the topics using
          TextRank~\cite{mihalcea2004textrank}. Each topic is connected to the
          other topics by edges weighted according to the semantic strength
          between the connected topics. TextRank ranking algorithm recursively
          gives high importance to topics strongly connected to as most
          important topics as possible.}
    \item{Extracting keyphrases among the candidates of the $N$ most important
          topics. To avoid topic redundancy, TopicRank only extracts one
          keyphrase per topic. Following previous
          observations~\cite{witten1999kea}, \newcite{bougouin2013topicrank}
          extract the first occurring candidate from each topic.}
  \end{enumerate}

  To fit our need, TopicRank++ has a different graph construction. We also
  replace the classic TextRank ranking algorithm by a co-ranking variant and
  we integrate AKA during the former AKE process.

  \subsection{Graph construction}
  \label{subsec:graph_construction}
    TopicRank++ operates over a unified graph that connects two graphs
    representing the domain's reference keyphrases, the document's topics and
    the relations between them. Formally, let $G = (V, E)$ denote the unified
    graph. Reference keyphrases and topics are vertices $V$, respectively $V_k$
    and $V_t$, connected to their equals by edges
    $E_\textnormal{\textit{intra}}$ and connected to the other vertices by edges
    $E_\textnormal{\textit{outer}}$ (see Figure~\ref{fig:topicrankpp_graph}).

    \begin{figure*}
      \newcommand{\xslant}{0.25}
      \newcommand{\yslant}{0}

      \centering
      \begin{tikzpicture}[transform shape, scale=.66]
        % frame
        \node [draw,
               rectangle,
               minimum width=.7\linewidth,
               minimum height=8em,
               xslant=\xslant,
               yslant=\yslant] (domain_graph) {};
        \node [above=of domain_graph,
               xshift=.36\linewidth,
               yshift=8em,
               anchor=south east] (domain_graph_label) {domain keyphrases};

        \node [draw,
               circle,
               above=of domain_graph,
               xshift=.3\linewidth,
             yshift=5em] (domain_node1) {$V_{k_1}$};
        \node [draw,
               circle,
               above=of domain_graph,
               xshift=-.3\linewidth,
               yshift=5em] (domain_node2) {$V_{k_2}$};
        \node [draw,
               circle,
               above=of domain_graph,
               yshift=5em] (domain_node3) {$V_{k_3}$};
        \node [draw,
               circle,
               above=of domain_graph,
               xshift=.15\linewidth,
               yshift=.75em] (domain_node4) {$V_{k_4}$};
        \node [draw,
               circle,
               above=of domain_graph,
               xshift=-.15\linewidth,
               yshift=.75em] (domain_node5) {$V_{k_5}$};

        \draw [<->] (domain_node1) -- (domain_node3);
        \draw [<->] (domain_node2) -- (domain_node3);
        \draw [<->] (domain_node2) -- (domain_node4);
        \draw [<->] (domain_node4) -- (domain_node5);
        \draw [<->] (domain_node4) -- (domain_node3);

        % document
        \node [draw,
               rectangle,
               minimum width=.7\linewidth,
               minimum height=8em,
               xslant=\xslant,
               yslant=\yslant,
               above=of domain_graph,
               xshift=-2em] (document_graph) {};
        \node [below=of document_graph,
               xshift=-.36\linewidth,
               yshift=-8em,
               anchor=north west] (document_graph_label) {document topics};

        \node [draw,
               regular polygon,
               regular polygon sides=8,
               below=of document_graph,
               xshift=.3\linewidth,
               yshift=-5em] (document_node1) {$V_{t_1}$};
        \node [draw,
               regular polygon,
               regular polygon sides=8,
               below=of document_graph,
               xshift=-.3\linewidth,
               yshift=-5em] (document_node2) {$V_{t_2}$};
        \node [draw,
               regular polygon,
               regular polygon sides=8,
               below=of document_graph,
             yshift=-5em] (document_node3) {$V_{t_3}$};
        \node [draw,
               regular polygon,
               regular polygon sides=8,
               below=of document_graph,
               xshift=.15\linewidth,
               yshift=-.75em] (document_node4) {$V_{t_4}$};
        \node [draw,
               regular polygon,
               regular polygon sides=8,
               below=of document_graph,
               xshift=-.1\linewidth,
               yshift=-.75em] (document_node5) {$V_{t_5}$};
        \node [draw,
               regular polygon,
               regular polygon sides=8,
               below=of document_graph,
               yshift=-.75em] (document_node6) {$V_{t_6}$};
        \node [draw,
               regular polygon,
               regular polygon sides=8,
               below=of document_graph,
               xshift=-.175\linewidth,
               yshift=-5em] (document_node7) {$V_{t_7}$};

        \draw [<->] (document_node2) -- (document_node7);
        \draw [<->] (document_node2) -- (document_node5);
        \draw [<->] (document_node7) -- (document_node5);
        \draw [<->] (document_node7) -- (document_node3);
        \draw [<->] (document_node5) -- (document_node6);
        \draw [<->] (document_node3) -- (document_node1);
        \draw [<->] (document_node1) -- (document_node4);
        \draw [<->] (document_node3) -- (document_node4);

        % extra link
        \draw [<->, dashed] (document_node2) -- (domain_node2);
        \draw [<->, dashed] (document_node6) -- (domain_node5);
        \draw [<->, dashed] (document_node6) -- (domain_node3);
        \draw [<->, dashed] (document_node4) -- (domain_node1);
        \draw [<->, dashed] (document_node3) -- (domain_node4);

        % legend
        \node [right=of document_graph, xshift=2em, yshift=-9.25em] (legend_title) {\underline{Legend:}};
        \node [below=of legend_title, xshift=-1em, yshift=2em] (begin_inner) {};
        \node [right=of begin_inner] (end_inner) {: $E_\textnormal{\textit{inner}}$};
        \node [below=of begin_inner, yshift=1.5em] (begin_outer) {};
        \node [right=of begin_outer] (end_outer) {: $E_\textnormal{\textit{outer}}$};

        \draw (legend_title.north  -| end_outer.east) rectangle (end_outer.south -| legend_title.west);

        \draw [<->] (begin_inner) -- (end_inner);
        \draw [<->, dashed] (begin_outer) -- (end_outer);
      \end{tikzpicture}
      \caption{Example of a unified graph constructed by TopicRank++ and its two
               kinds of edges: inner- and outer-graph edges
               \label{fig:topicrankpp_graph}}
    \end{figure*}

    We create edges $E_\textnormal{\textit{intra}}$ between two reference
    keyphrases or two topics when they co-occur, respectively, as reference
    keyphrases for a document of the domain or within a sentence of the
    document. Each edge $E_{\textnormal{\textit{intra}}_{ij}}$ is weighted by
    the normalized number of co-occurrences $w_{ij}$ between the reference
    keyphrases $V_{k_i}$ and $V_{k_j}$ or the topics $V_{t_i}$ and $V_{t_j}$.

    To unify the two graphs, we look for reference keyphrase occurrences within
    the document. The domain keyphrases can be seen as a category map of the
    domain. We try to connect the document to its potential categories. An edge
    $E_{\textnormal{\textit{outer}}_{ij}}$ is created to connect a reference
    keyphrase $V_{k_i}$ and a topic $V_{t_j}$ if the reference keyphrase is a
    member of the topic, i.e.~a keyphrase candidate that belongs to the topic.
    To accept flexions, such as plural flexions, we perform the comparison with
    stems.

  \subsection{Graph-based co-ranking}
  \label{subsec:graph_based_co_ranking}
    This ranking leverages both the reference keyphrase relations and the topic
    relations. Based on TextRank's random walk~\cite{mihalcea2004textrank}, our
    co-ranking algorithm combines the recommendations within each graph in such
    a way that the random part is replaced by a recommendation comming from the
    other graph:
    \begin{tiny}
      \begin{align}
        S(V_{t_i}) &= (1 - \lambda) \sum_{E_{\text{outer}_{ji}}}{\frac{S(V_{k_j})}{\mathlarger\sum_{E_{\text{outer}_{jm}}}{1}}} + \lambda \sum_{E_{\text{inner}_{ij}}}{\frac{w_{ij} S(V_{t_j})}{\mathlarger\sum_{E_{\text{inner}_{jm}}}{{w_{jm}}}}}\\
        S(V_{k_i}) &= (1 - \lambda) \sum_{E_{\text{outer}_{ij}}}{\frac{S(V_{t_j})}{\mathlarger\sum_{E_{\text{outer}_{mj}}}{1}}} + \lambda \sum_{E_{\text{inner}_{ij}}}{\frac{w_{ij} S(V_{k_j})}{\mathlarger\sum_{E_{\text{inner}_{jm}}}{{w_{jm}}}}}
      \end{align}
    \end{tiny}
    These formula embed two assumptions:
    \begin{enumerate}
      \item{A reference keyphrase is important if it is strongly linked to other
            reference keyphrases and a topic is important if it is strongly
            linked to other topics.}
      \item{A reference keyphrase is more important in the context of the
            document if it is related to the topics within the document and a
            topic is more important if it relates to domain keyphrases.}
    \end{enumerate}
    The first assumption is the former TextRank assumption. The second is
    induced by the unification of both domain keyphrases and document topics.

    The $\lambda$ factor helps to configure the influence of the document topics
    over the whole ranking (in percentage). A higher $\lambda$ gives a higher
    influence to the document topics than a lower $\lambda$. $\lambda=0.5$
    balances the influence of the ranking of the document and of the domain.

  \subsection{Keyphrase assignment and extraction}
  \label{subsec:keyphrase_assignment_and_extraction}
    To both assign and extract keyphrases, we first sort the reference
    keyphrases and the document's topics using their importance score obtained
    with the co-ranking, then we retain the top $N$ ones.

    We assign reference keyphrases over one condition. A reference keyphrase can
    be assigned to a document if it is directly or transitively connected to a
    topic of the document. In other words, if the ranking of a reference
    keyphrase has not been influenced by the document, we do not consider this
    reference keyphrase.

    We extract keyphrases from the topics using the former TopicRank strategy.
    Only one keyphrase is extracted per topic. The extracted keyphrase is the
    candidate of the topic that occurs first within the document.

    This AKA and AKE can be further improved. One can set the ratio of
    keyphrases to assign and keyphrases to extract, or even control the degree
    of generalization of the assigned keyphrases. One may be interested in
    assigning very specific keyphrases regarding the document(\TODO{example}),
    whereas one may want to assign more general keyphrases (\TODO{example}). The
    minimum deph between a reference keyphrase and a document topic encodes this
    degree of generalization. A reference keyphrase directly connected to a
    document topic has the lowest generalization regarding the document,
    followed by its connected reference keyphrases, and so on.


  \section{Experimental Settings}
\label{sec:experimental_settings}
  \subsection{Datasets}
  \label{subsec:datasets}
    We run experiments over three datasets: Inspec, SemEval and Inist (ling.),
    which differ in terms of language, nature and size. The following is the
    description of each dataset.

    \paragraph{Inspec~\textnormal{\cite{hulth2003keywordextraction}}} contains
    2000 English abstracts of journal papers collected from the Inspec database.
    The abstracts cover two fields of Computer Science: Computers and Control;
    Information Technology. Inspec is divided into three disjoint sets: a trial
    set containing 500 abstract, a training set containing 1000 abstracts and a
    test set containing 500 abstracts. \TODO{explain keyphrases set and chose
    which one to use}

    \paragraph{SemEval~\textnormal{\cite{kim2010semeval}}} contains 244 English
    scientific papers collected from the ACM Digital Libraries (conference and
    workshop papers). The papers represent four areas of Computer Science:
    Distributed Systems; Information Search and Retrieval; Distributed
    Artificial Intelligence -- Multiagent Systems; Social and Behavioral
    Sciences -- Economics. SemEval is divided into two disjoint sets: a training
    set containing 144 documents and a test set containing 100 documents. The
    associated keyphrases are provided by both authors and readers.

    \paragraph{Inist (ling.)} \TODO{uncertain yet}

    \paragraph{}
    Table~\ref{tab:corpus_statistics} shows factual information about each
    datasets. By looking at these, one can understand the importance of methods
    like ours. Indeed, ranging from \TODO{0.0\%} to \TODO{100\%}, the percentage
    of keyphrases impossible to extract from the documents is part of the
    keyphrases our method is able to find.

  \subsection{Preprocessing}
  \label{subsec:preprocessing}
    We apply the following preprocessing steps to every document of the
    datasets: sentence segmentation, word tokenization and Part-of-Speech (POS)
    tagging. We perform sentence segmentation with the PunktSentenceTokenizer
    provided by the Python Natural Language ToolKit~\cite[NLTK]{bird2009nltk}.
    We tokenize the sentence into words using the NLTK TreebankWordTokenizer for
    English and the Bonsai word tokenizer\footnote{The Bonsai word tokenizer is
    a tool provided with the Bonsai PCFG-LA parser:
    \url{http://alpage.inria.fr/statgram/frdep/fr_stat_dep_parsing.html}.} for
    French. Finally, we use the Stanford POS
    tagger~\cite{toutanova2003stanfordpostagger} for English POS tagging and
    MElt~\cite{denis2009melt} for French POS tagging.

  \subsection{Baselines}
  \label{subsec:baselines}
    We compare TopicRank++ with two baselines: TopicRank, to measure the
    improvement induced by the co-ranking, and KEA++, a state-of-the-art AKA
    method.

    We also evaluate TopicRank++ when only domain keyphrases are extracted
    (TopicRank++$_\textnormal{\emph{dom.}}$), when only document keyphrases are
    extracted (TopicRank++$_\textnormal{\emph{doc.}}$) and when the co-ranking
    is performed with candidate keyphrases instead of topics
    TopicRank++$_\textnormal{\emph{cdt}}$).

  \subsection{Evaluation measures}
  \label{subsec:evaluation_measures}
    To evaluate the performance of the keyphrase extraction methods, we use the
    common measures of precision (P), recall (R) and f-score (F). We cut-off the
    extracted keyphrases at the 10 best ranking ones.


  \section{Evaluation}
  \begin{frame}{Evaluation}
    \framesubtitle{Datasets}

    \begin{itemize}
      \item{Inspec contains 500 abstracts of journal papers ({\small\textsc{En}})}
      \begin{itemize}
        \setbeamertemplate{itemize items}[triangle]
        \item{9.8 keyphrases/document}
        \item{21.8\% of missing keyphrases}
      \end{itemize}
    \item{SemEval contains 100 scientific papers ({\small\textsc{En}})}
      \begin{itemize}
        \setbeamertemplate{itemize items}[triangle]
        \item{14.7 keyphrases/document}
        \item{19.3\% of missing keyphrases}
      \end{itemize}
      \item{WikiNews contains 100 news articles ({\small\textsc{Fr}})}
      \begin{itemize}
        \setbeamertemplate{itemize items}[triangle]
        \item{9.6 keyphrases/document}
        \item{4.4\% of missing keyphrases}
      \end{itemize}
      \item{DEFT contains 93 scientific papers ({\small\textsc{Fr}})}
      \begin{itemize}
        \setbeamertemplate{itemize items}[triangle]
        \item{5.2 keyphrases/document}
        \item{18.2\% of missing keyphrases}
      \end{itemize}
    \end{itemize}
  \end{frame}

  \begin{frame}{Evaluation}
    \framesubtitle{Evaluation measures}

    \begin{itemize}
      \item{Cut-off at 10 keyphrases}
      \item{F-score $\Rightarrow$ compromise between precision and recall}
    \end{itemize}

    \begin{center}
      $\text{f-score} = (1 + \beta^2) \times \frac{\text{precision} \times \text{recall}}{(\beta^2 \times precision) + recall}$,
      $\beta = 1$
    \end{center}
  \end{frame}

  \begin{frame}{Evaluation}
    \framesubtitle{Baselines}

    \begin{itemize}
      \item{TF-IDF weighting}
      \begin{itemize}
        \setbeamertemplate{itemize items}[triangle]
        \item{Best keyphrases contain words with hight TF-IDF}
      \end{itemize}
      \item{TextRank}
      \begin{itemize}
        \setbeamertemplate{itemize items}[triangle]
        \item{Word co-occurrence graph with a window of 2}
        \item{PageRank scoring/ranking}
        \item{Keyphrase generation based on keywords}
      \end{itemize}
      \item{SingleRank}
      \begin{itemize}
        \setbeamertemplate{itemize items}[triangle]
        \item{Word co-occurrence graph with a window of 10}
        \item{PageRank scoring/ranking}
        \item{Candidate keyphrases scored/ranked by the sum of their words' PageRank score}
      \end{itemize}
    \end{itemize}

    Redundant keyphrases are removed from the output.
  \end{frame}

  \begin{frame}{Evaluation}
    \framesubtitle{Main results}
    
    \begin{center}
      \begin{tabular}{rcccc}
        \toprule
        Methods & Inspec & SemEval & WikiNews & DEFT\\
        \midrule
        TF-IDF & 33.4 & 10.5 & 34.3 & 13.2\\
        TextRank & 12.7 & $~~$5.6 & $~~$8.6 & $~~$5.7\\
        SingleRank & \cellcolor{pink}{35.2} & $~~$3.7 & 19.7 & $~~$5.9\\
        TopicRank & 27.9 & \cellcolor{pink}{12.1} & \cellcolor{pink}{35.6} & \cellcolor{pink}{15.1}\\
        \bottomrule
      \end{tabular}
    \end{center}
  \end{frame}

  \begin{frame}{Evaluation}
    \framesubtitle{Contributions evaluation}
    
    \begin{center}
      \begin{tabular}{rcccc}
        \toprule
        Methods & Inspec & SemEval & WikiNews & DEFT\\
        \midrule
        SingleRank & 35.2 & $~~$3.7 & 19.7 & $~~$5.9\\
        +phrases & 22.1 & $~~$8.0 & 28.9 & 13.5\\
        +topics & 26.8 & 11.9 & 31.4 & 14.8\\
        +complete &  \cellcolor{pink}{35.5} & $~~$4.4 & 20.3 & $~~$5.8\\
        TopicRank & 27.9 & \cellcolor{pink}{12.1} & \cellcolor{pink}{35.6} & \cellcolor{pink}{15.1}\\
        \bottomrule
      \end{tabular}
    \end{center}
  \end{frame}

  \begin{frame}{Evaluation}
    \framesubtitle{Keyphrase selection evaluation}
    
    \begin{center}
      \begin{tabular}{rcccc}
        \toprule
        Keyphrase selection & Inspec & SemEval & WikiNews & DEFT\\
        \midrule
        \rowcolor{cyan!33} First position & 27.9 & 12.1 & 35.6 & 15.1\\
        Frequency & 26.8 & $~~$1.4 & 26.2 & $~~$2.5\\
        Centroid &  24.7 & $~~$1.5 & 28.5 & $~~$3.4\\
        \rowcolor{pink} Upper bound & 35.6 & 30.3 & 42.9 & 19.3\\
        \bottomrule
      \end{tabular}
    \end{center}
  \end{frame}


  \section{Conclusion et perspectives}
\label{sec:conclusion_et_perspectives}
  Dans ce travail, nous proposons une méthode à base de graphe pour l'extraction
  non supervisée de termes-clés. Cette méthode groupe les termes-clés candidats
  en sujets, détermine quels sont ceux les plus importants, puis extrait le
  terme-clé candidat qui représente le mieux chacun des sujets les plus
  importants. Cette nouvelle méthode offre plusieurs avantages vis-à-vis des
  précédentes à base de graphe. Le groupement des termes-clés potentiels en
  sujets distincts permet de rassembler des indices utiles auparavant éparpillés
  et le choix d'un seul terme-clé pour représenter un sujet important permet
  d'extraire un ensemble de termes-clés non redondants ( pour $k$ termes-clés
  extraits, exactement $k$ sujets sont couverts). Enfin, le graphe est complet
  et ne requiert plus le paramétrage d'une fenêtre de cooccurrences,
  contrairement aux autres méthodes à base de graphe.

  Les bons résultats de notre méthode montrent la pertinence d'un groupement en
  sujets des candidats pour ensuite les ordonner. Les expériences
  supplémentaires montrent aussi qu'il est encore possible d'améliorer notre
  méthode en proposant une nouvelle stratégie de sélection du terme-clé candidat
  le plus représentatif d'un sujet (pour un gain maximum allant de 4,2 à 15
  points de f-score).

  Nous avons aussi effectué une analyse d'erreurs à partir de laquelle trois
  perspectives de travaux futurs émergent~:

  Nous avons pour objectif d'améliorer la sélection des termes-clés candidats.
  Aussi, des méthodes empruntées à d'autres domaines du TAL peuvent être
  appliquées. Il semble, par exemple, pertinent d'évaluer l'apport des méthodes
  d'extraction terminologiques~\cite{castellvi2001automatictermdetection} pour
  la sélection des termes-clés candidats.
  
  Nous envisageons également d'améliorer le groupement en sujets,
  car celui-ci est très naïf et ne tient compte ni de la synonymie, ni de
  l'ambiguïté des mots. De plus, l'usage du
  radical~\cite{porter1980suffixstripping} des mots n'est pas sans introduire du
  bruit lié à certains faux positifs (p.~ex. \og{}\underline{empir}e\fg{} et
  \og{}\underline{empir}ique\fg{}). L'ajout de connaissances concernant les
  synonymes permettrait de créer des sujets plus complets et une étape de
  désambiguïsation éviterait un groupement systématique des termes-clés
  candidats ayant un ou plusieurs mots en commun. Nous envisageons aussi de
  remplacer la racinisation de \newcite{porter1980suffixstripping} par une
  méthode de lemmatisation. D'un point de vue plus technique, il faudrait
  explorer différentes méthodes de groupement, dont le groupement spectral
  (\textit{spectral clustering}) qui, dans d'autres travaux portant sur
  l'extraction automatique de termes-clés~\cite{liu2009keycluster}, montre de
  meilleures performances que le groupement hiérarchique agglomératif.

  Enfin, une étude détaillée des caractéristiques des termes-clés pourrait
  orienter notre travail vers des critères plus efficaces pour la définition
  d'une stratégie \og{}optimale\fg{} de sélection du terme-clé le plus
  représentatif d'un sujet. Un apprentissage supervisé à partir de certains
  critères est aussi envisagé, au même titre que l'usage de méthodes
  d'optimisation, telles que celle utilisée par
  \newcite{ding2011binaryintegerprogramming} dans leur méthode d'extraction
  automatique de termes-clés.



%  \section*{Acknowledgments}
%    The authors would like to thank the anonymous reviewers for their useful
%    advice and comments. This work was supported by the French National Research
%    Agency (TermITH project -- ANR-12-CORD-0029).

  \bibliographystyle{acl}
  \bibliography{../../biblio}
\end{document}
