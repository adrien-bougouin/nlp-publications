\section{Conclusion and Future Work}
\label{sec:conclusion}
  In this paper, we proposed a co-ranking approach to perform keyphrase
  extraction and keyphrase assignment simultaneously. Our method,
  TopicCoRank, uses keyphrases annotated on training data as a controlled
  vocabulary. %represented by a graph. 
  To extract and assign keyphrases to a document, we build two graphs: one with the
  document's topics and one with the controlled vocabulary. We design a strategy to unify the two graphs and 
  choose the keyphrases with  a co-ranking vote.
  
  %then co-rank topics and controlled keyphrases.
  
  We performed experiments on three datasets of different languages and
  domains. Results showed that our approach benefits from controlled
  keyphrases, improving both keyphrase extraction and keyphrase assignment.

  This work can be used to annotate keyphrases in the way of professional
  indexers. TopicCoRank is able to detect new concepts that could be added
  to terminological data.
  
  TopicCoRank can also be adapted to more specific problems. For instance,
  one may be interested in assigning very specific keyphrases regarding the
  document, whereas one may want to assign more general keyphrases. The
  minimum depth between a controlled keyphrase and a document topic within the unified graph encodes this
  degree of generalization. A controlled keyphrase directly connected to a
  document topic has the lowest generalization regarding the document,
  followed by its connected controlled keyphrases, and so on.
