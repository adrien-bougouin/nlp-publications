\section{TopicRank++}
\label{sec:topicrankpp}
  This section presents TopicRank++, a supervised extension of TopicRank that
  adds AKA. Topic\-Rank works in three steps:
  \begin{enumerate}
    \item{Clustering candidate keyphrases that belongs to the same topic.
          \newcite{bougouin2013topicrank} assumes that candidate of the same
          topic must share as many words as possible. They uses a Hierarchical
          Agglomerative Clustering (HAC) with a ``naive'' stem overlap
          similarity: at the biginning, each candidate is a single cluster and
          candidates sharing an average of $\unitfrac{1}{4}$ stemmed words with
          the candidates of a given cluster are added to this cluster.}
    \item{Building a graph of topics and ranking the topics using
          TextRank~\cite{mihalcea2004textrank}. Each topic is connected to the
          other topics by edges weighted according to the semantic strength
          between the connected topics. TextRank ranking algorithm recursively
          gives high importance to topics strongly connected to as most
          important topic as possible.}
    \item{Extracting keyphrases among the candidates of the $N$ most important
          topics. To avoid topic redundancy, TopicRank only extracts one
          keyphrase per topic. Following previous
          observations~\cite{witten1999kea}, \newcite{bougouin2013topicrank}
          extract the first occurring candidate from each topic.}
  \end{enumerate}

  To fit our need, TopicRank++ has a different graph construction. We also
  replace the classic TextRank ranking algorithm by a co-ranking variant and
  we integrate AKA during the former AKE process.

  \subsection{Graph construction}
  \label{subsec:graph_construction}
    TopicRank++ operates over a unified graph that connects two graphs
    representing the domain's reference keyphrases, the document's topics and
    the relations between them. Formally, let $G = (V, E)$ denote the unified
    graph. Reference keyphrases and topics are vertices $V$, respectively $V_k$
    and $V_t$, connected to their equals by edges
    $E_\textnormal{\textit{intra}}$ and connected to the other vertices by edges
    $E_\textnormal{\textit{outer}}$ (see Figure~\ref{fig:topicrankpp_graph}).

    \begin{figure*}
      \newcommand{\xslant}{0.25}
      \newcommand{\yslant}{0}

      \centering
      \begin{tikzpicture}[transform shape, scale=.66]
        % frame
        \node [draw,
               rectangle,
               minimum width=.7\linewidth,
               minimum height=8em,
               xslant=\xslant,
               yslant=\yslant] (domain_graph) {};
        \node [above=of domain_graph,
               xshift=.36\linewidth,
               yshift=8em,
               anchor=south east] (domain_graph_label) {domain keyphrases};

        \node [draw,
               circle,
               above=of domain_graph,
               xshift=.3\linewidth,
               yshift=5em] (domain_node1) {$k_1$};
        \node [draw,
               circle,
               above=of domain_graph,
               xshift=-.3\linewidth,
               yshift=5em] (domain_node2) {$k_2$};
        \node [draw,
               circle,
               above=of domain_graph,
               yshift=5em] (domain_node3) {$k_3$};
        \node [draw,
               circle,
               above=of domain_graph,
               xshift=.15\linewidth,
               yshift=.75em] (domain_node4) {$k_4$};
        \node [draw,
               circle,
               above=of domain_graph,
               xshift=-.15\linewidth,
               yshift=.75em] (domain_node5) {$k_5$};

        \draw [<->] (domain_node1) -- (domain_node3);
        \draw [<->] (domain_node2) -- (domain_node3);
        \draw [<->] (domain_node2) -- (domain_node4);
        \draw [<->] (domain_node4) -- (domain_node5);
        \draw [<->] (domain_node4) -- (domain_node3);

        % document
        \node [draw,
               rectangle,
               minimum width=.7\linewidth,
               minimum height=8em,
               xslant=\xslant,
               yslant=\yslant,
               above=of domain_graph,
               xshift=-2em] (document_graph) {};
        \node [below=of document_graph,
               xshift=-.36\linewidth,
               yshift=-8em,
               anchor=north west] (document_graph_label) {document topics};

        \node [draw,
               regular polygon,
               regular polygon sides=8,
               below=of document_graph,
               xshift=.3\linewidth,
               yshift=-5em] (document_node1) {$t_1$};
        \node [draw,
               regular polygon,
               regular polygon sides=8,
               below=of document_graph,
               xshift=-.3\linewidth,
               yshift=-5em] (document_node2) {$t_2$};
        \node [draw,
               regular polygon,
               regular polygon sides=8,
               below=of document_graph,
               yshift=-5em] (document_node3) {$t_3$};
        \node [draw,
               regular polygon,
               regular polygon sides=8,
               below=of document_graph,
               xshift=.15\linewidth,
               yshift=-.75em] (document_node4) {$t_4$};
        \node [draw,
               regular polygon,
               regular polygon sides=8,
               below=of document_graph,
               xshift=-.15\linewidth,
               yshift=-.75em] (document_node5) {$t_5$};
        \node [draw,
               regular polygon,
               regular polygon sides=8,
               below=of document_graph,
               yshift=-.75em] (document_node6) {$t_6$};
        \node [draw,
               regular polygon,
               regular polygon sides=8,
               below=of document_graph,
               xshift=-.225\linewidth,
               yshift=-5em] (document_node7) {$t_7$};

        \draw [<->] (document_node2) -- (document_node7);
        \draw [<->] (document_node2) -- (document_node5);
        \draw [<->] (document_node7) -- (document_node5);
        \draw [<->] (document_node7) -- (document_node3);
        \draw [<->] (document_node5) -- (document_node6);
        \draw [<->] (document_node3) -- (document_node1);
        \draw [<->] (document_node1) -- (document_node4);
        \draw [<->] (document_node3) -- (document_node4);

        % extra link
        \draw [<->, dashed] (document_node2) -- (domain_node5);
        \draw [<->, dashed] (document_node6) -- (domain_node5);
        \draw [<->, dashed] (document_node6) -- (domain_node3);
        \draw [<->, dashed] (document_node4) -- (domain_node1);
        \draw [<->, dashed] (document_node3) -- (domain_node4);

        % legend
        \node [right=of document_graph, xshift=2em, yshift=-9.25em] (legend_title) {\underline{Legend:}};
        \node [below=of legend_title, xshift=-1em, yshift=2em] (begin_inner) {};
        \node [right=of begin_inner] (end_inner) {: $E_\textnormal{\textit{inner}}$};
        \node [below=of begin_inner, yshift=1.5em] (begin_outer) {};
        \node [right=of begin_outer] (end_outer) {: $E_\textnormal{\textit{outer}}$};

        \draw (legend_title.north  -| end_outer.east) rectangle (end_outer.south -| legend_title.west);

        \draw [<->] (begin_inner) -- (end_inner);
        \draw [<->, dashed] (begin_outer) -- (end_outer);
      \end{tikzpicture}
      \caption{Example of a unified graph constructed by TopicRank++ and its two
               kinds of edges: inner- and outer-graph edges
               \label{fig:topicrankpp_graph}}
    \end{figure*}

  \subsection{Graph-based co-ranking}
  \label{subsec:graph_based_co_ranking}

  \subsection{Keyphrase assignment and extraction}
  \label{subsec:keyphrase_assignment_and_extraction}

  \subsection{Training protocol}
  \label{subsec:training_protocol}

