\section{Conclusion et perspectives}
\label{sec:conclusion_et_perspectives}
  Dans ce travail, nous proposons une méthode à base de graphe pour
  l'extraction non-supervisée de termes-clés. Cette méthode groupe les
  termes-clés candidats par sujets, détermine quels sont ceux les plus
  importants, puis extrait le terme-clé candidat qui représente le mieux chacun
  des sujets les plus importants. Cette nouvelle méthode offre plusieurs
  avantages vis-à-vis des précédentes méthodes à base de graphe. Dans un premier
  temps, le groupement des termes-clés potentiels en sujets distincts permet le
  rassemblement d'informations utiles qui sont éparpillés avec les autres
  méthodes. Dans un second temps, le choix d'un seul terme-clé pour représenter
  l'un des sujets les plus importants permet d'extraire un ensemble ne contenant
  pas de termes-clés redondants -- pour $k$ termes-clés extraits, exactement $k$
  sujets sont couverts.

  Plusieurs perspectives émergent de ce travail. Tout d'abord, le groupement
  qui est effectué est un peu naïf, car il ne prend pas en compte l'ambiguïté
  sémantique des mots et les relations synonymiques que certain entretiennent.
  Il est donc envisagé d'effectuer un groupement de meilleure qualité, à partir
  de connaissances linguistiques. Enfin, la stratégie de sélection qui consiste
  à extraire, pour chaque sujet, le candidat qui apparait en premier dans le
  document doit être confrontée à d'autres stratégies (sélectionner le candidat
  le plus fréquent, le candidat le plus similaire aux autres candidats du sujet,
  etc.). Il est même envisageable de chercher la solution la plus optimale au
  moyen de technique de Recherche Opérationnelle.

