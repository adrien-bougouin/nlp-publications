\documentclass[10pt, a4paper]{article}
%\documentclass[12pt, a4paper]{article}

\usepackage{lrec2006}
\usepackage{graphicx}
%
\usepackage[utf8]{inputenc}
\usepackage{color}
\usepackage{booktabs}
\usepackage{multirow}
\usepackage{amsmath}
\usepackage{varwidth}
\usepackage{etaremune}
\usepackage{url}

\newcommand\example[1]{
    \begin{center}
        \begin{varwidth}{.9\linewidth}
            \textit{\noindent
                #1
            }
        \end{varwidth}
    \end{center}
}
\newcommand\TODO[1]{\textcolor{red}{[TODO #1]}}

\title{TermITH-Eval: a French Standard-Based Resource\\for Keyphrase Extraction Evaluation}

\name{
    Adrien Bougouin$^1$,
    Sabine Barreaux$^2$,
    Laurent Romary$^3$,
    Florian Boudin$^1$,
    Béatrice Daille$^1$
}

\address{
    $^1$~Université de Nantes, LINA, 2 rue de la Houssinière, 44322 Nantes, France\\
    $^2$~INIST-CNRS, 2, allée du parc de Brabois, 54519 Vandoeuvre-lès-Nancy, France\\
    $^3$~Humboldt-Universität zu Berlin, Dorotheenstraße 24, 10117 Berlin, Germany\\
    E-mail: adrien.bougouin@univ-nantes.fr, sabine.barreaux@inist.fr, laurent.romary@inria.fr,\\
    florian.boudin@univ-nantes.fr, beatrice.daille@univ-nantes.fr
}


\abstract{
    Keyphrase extraction is the task of finding phrases that represent the important content of a document.
    The main aim of keyphrase extraction is to propose textual units that represent the most important topics developed in a document.
    The output keyphrases of automatic keyphrase extraction methods for test documents are typically evaluated by comparing them to manually assigned reference keyphrases.
    Each output keyphrase is considered correct if it matches one of the reference keyphrases.
    However, the choice of the appropriate textual unit (keyphrase) for a topic is sometimes subjective and evaluating by exact matching underestimates the performance.
    This paper presents a dataset of evaluation scores assigned to automatically extracted keyphrases by human evaluators.
    Along with the reference keyphrases, the manual evaluations can be used to validate new evaluation measures.
    Indeed, an evaluation measure that is highly correlated to the manual evaluation is appropriate for the evaluation of automatic keyphrase extraction methods.
    ~\\~\\
    \Keywords{
        TermITH-Eval, structured resource, automatic evaluation, keyphrase extraction.
    }
}

\begin{document}
    \maketitleabstract

    \section{Introduction}
\label{sec:section}
    Keyphrases are words or phrases that represent the main content of a document.
    Similar to an abstract, keyphrases give a synoptic picture of what is important in the document.
    Disimilar to an abstract, keyphrases are small grain units and are useful resources for many Natural Language Processing tasks: document clustering~\cite{han2007webdocumentclustering}, information retrieval~\cite{medelyan2008smalltrainingset}, document summarization~\cite{litvak2008graphbased}, etc.
    However, documents do not always contain keyphrases.
    As the daily flow of new documents grows, manually annotating documents has become impractical.
    Hence automatic keyphrase extraction recently attracts a lot of attention and many different methods are proposed~\cite{hasan2014state_of_the_art}.

    Automatic keyphrase extraction is the task of detecting important words or phrases within a document.
    Generally speaking, we divide keyphrase extraction methods into two categories: supervised and unsupervised.
    Supervised methods treat keyphrase extraction as a binary classification task, e.g.~\cite{witten1999kea}.
    Conversely, unsupervised methods usually rank keyphrase candidates by importance and select the top-ranked ones as keyphrases, e.g.~\cite{mihalcea2004textrank}.

    Although they tackle the keyphrase extraction problem differently, both supervised and unsupervised methods rely on a candidate selection step.
    Keyphrase candidate selection identifies words or phrases consistent with human-assigned keyphrase properties.
    %Although keyphrase candidate selection starts to draw attention~\cite{wang2014keyphraseextractionpreprocessing}, keyphrase extraction methods use simple heuristics: selection of n-grams, sequences of nouns and adjectives, etc.
    However, current selection methods use simple heuristics~\cite{wang2014keyphraseextractionpreprocessing}: candidates are n-grams or sequences of nouns and adjectives.
    %This work infers linguistic properties from human-assigned keyphrases and demonstrates their applicability on keyphrase candidate selection.
    This work proposes rules based on a comprehensive analysis of modifiers within human-assigned keyphrases.
    We demonstrate their applicability on keyphrase candidate selection.
    
    This paper is organized as follows.
    Section~\ref{sec:keyphrase_properties} presents an analysis of human-assigned keyphrases.
    Section~\ref{sec:candidate_selection} describes common keyphrase candidate selection methods followed by a description of our method in Section~\ref{sec:proposed_candidate_selection_method}. Finally, Section~\ref{sec:experiments} presents the expriments and Section~\ref{sec:conclusion} concludes our work.

    \section{TermITH-Eval Dataset}
\label{sec:termith_eval_dataset}
    We selected 400 French bibliographic records from the FRANCIS and PASCAL databases of the French Institute for Scientific and Technical Information (Inist).
    The records cover four specific-domains (100 each): Linguistics, Information Science, Archaeology and Chemistry.
    Every bibliographic record contains a title, an abstract, author keyphrases and reference keyphrases assigned by professional indexers.
    This work only take into account the titles, abstracts and keyphrases assigned by professional indexers (see Figure~\ref{fig:example_bibliographic_record}).
    \begin{figure*}
        \framebox[\linewidth]{ % linguistique_11-0080464_tei.xml
            \parbox{.99\linewidth}{\footnotesize
                \textbf{La cause linguistique}
                \hfill\underline{\textit{Linguistics}}\\

                L'objectif est de fournir une définition de base du concept linguistique de la cause en observant son expression.
                Dans un premier temps, l'A. se demande si un tel concept existe en langue.
                Puis il part des formes de son expression principale et directe (les verbes et les conjonctions de cause) pour caractériser linguistiquement ce qui fonde une telle notion.\\

                \textbf{Reference keyphrases:} Français; interprétation sémantique; conjonction; expression linguistique; concept linguistique; relation syntaxique; cause. 
        }
      }
      \caption{
        Example of the content of a bibliographic record
        \label{fig:example_bibliographic_record}
      }
    \end{figure*}


    The following subsections present the guidelines given to professional indexers for assigning the reference keyphrases, the three keyphrase extraction methods used to automatically extract keyphrases and the guidelines given to professional indexers for the evaluation of automatically extracted keyphrases.

    \subsection{Indexing Guidelines}
    \label{subsec:annotation_guidelines}
        Indexing is the process of describing and identifying a document in terms of its subject content, in order to facilitate the retrieval of information from a collection of documents.
        Professional indexers at the Inist work in their own specialized fields and follow five principles to ensure quality indexing: conformity, exhaustivity, consistency, specificity and impartiality.

        Conformity relies on a domain terminology (indexing language).
        Bibliographic records from a given research area are mainly indexed in accordance with the same indexing language and its usage rules.

        Exhaustivity completes keyphrases obtained when focusing on conformity.
        Professional indexers must identify every keyphrase, for a document, that has potential value for information retrieval.
        Professional indexers also need to include implicit keyphrases if they are useful for the contextualisation of a given keyphrase.

        Consistency increases the quality of document indexing and retrieval.
        If the same concept is important in two bibliographic records of the same domain, then the concept must be represented by the same keyphrase (preferably from the indexing language).

        Specificity relies on the term hierarchy in the domain.
        As a rule, keyphrases must be as specific as possible and more general ones can be added to point their place within the domain (e.g. \textit{Français} -- French -- in Figure~\ref{fig:example_bibliographic_record}).

        Impartiality is a required quality for professional indexers to posses.
        Keyphrases associated to documents must not convey the personal opinion of the indexer regarding the bibliographic record.
        
        To cope with the increasing amount of documents to be referenced, Inist indexers are helped by a pre-indexing system which proposes keyphrases to be validated and enriched.
        The pre-indexing system relies on pattern matching between text and predefined expressions related to potential keyphrases.
        The predefined expressions requires constant updating in order to generate appropriate keyphrases.

    \subsection{Automatic Keyphrase Extraction}
    \label{subsec:automatic_keyphrase_extraction}
        We selected three keyphrase extraction methods to extract 30 keyphrases (10 each) per bibliographic record.
        The methods cover the main techniques used for automatic keyphrase extraction: the statistical method TF-IDF~\cite{salton1975tfidf}, the classification method KEA~\cite{witten1999kea} and the graph-based method TopicRank~\cite{bougouin2013topicrank}.
        
        TF-IDF is a simple and common keyphrase extraction method that ranks the textual units of a document according to their TF-IDF score, frequently used in Information Retrieval.
        The idea is to give a high importance score to textual units which are both frequent in the document and specific to it.
        The specificity of a textual unit regarding a document is obtained using a collection of documents.
        The lower the number of documents in which a textual unit occurs, the more specific this textual unit is.
        
        KEA also relies on simple statistics.
        According to KEA, a keyphrase can be recognized by its importance (TF-IDF) and the position of its first occurrence within the document.
        Indeed, \newcite{witten1999kea} observed that keyphrases tend to appear earlier than later in a document.
        The two properties (TF-IDF and first position) are used as features of a Naive Bayes classifier that labels either the class of ``\textit{keyphrase}'' or ``\textit{non keyphrase}'' to every textual unit of the document.
        
        TopicRank is a graph-based method that ranks topics by importance and extracts one representative keyphrase for each important topic.
        Topics are clusters of textual units which ``contain'' the same concept and the representative keyphrase for each topic is its textual unit that appears first within the document.

        For comparison purposes, we implemented each method and integrated them on top of the same preprocessing tools.
        Every document is first segmented into sentences, sentences are tokenized into words and words are labelled according their morphological class (Part-of-Speech tagging --- POS tagging).
        We performed sentence segmentation with the PunktSentenceTokenizer provided by the Python Natural Language ToolKit (NLTK)\cite{bird2009nltk}, word tokenization using the French tokenizer Bonsai included with the French POS tagger MElt~\cite{denis2009melt}, which we use for POS tagging.

    \subsection{Manual Evaluation Guidelines}
    \label{subsec:manual_evaluation_guidelines}
        Four evaluators took part in the manual evaluation.
        Being chosen for their indexing experience and their expertise in the selected scientific disciplines, evaluators have been asked to follow the guidelines described below.
       
        After reading the title and the abstract of a bibliographic record, evaluators needed to assess if the automatically extracted keyphrases were relevant to the bibliographic record.
        This assessment is made regarding two aspects: appropriateness and silence.

        \subsubsection{Appropriateness}
        \label{subsubsec:appropriateness}
            Appropriateness is a property of an extracted keyphrase.
            Appropriate keyphrases suitably represents the subjects and questions discussed in the document described by the bibliographic record.
            The evaluation of appropriateness is formalized by assigning a score from 2 down to 0, for each extracted keyphrase:
            \begin{etaremune}[start=2]
                \item{
                    The extracted keyphrase is correct, appropriate.
                }
                \item{
                    The extracted keyphrase represents a subject or question discussed in the document but its textual form is not the most appropriate.
                    The extracted keyphrase is a synonym, a spelling variant, a morphosyntactic variant, an acronym, an abbreviation or a phrase with the wrong boundaries.
                    In all these cases, the extracted keyphrase is considered as a variant of a preferred form that is present in the text.
                    This preferred form can be proposed as a keyphrase, with a score of 2, and must be linked as the preferred form of the extracted keyphrase with score 1.
                }
                \item{
                    The extracted keyphrase is inappropriate.
                }
            \end{etaremune}
        
        \subsubsection{Silence}
        \label{subsubsec:silence}
            Silence is the property attached to reference keyphrases.
            A silence means that the information held by a given reference keyphrase is not represented by one or more extracted keyphrases.
            In order to evaluate the silence of the keyphrase extraction method, the evaluators need to check every reference keyphrase and determine whether it complements the assessed method or not.
            The evaluation of silence is formalized by assigning a score from 2 down to 0, for each reference keyphrase:
                \begin{etaremune}[start=2]
                    \item{
                        The reference keyphrase is highly complementary to the keyphrase extraction method.
                        The reference keyphrase contains a very important information missing from the extracted keyphrases.
                    }
                    \item{
                        The reference keyphrase is moderately complementary to the keyphrase extraction method.
                        The reference keyphrase contains a secondary or implicit information missing (or partially missing) from the extracted keyphrases.
                    }
                    \item{
                        The reference keyphrase is not complementary to the keyphrase extraction method.
                        The reference keyphrase has been extracted by the method or cannot be extracted because the notion is absent from the text.
                    }
                \end{etaremune}

    %%%%%%%%%%%%%%%%%%%%%%%%%%%%%%%%%%%%%%%%%%%%%%%%%%%%%%%%%%%%%%%%%%%%%%%%%%%%%%%%

%    ~\\Table~\ref{tab:termith_eval} \TODO{...}
%    \begin{table*}
%        \centering
%        \resizebox{\linewidth}{!}{
%            \begin{tabular}{l|ccccc}
%                \toprule
%                \textbf{Corpus} & \textbf{Documents} & \textbf{Words/Doc.} & \textbf{Ref. keyphrases/Doc.} & \textbf{Words/Ref. keyphrase} & \textbf{Missing ref. keyphrases (\%)} \\
%                \hline
%                Linguistics & 200 & 147.0 & $~~$8.9 & 1.8 & 62.8\% \\
%                Information Science & 200 & 157.0 & 10.2 & 1.7 & 66.9\% \\
%                Archaeology & 200 & 213.9 & 15.6 & 1.3 & 37.4\% \\
%                Chemestry & 200 & 103.9 & 14.6 & 2.4 & 78.8\% \\
%                \bottomrule
%            \end{tabular}
%        }
%        
%        \caption{Statistics of the TermITH-Eval datasets \TODO{update to 100 documents per corpus}
%                 \label{tab:termith_eval}}
%    \end{table*}

    \section{TermITH-Eval Format}
\label{sec:termith_eval_format}
    In the purpose of the TermITH-Eval dataset, we had to tackle the challenge that complex annotations combining automatic extractions, manual annotations, as well as scoring information, would occur within our document.
    Our choice for dealing with such a complex document structure was to use the TEI guidelines, which particularly offer customization facilities for the identification of an optimal trade-off between full compliance to the TEI architecture and integration of project specific constraints.
    More precisely we integrated two extensions to the TEI standard representation:
    \begin{itemize}
        \item{
            We used the work done in~\cite{romary2010tbxgoestei} to complement the TEI guidelines with terminological entries compliant to ISO standard 30042 (TBX, TermBase eXchange).
            This in turn has now become a proposal to the TEI consortium;
        }
        \item{
            We heavily experimented with the new proposal (\url{https://github.com/TEIC/TEI/issues/374}) for an in-document stand-off annotation element, which would allow to class together groups of annotations (e.g. from the same term extraction process).
            We also added TBX entries as possible body objects (in the sense of the Open Annotation framework) to the stand-off proposal.
        }
    \end{itemize}
    
    All in all, this work of compiling the best of existing but also on-going standardisation efforts, has proved to be highly effective for our project, especially when keyphrase extraction outputs had to be sent for evaluation, and we see this as a possible reference framework for similar projects.

    \section{TermITH-Eval Analysis}
\label{sec:termith_eval_analysis}

    %\subsection{Automatic Evaluation Vs. Manual Evaluation}
    %\label{subsec:automatic_evaluation_vs_manual_evaluation}
        Here, we present and analyse the evaluation scores given by human evaluators regarding the three automatic keyphrase extraction methods applied to each specific domain of our dataset.
        To allow comparison with automatic keyphrases, Table~\ref{tab:automatic_evaluation} shows the f1-scores obtained by each method using the standard automatic evaluation approach.
        \begin{table}[h]
            \resizebox{\linewidth}{!}{
                \begin{tabular}{l|cccc}
                    \toprule
                    \textbf{Method} & \textbf{Linguistics} & \textbf{Information Science} & \textbf{Archaeology} & \textbf{Chemistry} \\
                    \hline
                    TF-IDF & 14.0 & \textbf{13.2} & 22.1 & 12.6 \\
                    KEA & \textbf{14.7} & 12.5 & \textbf{23.9} & \textbf{12.8} \\
                    TopicRank & 11.9 & 12.1 & 21.8 & 11.8 \\
                    \bottomrule
                \end{tabular}
            }
            \caption{
                Results of the automatic evaluation of TF-IDF, KEA and TopicRank in term of f1-score on each specific domain
                \label{tab:automatic_evaluation}
            }
        \end{table}
        
        Table~\ref{tab:appropriateness_manual_evaluation} shows the ratios of appropriateness scores per each method per specific domain of our dataset.
        To judge if one method outperforms others, we looked for a highest ratio of keyphrases with score 2, a highest ratio of non redundant keyphrases with score 1, a lowest ratio of redundant keyphrases with score 1 and a lowest ratio of keyphrases with score 0.
        Non redundant and redundant keyphrases with score 1 are distinguished by the \texttt{PreferedForm} given by the evaluator.
        If the extracted keyphrase with a score of 1 has a specified \texttt{PreferedForm}, then it is considered redundant because it is similar to another keyphrase that has also been extracted.
        First, we observe that our guidelines enable a deeper analysis of the methods.
        Indeed, looking at the results of TopicRank proves that it is less redundant than other methods.
        The latter observation is one of the main objectives of the author~\cite{bougouin2013topicrank} of TopicRank.
        However, their evaluation using the standard approach did not show that TopicRank extracts less redundant keyphrases than other methods.
        Secondly, the ordering of the methods from the best performing to the worst performing changed according to whether evaluation was automatic or manual.
        With the automatic evaluation, TopicRank is the method that performs the worst yet it performs better than TF-IDF in every case and better than KEA in half of the cases when analysed with manual evaluation.
        This is due to the fact that automatic evaluation is much more pessimistic than manual evaluation, which deals with subjectivity.
        As a few researchers have stated~\cite{zesch2009rprecision,kim2010rprecision}, the automatic evaluation of keyphrase extraction methods must change to enable it to take subjectivity into account, e.g. by accepting variant forms of reference keyphrases.
        \begin{table*}[t]
            \resizebox{\linewidth}{!}{
                \begin{tabular}{l|ccc|ccc|ccc|ccc}
                    \toprule
                    \multirow{2}{*}{\textbf{Score}} & \multicolumn{3}{c|}{\textbf{Linguistics}} & \multicolumn{3}{c|}{\textbf{Information Science}} & \multicolumn{3}{c|}{\textbf{Archaeology}} & \multicolumn{3}{c}{\textbf{Chemistry}} \\
                    \cline{2-13}
                    & TF-IDF & KEA & TopicRank & TF-IDF & KEA & TopicRank & TF-IDF & KEA & TopicRank & TF-IDF & KEA & TopicRank \\
                    \hline
                    2 & 35.3 & \textbf{37.2} & 37.1 & 34.7 & 34.2 & \textbf{36.3} & 46.0 & 49.9 & \textbf{51.6} & 50.9 & \textbf{54.0} & 53.7 \\
                    1 -- non redundant & $~~$4.2 & \textbf{$~~$9.8} & $~~$5.7 & 15.3 & 18.3 & \textbf{18.5} & 14.1 & \textbf{16.3} & 15.4 & 25.9 & 24.1 & \textbf{29.1} \\
                    1 -- redundant & $~~$6.8 & $~~$8.9 & \textbf{$~~$0.9} & $~~$8.1 & $~~$7.6 & \textbf{$~~$2.8} & $~~$4.0 & $~~$5.7 & \textbf{$~~$0.8} & $~~$4.6 & $~~$5.7 & \textbf{$~~$1.2} \\
                    0 & 53.8 & \textbf{44.0} & 56.3 & 41.9 & \textbf{39.9} & 42.4 & 35.9 & \textbf{28.1} & 32.2 & 18.7 & 16.3 & \textbf{16.0} \\
                    \bottomrule
                \end{tabular} 
            }
            \caption{
                Appropriateness ratios of TF-IDF, KEA and TopicRank on each specific domain
                \label{tab:appropriateness_manual_evaluation}
            }
        \end{table*}
        
        Table~\ref{tab:silence_manual_evaluation} shows the ratios of silence scores per each method per specific domain of our dataset.
        To judge if a method outperforms others, we look for a lowest ratio of reference keyphrases with score of 2, a lowest ratio of reference keyphrases with score 1 and a higher ratio of keyphrases with score 0.
        This new aspect for the evaluation of keyphrases is interesting because it compares the methods regarding the importance of information held by the extracted keyphrases.
        Once again, the finding vary according to whether the evaluation is automatic or manual.
        In the future, it would be interesting to see new automatic evaluation measurement techniques that could assess whether a keyphrase extraction method outputs the most important keyphrases first when given ordered reference keyphrase to analyse.
        \begin{table*}[t]
            \resizebox{\linewidth}{!}{
                \begin{tabular}{l|ccc|ccc|ccc|ccc}
                    \toprule
                    \multirow{2}{*}{\textbf{Score}} & \multicolumn{3}{c|}{\textbf{Linguistics}} & \multicolumn{3}{c|}{\textbf{Information Science}} & \multicolumn{3}{c|}{\textbf{Archaeology}} & \multicolumn{3}{c}{\textbf{Chemistry}} \\
                    \cline{2-13}
                    & TF-IDF & KEA & TopicRank & TF-IDF & KEA & TopicRank & TF-IDF & KEA & TopicRank & TF-IDF & KEA & TopicRank \\
                    \hline
                    2 & 20.1 & \textbf{16.2} & 16.8 & 25.5 & 22.0 & \textbf{21.6} & 38.2 & 33.0 & \textbf{32.8} & 22.0 & \textbf{17.1} & 19.2 \\
                    1 & 48.5 & \textbf{45.3} & 48.3 & 25.8 & 25.8 & \textbf{25.3} & 23.3 & 23.2 & \textbf{22.7} & \textbf{32.0} & 32.2 & 32.2 \\
                    0 & 31.4 & \textbf{38.5} & 35.0 & 48.7 & 52.2 & \textbf{53.1} & 38.5 & 43.9 & \textbf{44.5} & 46.0 & \textbf{50.7} & 48.6 \\
                    \bottomrule
                \end{tabular} 
            }
            \caption{
                Silence ratios of TF-IDF, KEA and TopicRank on each specific domain
                \label{tab:silence_manual_evaluation}
            }
        \end{table*}

%    \subsection{Correlation Analysis}
%    \label{subsec:correlation_analysis}
%        \TODO{show the correlation of Precicision, Recall, F-measure and R-precision with the human evaluation}
%        \TODO{discuss the correlation results}
%        \begin{table*}
%            \centering
%            \begin{tabular}{l|cccc}
%                \toprule
%                \textbf{Corpus} & \textbf{P} & \textbf{R} & \textbf{F1} & \textbf{\newcite{kim2010rprecision}} \\
%                \hline
%                Linguistics & 0.000 & 0.000 & 0.000 & 0.000 \\
%                Information Science & 0.000 & 0.000 & 0.000 & 0.000 \\
%                Archaeology & 0.000 & 0.000 & 0.000 & 0.000 \\
%                Chemestry & 0.000 & 0.000 & 0.000 & 0.000 \\
%                \bottomrule
%            \end{tabular}
%            
%            \caption{
%                Correlation of four automatic evaluation measures to human evaluation \TODO{...}
%                \label{tab:evaluation_correlations}}
%        \end{table*}

    \section{Conclusion et perspectives}
\label{sec:conclusion_et_perspectives}
  Dans ce travail, nous proposons une méthode à base de graphe pour l'extraction
  non supervisée de termes-clés. Cette méthode groupe les termes-clés candidats
  en sujets, détermine quels sont ceux les plus importants, puis extrait le
  terme-clé candidat qui représente le mieux chacun des sujets les plus
  importants. Cette nouvelle méthode offre plusieurs avantages vis-à-vis des
  précédentes à base de graphe. Le groupement des termes-clés potentiels en
  sujets distincts permet de rassembler des indices utiles auparavant éparpillés
  et le choix d'un seul terme-clé pour représenter un sujet important permet
  d'extraire un ensemble de termes-clés non redondants ( pour $k$ termes-clés
  extraits, exactement $k$ sujets sont couverts). Enfin, le graphe est complet
  et ne requiert plus le paramétrage d'une fenêtre de cooccurrences,
  contrairement aux autres méthodes à base de graphe.

  Les bons résultats de notre méthode montrent la pertinence d'un groupement en
  sujets des candidats pour ensuite les ordonner. Les expériences
  supplémentaires montrent aussi qu'il est encore possible d'améliorer notre
  méthode en proposant une nouvelle stratégie de sélection du terme-clé candidat
  le plus représentatif d'un sujet (pour un gain maximum allant de 4,2 à 15
  points de f-score).

  Nous avons aussi effectué une analyse d'erreurs à partir de laquelle trois
  perspectives de travaux futurs émergent~:

  Nous avons pour objectif d'améliorer la sélection des termes-clés candidats.
  Aussi, des méthodes empruntées à d'autres domaines du TAL peuvent être
  appliquées. Il semble, par exemple, pertinent d'évaluer l'apport des méthodes
  d'extraction terminologiques~\cite{castellvi2001automatictermdetection} pour
  la sélection des termes-clés candidats.
  
  Nous envisageons également d'améliorer le groupement en sujets,
  car celui-ci est très naïf et ne tient compte ni de la synonymie, ni de
  l'ambiguïté des mots. De plus, l'usage du
  radical~\cite{porter1980suffixstripping} des mots n'est pas sans introduire du
  bruit lié à certains faux positifs (p.~ex. \og{}\underline{empir}e\fg{} et
  \og{}\underline{empir}ique\fg{}). L'ajout de connaissances concernant les
  synonymes permettrait de créer des sujets plus complets et une étape de
  désambiguïsation éviterait un groupement systématique des termes-clés
  candidats ayant un ou plusieurs mots en commun. Nous envisageons aussi de
  remplacer la racinisation de \newcite{porter1980suffixstripping} par une
  méthode de lemmatisation. D'un point de vue plus technique, il faudrait
  explorer différentes méthodes de groupement, dont le groupement spectral
  (\textit{spectral clustering}) qui, dans d'autres travaux portant sur
  l'extraction automatique de termes-clés~\cite{liu2009keycluster}, montre de
  meilleures performances que le groupement hiérarchique agglomératif.

  Enfin, une étude détaillée des caractéristiques des termes-clés pourrait
  orienter notre travail vers des critères plus efficaces pour la définition
  d'une stratégie \og{}optimale\fg{} de sélection du terme-clé le plus
  représentatif d'un sujet. Un apprentissage supervisé à partir de certains
  critères est aussi envisagé, au même titre que l'usage de méthodes
  d'optimisation, telles que celle utilisée par
  \newcite{ding2011binaryintegerprogramming} dans leur méthode d'extraction
  automatique de termes-clés.



    \section{Acknowledgments}
    This work was partially supported by the French National Research Agency (TermITH project --- ANR--12--CORD--0029) and by the French National Center for Scientific Research (GOLEM project --- grant of CNRS PEPS FaSciDo 2015, \url{http://boudinfl.github.io/GOLEM/}).

    \bibliographystyle{lrec2006}
    \bibliography{biblio}
\end{document}
