\section{Extraction de termes-clés}
  \begin{frame}{Extraction de termes-clés}
    \framesubtitle{Chaîne de traitements}

    \tikzstyle{io}=[
      ellipse,
      minimum width=5cm,
      minimum height=2cm,
      fill=green!20,
      draw=green!33,
      transform shape,
      font={\Large\bfseries}
    ]
    \tikzstyle{component}=[
      text centered,
      thick,
      rectangle,
      minimum width=12.5cm,
      minimum height=2.5cm,
      fill=cyan!20,
      draw=cyan!33,
      transform shape,
      font={\Large\bfseries}
    ]

    \begin{center}
      \begin{tikzpicture}[thin,
                          align=center,
                          scale=.4,
                          node distance=2cm,
                          every node/.style={text centered, transform shape}]
        \node[io] (document) {document};
        \node[component] (preprocessing) [right=of document] {Prétraitement};
        \node[component] (candidate_extraction) [below=of preprocessing] {Sélection des candidats};
        \node[component] (candidate_classification_and_ranking) [below=of candidate_extraction] {
          \begin{tabular}{r|l}
            Classification & des candidats\\
            Ordonnancement & \\
          \end{tabular}
        };
        \node[component] (keyphrase_selection) [below=of candidate_classification_and_ranking] {Sélection de termes-clés};
        \node[io] (keyphrases) [right=of keyphrase_selection] {termes-clés};

        \path[->, thick] (document) edge (preprocessing);
        \path[->, thick] (preprocessing) edge (candidate_extraction);
        \path[->, thick] (candidate_extraction) edge (candidate_classification_and_ranking);
        \path[->, thick] (candidate_classification_and_ranking) edge (keyphrase_selection);
        \path[->, thick] (keyphrase_selection) edge (keyphrases);

        \visible<2->{
            \node[draw=red, dashed, yshift=-1cm, minimum width=13cm, minimum
          height=7.5cm, label={[xshift=-5.75cm]\Large\textbf{\textcolor{red}{c\oe{}ur}}}] (core) at (candidate_extraction.south) {};
          \visible<3->{
            \coordinate[xshift=3.5em, yshift=1em] (classification) at (candidate_classification_and_ranking.west);
            \coordinate[xshift=-6em, yshift=5em] (supervised_coordinates) at (candidate_classification_and_ranking.west);
            \coordinate[xshift=.5em, yshift=-1em] (ranking) at (candidate_classification_and_ranking.west);
            \coordinate[xshift=-6em, yshift=-5em] (unsupervised_coordinates) at (candidate_classification_and_ranking.west);

            \node (supervised) at (supervised_coordinates) {\Large\textbf{supervisé}};
            \node (unsupervised) at (unsupervised_coordinates) {\Large\textbf{non-supervisé}};

            \path[->] (supervised) edge (classification);
            \path[->] (unsupervised) edge (ranking);
          }
          \visible<4->{
            \node[draw, ellipse] (unsupervised) at (unsupervised_coordinates) {\Large\textbf{non-supervisé}};
          }
        }
      \end{tikzpicture}
    \end{center}
  \end{frame}

  \begin{frame}{Extraction de termes-clés}
    \framesubtitle{Prétraitement}

    \begin{enumerate}
      \item<1->{Segmentation en phrases}
      \begin{itemize}
        \item{\textit{PunktSentenceTokenzer} de la librairie \textsc{Nltk}}
        \item{Modèle appri à partir d'articles dans Le Monde}
      \end{itemize}
      \item<2->{Segmentation des phrases en mots}
      \begin{itemize}
        \item{Outil Bonsai}
        \item{Segmentateur à base d'expressions régulières}
      \end{itemize}
      \item<3->{Étiquetage grammatical des mots}
      \begin{itemize}
        \item{Outil MElt}
        \item{Modèle appri à partir du French TreeBank}
      \end{itemize}
    \end{enumerate}
  \end{frame}

  \begin{frame}{Extraction de termes-clés}
    \framesubtitle{Sélection des candidats}

  \end{frame}

  \begin{frame}{Extraction de termes-clés}
    \framesubtitle{Ordonnancement des candidats}

  \end{frame}

