\section{Expériences}
\label{sec:experiences}
  Dans cette section, nous présentons les expériences menées dans le but
  d'observer l'échelle de difficulté pour l'extraction automatique de
  termes-clés en domaines de spécialité à partir des méthodes TF-IDF et
  TopicRank et en fonction des candidats qui sont sélectionnés~:
  $\{1..3\}$-grammes filtrés, plus longues séquences de noms et d'adjectifs et
  candidats termes.

  \subsection{Mesure d'évaluation}
  \label{subsec:mesure_d_evaluation}
    Afin de mesurer l'échelle de difficulté pour l'extraction automatique de
    termes-clés en domaines de spécialité, nous utilisons la MAP (\textit{Mean
    Average Precision}), qui mesure la capacité d'une méthode à ordonner
    correctement les termes-clés de référence parmi tous les termes-clés
    candidats, c'est-à-dire à extraire en premier
    des candidats qui sont présents dans la liste des termes-clés de référence
    (cf. équation~\ref{equ:map}). Alors qu'il est plus courant d'utiliser la
    précision, le rappel et la f-mesures pour comparer les méthodes entre elles,
    notre choix se porte sur la MAP à cause du nombre variable de termes-clés de
    référence assignés aux documents par discipline (de 8,0 en
    linguistique à 16,6 en archéologie). La MAP étant appliquée à tous les
    candidats ordonnés et non pas à un sous ensemble (p.~ex. les 10 premiers,
    pour la précision, le rappel et la f-mesure), il ne peut y avoir de biais
    lorsque nous comparons l'extraction de termes-clés entre deux disciplines.
    \begin{align}
      \text{MAP} &= \frac{1}{\|\text{DOCUMENTS}\|} \sum_{d \in \text{DOCUMENTS}} \frac{\mathlarger\sum_{t_i \in \text{extraction}(d)~\cap~\text{référence}(d)} \text{précision}@i}{\|\text{référence}(d)\|} \label{equ:map}
    \end{align}
    \begin{center}
      \begin{tabular}{@{}lcl@{}}
        \multicolumn{2}{l}{où~:}\\
        & \multicolumn{2}{@{~~~-~~~}l}{extraction$(d)$ founie l'ensemble ordonné
          des termes-clés candidats $t_i$ de rang $i$ pour le document $d$,}\\
        & \multicolumn{2}{@{~~~-~~~}l}{référence$(d)$ fournie l'ensemble des
          termes-clés de référence du document $d$,}\\
        & \multicolumn{2}{@{~~~-~~~}l}{précision$@i$ représente la précision de
          l'extraction calculée au rang $i$,}\\
        & \multicolumn{2}{@{~~~-~~~}l}{DOCUMENTS est l'ensemble des documents de
          la collection pour laquelle les termes-clés sont extraits.}\\
      \end{tabular}
    \end{center}
    
    En accord avec l'évaluation menée dans les travaux précédents, nous
    considérons correcte l'extraction d'une variante flexionnelle d'un terme-clé
    de référence~\cite{kim2010semeval}. Les opérations de comparaison entre les
    termes-clés de référence et les termes-clés extraits sont donc effectuées à
    partir de la racine des mots qui les composent en utilisant la méthode de
    racinisation de \newcite{porter1980suffixstripping}.

  \subsection{Résultats}
  \label{subsec:resultats}
    La figure~\ref{fig:resultats} montre la performance des méthodes
    d'extraction automatique de termes-clés lorsque les candidats sélectionnés
    sont soit les $\{1..3\}$-grammes filtrés, soit les plus longues séquences de
    noms et d'adjectifs, soit tous les candidats termes extraits par TermSuite
    (sans filtrage). Notre hypothèse de départ selon laquelle la tâche
    d'extraction de termes-clés présente un degré de difficulté différent selon
    la discipline scientifique se vérifie. L'archéologie est la discipline pour
    laquelle la tâche d'extraction automatique de termes-clés est la moins
    difficile, la chimie étant la discipline la plus difficile, précédée par la
    psychologie, les sciences de l'information et la linguistique. Quelle que
    soit la discipline traitée, nous pouvons aussi observer la faible
    performance des méthodes d'extraction de termes-clés (cf. exemple
    figure~\ref{fig:exemple_extraction}). Ceci peut s'expliquer par le faible
    rappel maximum pouvant être atteint, ainsi que par l'évaluation stricte qui
    n'accepte pas les correspondances partielles (p.~ex. \og{}articles\fg{} et
    \og{}articles de recherche\fg{} qui dans le contexte de la notice de la
    figure~\ref{fig:exemple_extraction} représentent le même concept).

    Globalement, les meilleurs résultats sont obtenus avec la méthode TF-IDF. De
    plus, bien que dans le meilleur cas elle soit compétitive avec TF-IDF, la
    méthode TopicRank n'est pas stable. Lorsque les $\{1..3\}$-grammes
    sont utilisés comme candidats nous observons une forte dégradation des
    résultats de TopicRank, alors que la dégradation des résultats de TF-IDF est
    plus modérée. Cette différence de comportement face à un ensemble de
    termes-clés candidats de mauvaise qualité s'explique par le fait que le
    groupement en sujets de TopicRank n'est pas adapté pour de tels candidats et
    aussi parce que TF-IDF tire profit de la spécificité des mots (IDF), lui
    permettant, contrairement à TopicRank, de ne pas attribuer un fort poids aux
    candidats erronés tels que \og{}d'\fg{} (cf.
    figure~\ref{fig:exemple_extraction}). En ce qui concerne les résultats
    obtenus avec les deux autres méthodes de sélection des termes-clés
    candidats, les performances sont meilleures avec les plus longues séquences
    de noms et d'adjectifs. La différence de performance observée avec ces deux
    méthodes de sélection est principalement liée à la richesse des patrons
    grammaticaux utilisés par TermSuite. En effet, ses patrons grammaticaux
    contenant des déterminants et des prépositions ne reflètent qu'une infime
    quantité de termes-clés de référence (3,5~\%) et ont donc pour effet
    d'ajouter plus de bruit que de candidats positifs.
    \begin{figure}
      \centering
      \subfigure[$\{1..3\}$-grammes]{
        \begin{tikzpicture}%[scale=.75]
          \pgfkeys{/pgf/number format/.cd, use comma, fixed, fixed zerofill, precision=3}
          \begin{axis}[axis lines=left,
                       symbolic x coords={Archéologie, Linguistique, Sciences de l'information, Psychologie, Chimie},
                       xtick=data,
                       enlarge x limits=0.125,
                       x=.1\linewidth,
                       xticklabel style={anchor=east, xshift=.5em, yshift=-.25em, rotate=22.5},
                       nodes near coords,
                       nodes near coords align={vertical},
                       every node near coord/.append style={font=\scriptsize},
                       ytick={0, 0.100, 0.200, 0.300, 0.400, 0.500},
                       y=\linewidth,
                       ymin=0,
                       ymax=0.22,
                       ybar=10pt,
                       ylabel=MAP,
                       ylabel style={at={(ticklabel* cs:1)},
                                     anchor=south,
                                     rotate=270}]
            \addplot[black!66,
                     pattern=north east lines,
                     pattern color=black!40] coordinates{
              (Archéologie, 0.156)
              (Linguistique, 0.098)
              (Sciences de l'information, 0.087)
              (Psychologie, 0.065)
              (Chimie, 0.060)
            };
            \addplot[black!66,
                     pattern=north west lines,
                     pattern color=black!66] coordinates{
              (Archéologie, 0.035)
              (Linguistique, 0.039)
              (Sciences de l'information, 0.035)
              (Psychologie, 0.027)
              (Chimie, 0.021)
            };
            \legend{TF-IDF, TopicRank}
          \end{axis}
        \end{tikzpicture}
      }
      \subfigure[\texttt{/(NOM | ADJ)+/}]{
        \begin{tikzpicture}%[scale=.75]
          \pgfkeys{/pgf/number format/.cd, use comma, fixed, fixed zerofill, precision=3}
          \begin{axis}[axis lines=left,
                       symbolic x coords={Archéologie, Linguistique, Sciences de l'information, Psychologie, Chimie},
                       xtick=data,
                       enlarge x limits=0.125,
                       x=.1\linewidth,
                       xticklabel style={anchor=east, xshift=.5em, yshift=-.25em, rotate=22.5},
                       nodes near coords,
                       nodes near coords align={vertical},
                       every node near coord/.append style={font=\scriptsize},
                       ytick={0, 0.100, 0.200, 0.300, 0.400, 0.500},
                       y=\linewidth,
                       ymin=0,
                       ymax=0.22,
                       ybar=10pt,
                       ylabel=MAP,
                       ylabel style={at={(ticklabel* cs:1)},
                                     anchor=south,
                                     rotate=270}]
            \addplot[black!66,
                     pattern=north east lines,
                     pattern color=black!40] coordinates{
              (Archéologie, 0.180)
              (Linguistique, 0.103)
              (Sciences de l'information, 0.100)
              (Psychologie, 0.075)
              (Chimie, 0.071)
            };
            \addplot[black!66,
                     pattern=north west lines,
                     pattern color=black!66] coordinates{
              (Archéologie, 0.148)
              (Linguistique, 0.098)
              (Sciences de l'information, 0.094)
              (Psychologie, 0.073)
              (Chimie, 0.063)
            };
            \legend{TF-IDF, TopicRank}
          \end{axis}
        \end{tikzpicture}
      }
      \subfigure[Candidats termes]{
        \begin{tikzpicture}%[scale=.75]
          \pgfkeys{/pgf/number format/.cd, use comma, fixed, fixed zerofill, precision=3}
          \begin{axis}[axis lines=left,
                       symbolic x coords={Archéologie, Linguistique, Sciences de l'information, Psychologie, Chimie},
                       xtick=data,
                       enlarge x limits=0.125,
                       x=.1\linewidth,
                       xticklabel style={anchor=east, xshift=.5em, yshift=-.25em, rotate=22.5},
                       nodes near coords,
                       nodes near coords align={vertical},
                       every node near coord/.append style={font=\scriptsize},
                       ytick={0, 0.100, 0.200, 0.300, 0.400, 0.500},
                       y=\linewidth,
                       ymin=0,
                       ymax=0.22,
                       ybar=10pt,
                       ylabel=MAP,
                       ylabel style={at={(ticklabel* cs:1)},
                                     anchor=south,
                                     rotate=270}]
            \addplot[black!66,
                     pattern=north east lines,
                     pattern color=black!40] coordinates{
              (Archéologie, 0.165)
              (Linguistique, 0.107)
              (Sciences de l'information, 0.093)
              (Psychologie, 0.068)
              (Chimie, 0.070)
            };
            \addplot[black!66,
                     pattern=north west lines,
                     pattern color=black!66] coordinates{
              (Archéologie, 0.166)
              (Linguistique, 0.105)
              (Sciences de l'information, 0.078)
              (Psychologie, 0.072)
              (Chimie, 0.058)
            };
            \legend{TF-IDF, TopicRank}
          \end{axis}
        \end{tikzpicture}
      }
      \caption{Performances des méthodes d'extraction de termes-clés en domaines
               de spécialité à partir de différents type de candidats.
               \label{fig:resultats}}
    \end{figure}

    \begin{figure}
      \begin{minipage}{\linewidth}
        \centering
        % Linguistique_08-0265302_TEI_final.xml
        \framebox[\linewidth]{
          \parbox{.99\linewidth}{
            \textbf{Termes techniques et marqueurs d'argumentation : pour
            débusquer l'argumentation cachée dans}
            \hfill\underline{\textit{Linguistique}}\\
            \textbf{les articles de
            recherche}\\

            \vspace{-0.5em}
            Les articles de recherche présentent les résultats d'une expérience
            qui modifie l'état de la connaissance dans le domaine concerné. Le
            lecteur néophyte a tendance à considérer qu'il s'agit d'une simple
            description et à passer à côté de l'argumentation au cours de laquelle
            le scientifique cherche à convaincre ses pairs de l'innovation et de
            l'originalité présentées dans l'article et du bien-fondé de sa
            démarche tout en respectant la tradition scientifique dans laquelle il
            s'insère. Ces propriétés spécifiques du discours scientifique peuvent
            s'avérer un obstacle supplémentaire à la compréhension, surtout
            lorsqu'il s'agit d'un article en langue étrangère. C'est pourquoi il
            peut être utile d'incorporer dans l'enseignement des langues de
            spécialité une sensibilisation aux marqueurs linguistiques
            (terminologiques et argumentatifs), qui permettent de dépister le
            développement de cette rhétorique. Les auteurs s'appuient sur deux
            articles dans le domaine de la microbiologie.\\

            \vspace{-0.5em}
            \textbf{Termes-clés de référence~:} Langue scientifique$^*$,
            \underline{argumentation$^*$},  \underline{rhétorique$^*$}, 
            \underline{langue de spécialité$^*$}, \underline{enseignement des
            langues$^*$}, linguistique appliquée$^*$,  \underline{discours
            scientifique$^*$},  \underline{article de recherche}.\\

            \vspace{-0.5em}
            \textbf{Termes-clés extraits~:\\}
            \begin{tabular}{lll}
              \multicolumn{3}{c}{$\{1..3\}$-grammes}\\
              \hline~\vspace{-0.75em}\\
              \multicolumn{3}{p{.975\linewidth}}{\textbf{TF-IDF~:}
                \textbf{Argumentation},
                scientifique,
                articles,
                d' argumentation,
                l' argumentation,
                tradition scientifique,
                \textbf{discours scientifique},
                marqueurs.
              }\\~\vspace{-0.75em}\\
              \multicolumn{3}{p{.975\linewidth}}{\textbf{TopicRank~:}
                Articles,
                d',
                qu' il s',
                débusquer l' argumentation,
                \textbf{articles de recherche},
                scientifique,
                s' agit d',
                marqueurs d' argumentation.
              }\\~\vspace{-0.75em}\\
              \multicolumn{3}{c}{\texttt{/(NOM | ADJ)+/}}\\
              \hline~\vspace{-0.75em}\\
              \multicolumn{3}{p{.975\linewidth}}{\textbf{TF-IDF~:}
                \textbf{Argumentation},
                scientifique,
                articles,
                tradition scientifique,
                \textbf{discours scientifique},
                marqueurs,
                microbiologie,
                domaine.
              }\\~\vspace{-0.75em}\\
              \multicolumn{3}{p{.975\linewidth}}{\textbf{TopicRank~:}
                Article,
                \textbf{argumentation},
                recherche,
                marqueurs,
                domaine,
                langue étrangère,
                scientifique,
                résultats.
              }\\~\vspace{-0.75em}\\
              \multicolumn{3}{c}{Candidats termes}\\
              \hline~\vspace{-0.75em}\\
              \multicolumn{3}{p{.975\linewidth}}{\textbf{TF-IDF~:}
                \textbf{Argumentation},
                scientifique,
                tradition scientifique,
                \textbf{discours scientifique},
                marqueurs,
                microbiologie,
                néophyte,
                marqueurs d' argumentation.
              }\\~\vspace{-0.75em}\\
              \multicolumn{3}{p{.975\linewidth}}{\textbf{TopicRank~:}
                \textbf{Argumentation},
                marqueurs,
                \textbf{articles de recherche},
                scientifique,
                techniques,
                termes,
                article,
                langue.
              }\\
            \end{tabular}
          }
        }
      \end{minipage}
      \caption{Exemple d'extraction automatique de (huit) termes-clés à partir
               de la notice de linguistique présentée dans la
               figure~\ref{fig:exemple_notice_inist}. Les termes-clés de
               référence soulignés représentent les concepts pouvant être
               extraits depuis le titre ou le résumé de la notice. Les
               termes-clés de référence marqués d'une $^*$ font partie des
               termes-clés contrôlés. Les termes-clés extraits mis en gras sont
               les termes-clés correctement extraits. Conformément au processus
               d'évaluation standard pour les méthodes d'extraction automatique
               de termes-clés, les variantes flexionnelles sont jugées correctes
               (p.~ex. \og{}langue\underline{s} de spécialité\fg{} peut être
               extrait à la place de \og{}langue de spécialité\fg{}).
               \label{fig:exemple_extraction}}
    \end{figure}

