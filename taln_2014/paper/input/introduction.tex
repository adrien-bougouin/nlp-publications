\section{Introduction}
\label{sec:introduction}
  Un terme-clé est un mot ou une expression polylexicale qui représente un
  concept important d'un document auquel il est associé. En pratique, plusieurs
  termes-clés représentant des concepts différents sont associés à un même
  document. Ils forment alors un ensemble à partir duquel il est
  possible de caractériser, synthétiser, le contenu du document. Du
  fait de cette capacité de synthèse, les termes-clés sont utilisés dans de
  nombreuses applications telles que le résumé
  automatique~\cite{avanzo2005keyphrase}, la classification de
  documents~\cite{han2007webdocumentclustering} ou l'indexation
  automatique~\cite{medelyan2008smalltrainingset}. Cependant, tous les documents
  ne sont pas accompagnés de termes-clés et leur extraction manuelle est une
  tâche coûteuse et chronophage.
  Pour pallier ce problème, de plus en plus
  de chercheurs s'intéressent à l'extraction automatique de termes-clés, en
  témoignent les récentes campagnes
  d'évaluation~\cite{paroubek2012deft,kim2010semeval}, ainsi que les nombreux
  travaux sur ce sujet~\cite{hasan2014state_of_the_art}.

  L'extraction automatique de termes-clés consiste à extraire les unités
  textuelles les plus importantes d'un document, celles qui permettent d'en
  résumer le contenu. Parmi les méthodes d'extraction
  automatique de termes-clés existantes, nous distinguons deux catégories~: les
  méthodes supervisées et les méthodes non-supervisées. Dans le cadre supervisé,
  la tâche d'extraction de termes-clés est considérée comme une tâche de
  classification~\cite{witten1999kea} où il s'agit d'attribuer la classe
  \og{}\textit{terme-clé}\fg{} ou \og{}\textit{non terme-clé}\fg{} à des
  termes-clés candidats extraits du document. Une collection de documents
  annotés en termes-clés est nécessaire pour l'apprentissage d'un modèle de
  classification reposant sur divers traits tels que la fréquence du terme-clé
  candidat ou sa position dans le document. Dans le cadre non-supervisé, les
  méthodes attribuent un score d'importance aux candidats selon divers
  indicateurs tels que leur degré de spécificité~\cite{paukkeri2010likey} ou les relations de coocurrence que leurs
  mots entretiennent~\cite{mihalcea2004textrank}. Les méthodes supervisées sont
  plus performantes que les méthodes non-supervisées, mais leur besoin en
  données d'apprentissage annotées et leur dépendance vis-à-vis du domaine de
  ces données d'apprentissage poussent les
  chercheurs à s'intéresser aux méthodes non-supervisées.

  Dans cet article, nous nous plaçons dans le contexte de l'extraction
  non-supervisée de termes-clés à partir de documents de nature scientifique.
  \TODO{Hypothèse}\TODO{Objectif}\TODO{INTÉRÊTS (montrer que les méthodes
  actuelle sans ressources externes ne sont pas suffisantes, \dots)}
  Au moyen de cinq corpus disciplinaires
  (section~\ref{sec:presentation_des_donnees}), nous utilisons
  différentes méthodes d'extraction automatique de termes-clés
  (section~\ref{sec:extraction_automatique_de_termes_cles}) et observons leur
  performance en domaine de spécialité pour déduire l'échelle de difficulté
  disciplinaire (section~\ref{sec:experiences}). Enfin, nous proposons une
  analyse des résultats et déterminons les facteurs qui influent sur cette
  difficulté (section~\ref{sec:discussion}).

