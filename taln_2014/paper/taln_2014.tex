\documentclass[10pt,a4paper,twoside]{article}

\usepackage{times}
\usepackage[utf8]{inputenc}
\usepackage[T1]{fontenc}
\usepackage{graphicx}

% faire les \usepackage dont vous avez besoin AVANT le \usepackage{jeptaln2012} 
% add the \usepackage for you packages BEFORE the \usepackage{jeptaln2012}
\usepackage{ifthen}
\usepackage{color}
\usepackage{tikz}
\usepackage{pgfplots}
\usepackage{booktabs}
\usepackage{multirow}
\usepackage{lipsum}
\usepackage{subfigure}
\usepackage[normalem]{ulem} % normalem pour ne pas souligner les et al. des citations
\usepackage[inline]{enumitem}

\usepackage{taln2014}
\usepackage[frenchb]{babel}

\pgfplotsset{compat=1.8}
\usetikzlibrary{calc, positioning, topaths, shapes, arrows, patterns}

\newcommand\newcite[2][]{\ifthenelse{\equal{#1}{}}{\shortcite{#2}}{\shortcite[#1]{#2}}}
\newcommand\TODO[1]{\textcolor{red}{[TODO #1]}}
%\renewcommand\TODO[1]{}
%\renewcommand\lipsum[1][]{}

\title{Influence des domaines de spécialité dans l'extraction de termes-clés}

\author{Adrien Bougouin \quad Florian Boudin \quad Béatrice Daille\\
        LINA -- UMR CNRS 6241, 2 rue de la Houssinière 44322 Nantes Cedex 3, France\\ 
        prenom.nom@univ-nantes.fr}

% Titre qui apparait en en-tête (1 ligne maxi)
\fancyhead[CO]{Influence des domaines de spécialité dans l'extraction de termes-clés}
% Auteurs qui apparaissent en en-tête (1 ligne maxi)
\fancyhead[CE]{Adrien Bougouin, Florian Boudin, Béatrice Daille}

\begin{document}
  \maketitle

  \resume{
    Les termes-clés sont les mots ou les expressions polylexicales qui
    représentent le contenu principal d'un document. Ils sont utiles pour
    diverses applications telles que l'indexation automatique ou le résumé
    automatique, mais ne sont pas toujours disponibles. De ce fait, nous nous
    intéressons à la tâche d'extraction automatique de termes-clés et, plus
    particulièrement, à la difficulté de cette tâche lors du traitement de
    documents appartenant à certaines disciplines scientifiques. Au moyen de
    cinq corpus représentant cinq disciplines différentes (Archéologie, Sciences
    de l'Information, Linguistique, Psychologie, Chimie), nous déduisons une
    échelle de difficulté disciplinaire et analysons les facteurs qui influent
    sur cette difficulté.
  }

  \abstract{
    Keyphrases are single or multi-word expressions that represent the main
    content of a document. Keyphrases are useful in many applications such as
    document indexing or text summarization, which are very useful for
    researchers. However, most documents are not provided with keyphrases. To
    tackle this problem, researchers propose methods to automatically extract
    keyphrases from documents of various nature. In this paper, we focus on the
    difficulty of the automatic keyphrase extraction from scientific papers from
    various areas. Using five corpora representing five areas (Archaeology,
    Information Sciences, Linguistic, Psychology and Chemistry), we observe the
    difficulty scale and analyze factors inducing a higher or a lower
    difficulty.
  }

  \motsClefs{Extraction de termes-clés, articles scientifiques, domaines de
  spécialité, méthodes non-supervisées}{Keyphrase extraction, scientific papers,
  specific domain, unsupervised methods}

  \section{Introduction}
\label{sec:section}
    Keyphrases are words or phrases that represent the main content of a document.
    Similar to an abstract, keyphrases give a synoptic picture of what is important in the document.
    Disimilar to an abstract, keyphrases are small grain units and are useful resources for many Natural Language Processing tasks: document clustering~\cite{han2007webdocumentclustering}, information retrieval~\cite{medelyan2008smalltrainingset}, document summarization~\cite{litvak2008graphbased}, etc.
    However, documents do not always contain keyphrases.
    As the daily flow of new documents grows, manually annotating documents has become impractical.
    Hence automatic keyphrase extraction recently attracts a lot of attention and many different methods are proposed~\cite{hasan2014state_of_the_art}.

    Automatic keyphrase extraction is the task of detecting important words or phrases within a document.
    Generally speaking, we divide keyphrase extraction methods into two categories: supervised and unsupervised.
    Supervised methods treat keyphrase extraction as a binary classification task, e.g.~\cite{witten1999kea}.
    Conversely, unsupervised methods usually rank keyphrase candidates by importance and select the top-ranked ones as keyphrases, e.g.~\cite{mihalcea2004textrank}.

    Although they tackle the keyphrase extraction problem differently, both supervised and unsupervised methods rely on a candidate selection step.
    Keyphrase candidate selection identifies words or phrases consistent with human-assigned keyphrase properties.
    %Although keyphrase candidate selection starts to draw attention~\cite{wang2014keyphraseextractionpreprocessing}, keyphrase extraction methods use simple heuristics: selection of n-grams, sequences of nouns and adjectives, etc.
    However, current selection methods use simple heuristics~\cite{wang2014keyphraseextractionpreprocessing}: candidates are n-grams or sequences of nouns and adjectives.
    %This work infers linguistic properties from human-assigned keyphrases and demonstrates their applicability on keyphrase candidate selection.
    This work proposes rules based on a comprehensive analysis of modifiers within human-assigned keyphrases.
    We demonstrate their applicability on keyphrase candidate selection.
    
    This paper is organized as follows.
    Section~\ref{sec:keyphrase_properties} presents an analysis of human-assigned keyphrases.
    Section~\ref{sec:candidate_selection} describes common keyphrase candidate selection methods followed by a description of our method in Section~\ref{sec:proposed_candidate_selection_method}. Finally, Section~\ref{sec:experiments} presents the expriments and Section~\ref{sec:conclusion} concludes our work.

  %\section{État de l'art}
\label{sec:etat_de_l_art}
  \begin{itemize}
    \item{extraction de termes-clés candidats}
    \begin{itemize}
      \item{n-grammes}
      \item{np-chunks}
      \item{patrons gramaticaux}
      \item{pourquoi pas les termes ?}
    \end{itemize}
    \item{classification/ordonnancement des termes-clés candidats}
    \begin{itemize}
      \item{méthodes supervisees}
      \item{méthodes non-supervisées}
      \item[]{~\ \ \ - fréquence et spécificité (tf-idf et likey)}
      \item[]{~\ \ \ - regroupement (keycluster)}
      \item[]{~\ \ \ - ordonnancement à base de graphe (textrank, singlerank, expandrank, etc.)}
    \end{itemize}
  \end{itemize}


  \section{Collections de données}
\label{sec:presentation_des_donnees}
  Nous disposons de cinq collections de notices bibliographiques (Archéologie,
  Sciences de l'Information, Linguistique, Psychologie et Chimie) fournies par
  l'Institut de l’Information Scientifique et
  Technique\footnote{\url{http://www.inist.fr}} (INIST). Chacune de ces notices
  INIST contient le titre et le résumé d'un article, ainsi que ses termes-clés
  associés. Ces derniers sont obtenus semi-automatiquement %~:
  %
  à partir des textes intégraux (non disponibles pour nos travaux) et à partir
  de ressources disciplinaires, telles qu'une terminologie ou des spécifications
  précises quant aux types d'informations que les termes-clés doivent
  représenter (e.g.~lieu, période et autre, en Archéologie).
  %des indexeurs
  %professionnels vérifient, corrigent et complètent les sorties d'un système
  %dont les entrées sont un document et un (ou plusieurs) référentiel(s)
  %terminologique(s). Un référentiel terminologique est une liste disciplinaire
  %de termes associés manuellement à des déclencheurs, i.e. des unités textuelles
  %qui, lorsqu'elles sont présentes dans le document, impliquent l'usage du terme
  %associé en tant que terme-clé. En addition, les indexeurs disposent de règles
  %d'indexation (non formatées) qui précisent quels types d'informations doivent
  %être présents dans l'ensemble de termes-clés (e.g. en Archéologie il est
  %important de connaître la période et la localisation de ce qui fait l'objet de
  %l'article).

  Le corpus d'\textbf{Archéologie} est composé de 718 notices INIST. Celles-ci
  représentent des articles parus entre 2001 et 2012 dans 22 revues différentes
  (\textit{Paléo}, \textit{Le bulletin de la Société préhistorique française},
  etc.).

  Le corpus de \textbf{Sciences de l'Information} contient 706 notices INIST
  d'articles publiés entre 2001 et 2012 dans six revues différentes
  (\textit{Documentaliste -- Sciences de l'information}, \textit{Document
  numérique}, etc.).

  Le corpus de \textbf{Linguistique} est constitué de 716 notices INIST
  d'articles parus entre 2000 à 2012 dans 12 revues différentes
  (\textit{Linx -- Revue des linguistes de l'Université Paris Ouest Nanterre La
  Défense}, \textit{Travaux de linguistique}, etc.).

  Le corpus de \textbf{Psychologie} contient 720 notices INIST d'articles
  publiés entre 2001 et 2012 dans sept revues différentes
  (\textit{Enfance}, \textit{Revue internationale de psychologie et de gestion
  des comportements organisationnels}, etc.).

  Le corpus de \textbf{Chimie} est composé de 782 notices INIST d'articles
  publiés entre 1983 et 2012 dans quatre revues (\textit{Comptes Rendus de
  l'Académie des Sciences}, \textit{Comptes Rendus Chimie}, etc.).

  Le tableau~\ref{tab:statistiques_des_corpus} présente les caractéristiques des
  cinq collections de données présentées ci-dessus. Nous observons tout d'abord
  une différence concernant la taille des documents. Les notices archéologiques
  ont plus de contenu que les autres notices et les notices de Chimie en ont
  moins. Ceci est dû au fait que les notices archéologiques font l'objet d'un
  effort de présentation du contexte historique auquel s'intéresse l'article,
  tandis que les notices de Chimie, représentant en grande partie des comptes
  rendus, ne donnent parfois que le nom de l'expérience réalisée et les noms
  des éléments chimiques qui entrent en jeu. %Nous remarquons aussi qu'en
  %fonction des collections, le nombre moyen de termes-clés assignés aux
  %documents varie de 5,8 termes-clés en Sciences de l'Information à 17,7 en
  %Archéologie. Cette variation est due à différents facteurs %~:
  %
  %dont les ressources disciplinaires.
  %
  %\begin{itemize}
  %  \item{taille des notices~: \TODO{plus de contenu = potentiellement plus de
  %        termes-clés}~;}
  %  \item{disponibilité du contenu de l'article~: les indexeurs ont parfois
  %        accès aux articles intégraux pour compléter l'ensemble de termes-clés
  %        extrait automatiquement~;}
  %  \item{référentiel terminologique~: plus un référentiel est précis, plus il
  %        contient de termes et de déclencheurs, alors plus le nombre de
  %        termes-clés extraits peut être important~;}
  %  \item{règles d'indexation~: plus il y a de types d'information nécessaires,
  %        alors plus il doit y avoir de termes-clés à extraire.}
  %\end{itemize}
  La longueur des termes-clés varie aussi selon les disciplines. Il est fréquent
  que les termes-clés en Archéologie ne soient que des mots, en général des
  entités nommées (49,6\% des termes-clés de références présents dans les
  notices contiennent des noms propres) de type \textit{période}
  (\og{}Paléolithique\fg{}, \og{}Mésolithique\fg{}, \og{}Néolithique\fg{}, etc.)
  ou \textit{lieu} (\og{}Asie\fg{}, \og{}Europe\fg{}, \og{}France\fg{}, etc.),
  tandis que les termes-clés en Chimie sont principalement des composés
  syntagmatiques, du fait de la spécialisation systématique de certains termes
  tels que \og{}composé\fg{}, qui peut être \og{}organique\fg{},
  \og{}aliphatique\fg{}, \og{}éthylénique\fg{}, etc. Enfin, nous observons
  d'importantes différences entre les proportions de termes-clés ne pouvant être
  extraits (sous une quelconque forme fléchie). Cette différence, due à l'usage
  de ressources externes lors de l'indexation, a une influence sur les résultats
  des méthodes qui extraient uniquement des termes-clés à partir d'unités
  textuelles présentes dans les documents.
  \begin{table}
    \centering
    \begin{tabular}{@{~}r|ccccc@{~}}
      \toprule
        & & \textbf{Sciences} & & &\\
        \textbf{Statistique} & \textbf{Archéologie} & \textbf{de} & \textbf{Linguistique} & \textbf{Psychologie} & \textbf{Chimie}\\
        & & \textbf{l'Information} & & &\\
      \hline
        Documents & 718 & 706 & 716 & 720 & 782\\
        Mots/doc. & 219,1 & 119,7 & 156,4 & 185,8 & 104,9\\
        Termes-clés/doc. & 17,7 & 5,8 & 8,0 & 11,0 & 12,9\\
        Mots/terme-clé & 1,3 & 1,7 & 1,7 & 1,6 & 2,2\\
        Termes-clés non extractibles & 35,7\% & 78,1\% & 58,8\% & 72,4\% & 59,8\%\\
        Termes-clés (extractibles) avec Np & 49,6\% & 18,7\% & 9,8\% & 12,5\% & 9,2\%\\
      \bottomrule
    \end{tabular}
    \caption{Caractéristiques des corpus disciplinaires. Le pourcentage de
             termes-clés non extractibles correspond au pourcentage de
             termes-clés qui ne sont pas présents (sous une quelconque forme
             fléchie) dans le document. Le pourcentage de termes-clés avec Np
             correspond au pourcentage de termes-clés de références qui
             contiennent un mot étiqueté Np (nom propre) par notre outils
             d'étiquetage morphosyntaxique.
             \label{tab:statistiques_des_corpus}}
  \end{table}


  \section{Extraction automatique de termes-clés}
\label{sec:extraction_automatique_de_termes_cles}
  \subsection{Extraction des termes-clés candidats}
  \label{subsec:extraction_de_termes_cles_candidats}
    \TODO{Tout revoir.}
    Après la préparation des données, l'étape à ne pas négliger est celle
    de l'extraction des candidats. Dans notre cas, seules les unités textuelles
    présentes dans le document peuvent être extraites en tant que termes-clés.
    Cependant, utiliser toutes les séquences de mots possibles ensemble de
    termes-clés candidats n'est pas la meilleure stratégie. En effet, plus il y
    a de candidats, alors plus il peut être difficile d'extraire les termes-clés
    et plus le temps de calcul est important. De ce fait, nous répétons nos
    expériences avec deux méthodes d'extraction de candidats différentes, l'une
    extrayant des tri-grammes et l'autre des termes. En adéquation avec les
    travaux précédents~\cite{witten1999kea}, les tri-grammes sont filtrés avec
    une liste de mots outils, qui regroupe les mots fonctionnels de la langue
    (conjonctions, prépositions, etc.) et les mots courants (e.g.
    \og{}beaucoup\fg{}, \og{}près\fg{}, etc.). Quand aux termes, ceux-ci sont
    extraits avec l'outil d'extraction terminologique mono- et multilingue
    TermSuite~\cite{rocheteau2011termsuite}. Une terminologie par discipline est
    préalablement extraite des corpus (32~119 termes en Archéologie, 16~557
    termes en Sciences de l'Information, 21~330 termes en linguistique, 24~680
    termes en Psychologie et 21~020 termes en chimie) afin de n'extraire des
    documents que les unités textuelles appartenant à la terminologie de la
    discipline du document.

  \subsection{Extraction de termes-clés}
  \label{subsec:extraction_de_termes_cles}
    \TODO{Tout revoir.}
    Afin d'avoir un meilleur aperçu de la difficulté d'extraction de termes-clés
    selon les disciplines, nous utilisons deux méthodes non-supervisées
    employant des techniques différentes, une méthode reposant sur la
    pondération TF-IDF~\cite{jones1972tfidf} et la méthode
    TopicRank~\cite{bougouin2013topicrank}.

    Le principe de la méthode utilisant la pondération TF-IDF consiste à
    extraite en tant que termes-clés les candidats contenant les mots les plus
    importants (fort poids TF-IDF). Un mot est considéré important s'il est
    fréquent dans le document et s'il est spécifique à celui-ci. La spécificité
    est déterminée à partir tous les documents de la collection. Un mot est
    considéré comme spécifique lorsqu'il apparaît dans très peu de documents.

    La méthode TopicRank extrait les termes-clés d'un document à partir d'une
    représentation sous forme de graphe de celui-ci. Tout d'abord, les
    termes-clés candidats sont groupés par sujets, puis ces sujets sont utilisés
    pour construire un graphe complet dans lequel ils représentent chacun un
    n\oe{}ud. L'algorithme d'ordonnancement
    TextRank~\cite{mihalcea2004textrank} est ensuite appliqué afin d'obtenir un
    score d'importance pour chaque n\oe{}ud du graphe. Enfin, les $k$ sujets les
    plus importants, selon TextRank, sont sélectionnés et, pour chacun d'eux, le
    candidat le plus représentatif est extrait en tant que terme-clé. Dans la
    méthode originale, le groupement en sujet est effectué par similarité
    lexicale. Lorsque les termes-clés candidats sont les termes extraits par
    TermSuite, nous utilisons le groupement terme/variantes de TermSuite à la
    place de la similarité lexicale.


  \section{Expériences}
  \begin{frame}[allowframebreaks]{Expériences}
    \framesubtitle{}
  \end{frame}


  \section{Discussion}
\label{sec:discussion}
  Dans cette section, nous revenons sur les résultats présentés dans la
  section~\ref{sec:experiences} et pointons, pour les différentes disciplines,
  les variations qui, selon nous, influent sur la difficulté de la tâche
  d'extraction de termes-clés en domaines de spécialité. À partir des résultats
  obtenus, nous déduisons l'échelle de difficulté suivante (de la discipline la
  plus difficile à la plus facile)~:
  \begin{figure}[h!]
    \centering
    \begin{tikzpicture}[thin,
                        align=center,
                        scale=1.25,
                        every node/.style={text centered, transform shape}]
      \coordinate (start) at (4.5, 0);
      \coordinate (c) at (5.7, 0);
      \coordinate (p) at (6.3, 0);
      \coordinate (s) at (8.1, 0);
      \coordinate (l) at (9.2, 0);
      \coordinate (a) at (14.2, 0);
      \coordinate (end) at (15.4, 0);
      \node (chimie) at (5.7, -0.3) {\small Chimie};
      \node (psychologie) at (6.3, 0.3) {\small Psychologie};
      \node (sciences_de_l_information) at (8.1, -0.3) {\small Sciences de l'information};
      \node (linguistique) at (9.2, 0.3) {\small Linguistique};
      \node (archeologie) at (14.2, -0.3) {\small Archéologie};
      \draw[-|, thick] (start) node[xshift=-1.75em] (dificile) {\small Difficile} -- (c);
      \draw[-|, thick] (c) -- (p);
      \draw[-|, thick] (p) -- (s);
      \draw[-|, thick] (s) -- (l);
      \draw[-|, thick] (l) -- (a);
      \draw[->, thick] (a) -- (end) node[xshift=1.25em] (facile) {\small Facile};
    \end{tikzpicture}
    \caption{
             \label{fig:}}
  \end{figure}
  Selon cette échelle de difficulté, ainsi que selon nos observations du contenu
  des notices, nous définissons trois catégories pour lesquelles la difficulté
  n'est pas la même~:
  \begin{enumerate}
    \item{Travaux expérimentaux (Chimie)}
    \item{Travaux analytiques (Psychologie, Linguistique et Sciences de
          l'Information)}
    \item{Travaux pratiques, i.e.~fondés sur des faits non sujets à subjectivité
          (Archéologie)}
  \end{enumerate}
  \TODO{discuter plus ces catégories}

  Dans un premier temps, nous constatons que l'usage d'une pondération fondée
  sur la spécificité d'un mot vis-à-vis du document
  peut améliorer\TODO{stabiliser?} la performance des méthodes d'extraction de termes-clés. Nous
  en déduisons que la nature linguistique des termes utilisés dans une
  discipline est un facteur influant sur la difficulté de l'extraction des
  termes-clés pour cette même discipline. Ainsi, une forte tendance à l'usage de
  composés syntagmatiques constitués de mots généraux dans la discipline, tels
  que \og{}réaction\fg{} qui est présent dans le terme-clé \og{}réaction
  topotactique\fg{} en Chimie, augmente la difficulté de l'extraction des
  termes-clés.

  Dans un second temps, nous constatons que la capacité à créer des liens entre
  différentes unités textuelles peut aider lors de l'extraction des termes-clés.
  Après observation du contenu des notices, nous remarquons que l'organisation
  du discours du résumé dans les différentes disciplines est un second facteur
  influant sur la difficulté de la tâche d'extraction de termes-clés. Pour
  chaque discipline, le lecteur visé n'est pas le même et le résumé est donc
  organisé différemment. Dans le cas de documents se basant sur des faits
  concrets, tels que les documents d'Archéologie, le lecteur (archéologue ou
  non) a besoin d'une définition du contexte et des relations entre les faits
  donnés. Un document insistant sur différents éléments importants et créant des
  liens entre ces éléments est plus aisé à traiter qu'un document se reposant
  sur un acquis supposé (non explicité) du lecteur. Une observation similaire
  peut-être faite pour les travaux analytiques, où les hypothèses sont
  clairement explicitées. Au contraire, les documents au sujet de travaux
  expérimentaux sont très techniques et se reposent sur un acquis supposé du
  lecteur. En Chimie, les notices sont très souvent énumératives et dépourvues
  de détails explicatifs, superflus pour un initié. Dans ce cas, moins de liens
  sont établis entre les termes-clés candidats, et la tâche d'extraction
  automatique de termes-clés est plus difficile.


  \section{Conclusion et perspectives}
\label{sec:conclusion_et_perspectives}
  Dans ce travail, nous proposons une méthode à base de graphe pour l'extraction
  non supervisée de termes-clés. Cette méthode groupe les termes-clés candidats
  en sujets, détermine quels sont ceux les plus importants, puis extrait le
  terme-clé candidat qui représente le mieux chacun des sujets les plus
  importants. Cette nouvelle méthode offre plusieurs avantages vis-à-vis des
  précédentes à base de graphe. Le groupement des termes-clés potentiels en
  sujets distincts permet de rassembler des indices utiles auparavant éparpillés
  et le choix d'un seul terme-clé pour représenter un sujet important permet
  d'extraire un ensemble de termes-clés non redondants ( pour $k$ termes-clés
  extraits, exactement $k$ sujets sont couverts). Enfin, le graphe est complet
  et ne requiert plus le paramétrage d'une fenêtre de cooccurrences,
  contrairement aux autres méthodes à base de graphe.

  Les bons résultats de notre méthode montrent la pertinence d'un groupement en
  sujets des candidats pour ensuite les ordonner. Les expériences
  supplémentaires montrent aussi qu'il est encore possible d'améliorer notre
  méthode en proposant une nouvelle stratégie de sélection du terme-clé candidat
  le plus représentatif d'un sujet (pour un gain maximum allant de 4,2 à 15
  points de f-score).

  Nous avons aussi effectué une analyse d'erreurs à partir de laquelle trois
  perspectives de travaux futurs émergent~:

  Nous avons pour objectif d'améliorer la sélection des termes-clés candidats.
  Aussi, des méthodes empruntées à d'autres domaines du TAL peuvent être
  appliquées. Il semble, par exemple, pertinent d'évaluer l'apport des méthodes
  d'extraction terminologiques~\cite{castellvi2001automatictermdetection} pour
  la sélection des termes-clés candidats.
  
  Nous envisageons également d'améliorer le groupement en sujets,
  car celui-ci est très naïf et ne tient compte ni de la synonymie, ni de
  l'ambiguïté des mots. De plus, l'usage du
  radical~\cite{porter1980suffixstripping} des mots n'est pas sans introduire du
  bruit lié à certains faux positifs (p.~ex. \og{}\underline{empir}e\fg{} et
  \og{}\underline{empir}ique\fg{}). L'ajout de connaissances concernant les
  synonymes permettrait de créer des sujets plus complets et une étape de
  désambiguïsation éviterait un groupement systématique des termes-clés
  candidats ayant un ou plusieurs mots en commun. Nous envisageons aussi de
  remplacer la racinisation de \newcite{porter1980suffixstripping} par une
  méthode de lemmatisation. D'un point de vue plus technique, il faudrait
  explorer différentes méthodes de groupement, dont le groupement spectral
  (\textit{spectral clustering}) qui, dans d'autres travaux portant sur
  l'extraction automatique de termes-clés~\cite{liu2009keycluster}, montre de
  meilleures performances que le groupement hiérarchique agglomératif.

  Enfin, une étude détaillée des caractéristiques des termes-clés pourrait
  orienter notre travail vers des critères plus efficaces pour la définition
  d'une stratégie \og{}optimale\fg{} de sélection du terme-clé le plus
  représentatif d'un sujet. Un apprentissage supervisé à partir de certains
  critères est aussi envisagé, au même titre que l'usage de méthodes
  d'optimisation, telles que celle utilisée par
  \newcite{ding2011binaryintegerprogramming} dans leur méthode d'extraction
  automatique de termes-clés.



  \section*{Remerciements}
    Ce travail a bénéficié d'une aide de l'Agence Nationale de la Recherche
    portant la référence (ANR-12-CORD-0029).

  \bibliographystyle{taln2002}
  \bibliography{../../biblio}
\end{document}

