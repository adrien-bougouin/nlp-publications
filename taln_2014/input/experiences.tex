\section{Expériences}
\label{sec:experiences}
  Dans cette section, nous présentons les expériences menées dans le but
  d'observer l'échelle de difficulté pour l'extraction automatique de
  termes-clés en domaine de spécialité. Deux méthodes non-supervisées
  d'extraction automatique de termes-clés sont sélectionnées pour extraire les
  termes-clés puis leur performance est évaluée pour chacune des cinq
  collections présentées dans la section~\ref{sec:presentation_des_donnees}.

  \subsection{Mesure d'évaluation}
  \label{subsec:mesure_d_evaluation}
    La performance des méthodes d'extraction automatique de termes-clés est
    exprimée avec la F1-mesure. Cette mesure est un compromis entre la précision
    et le rappel, qui mesurent la capacité d'une méthode à, respectivement,
    minimiser le nombre de faux positifs et maximiser le nombre de vrais
    positifs. Dans notre évaluation, nous concidérons correcte l'extraction
    d'une variante flexionnelle d'un terme-clé de référence. Les opérations de
    comparaison entre les termes-clés de référence et les termes-clés extraits
    sont donc effectuées à partir de la racine, calculée automatiquement, des
    mots qui les composent.

  \subsection{Préparation des données}
  \label{subsec:preparation_des_donnees}
    Les documents des collections de données utilisées subissent les mêmes
    prétraitements. Ils sont tout d'abord segmentés en phrases, puis en mots et
    enfin étiquetés en parties du discours. La segmentation en mots est
    effectuée avec l'outil Bonsai, du Bonsai PCFG-LA
    parser\footnote{\url{http://alpage.inria.fr/statgram/frdep/fr_stat_dep_parsing.html}}
    et l'étiquetage en parties du discours est réalisé avec
    MElt~\cite{denis2009melt}. Tous ces outils sont utilisés sans modification
    de leurs paramètres par défaut.

  \subsection{Configuration des méthodes}
  \label{subsec:configuration_des_methodes}
    \TODO{Paramètrage de TermSuite}

    \TODO{Groupement en sujet de TopicRank}

    \TODO{Paramétrage de NEMESIS ?}

    \TODO{Extraction et filtrage de n-grammes}

  \subsection{Résultats}
  \label{subsec:resultats}
    La figure~\ref{fig:resultats} montre la performance des méthodes
    d'extraction de termes-clés lorsqu'elles extraient 10 termes-clés à partir
    des tri-grammes ou des termes des documents. Moins la F1-mesure est élevée,
    plus nous supposons que l'extraction de termes-clés est difficile.
    \begin{figure}
      \centering
      \subfigure[Tri-grammes]{
        \begin{tikzpicture}[scale=.75]
          \begin{axis}[axis lines=left,
                       symbolic x coords={Archéologie, Sciences de l'Information, Linguistique, Psychologie, Chimie},
                       xtick=data,
                       enlarge x limits=0.125,
                       x=.1\linewidth,
                       xticklabel style={anchor=east, xshift=.5em, yshift=-.25em, rotate=22.5},
                       nodes near coords,
                       nodes near coords align={vertical},
                       every node near coord/.append style={font=\scriptsize},
                       ytick={0, 10, 20, 30, 40, 50},
                       y=0.01\linewidth,
                       ymin=0,
                       ymax=25,
                       ybar=7.5pt,
                       ylabel=F1-mesure (\%),
                       ylabel style={at={(ticklabel* cs:1)},
                                     anchor=south,
                                     rotate=270}]
            \addplot[black!66,
                     pattern=north east lines,
                     pattern color=black!40] coordinates{
              (Archéologie, 17.8)
              (Sciences de l'Information, 10.2)
              (Linguistique, 11.2)
              (Psychologie, 9.5)
              (Chimie, 8.0)
            };
            \addplot[black!66,
                     pattern=north west lines,
                     pattern color=black!66] coordinates{
              (Archéologie, 6.9)
              (Sciences de l'Information, 5.1)
              (Linguistique, 6.4)
              (Psychologie, 5.3)
              (Chimie, 3.7)
            };
            \legend{TF-IDF, TopicRank}
          \end{axis}
        \end{tikzpicture}
      }
      \subfigure[Termes]{
        \begin{tikzpicture}[scale=.75]
          \begin{axis}[axis lines=left,
                       symbolic x coords={Archéologie, Sciences de l'Information, Linguistique, Psychologie, Chimie},
                       xtick=data,
                       enlarge x limits=0.125,
                       x=.1\linewidth,
                       xticklabel style={anchor=east, xshift=.5em, yshift=-.25em, rotate=22.5},
                       nodes near coords,
                       nodes near coords align={vertical},
                       every node near coord/.append style={font=\scriptsize},
                       ytick={0, 10, 20, 30, 40, 50},
                       y=0.01\linewidth,
                       ymin=0,
                       ymax=25,
                       ybar=7.5pt,
                       ylabel=F1-mesure (\%),
                       ylabel style={at={(ticklabel* cs:1)},
                                     anchor=south,
                                     rotate=270}]
            \addplot[black!66,
                     pattern=north east lines,
                     pattern color=black!40] coordinates{
              (Archéologie, 19.2)
              (Sciences de l'Information, 11.1)
              (Linguistique, 12.7)
              (Psychologie, 10.3)
              (Chimie, 9.7)
            };
            \addplot[black!66,
                     pattern=north west lines,
                     pattern color=black!66] coordinates{
              (Archéologie, 20.9)
              (Sciences de l'Information, 9.9)
              (Linguistique, 13.7)
              (Psychologie, 11.5)
              (Chimie, 8.8)
            };
            \legend{TF-IDF, TopicRank}
          \end{axis}
        \end{tikzpicture}
      }
      \caption{Résultats de l'extraction de 10 termes-clés.
               \label{fig:resultats}}
    \end{figure}

    Deux disciplines se distinguent sur l'échelle de difficulté, l'Archéologie
    se veut être la discipline la plus facile pour la tâche d'extraction de
    termes-clés, en opposition avec la Chimie qui, elle, est la plus difficile.
    Quant aux Sciences de l'Information, à la Linguistique et à la Psychologie,
    celles-ci ont sensiblement le même degré de difficulté. Les deux méthodes
    d'extraction de termes-clés se distinguent aussi par leur comportement. La
    méthode TF-IDF est plus stable que TopicRank lorsque l'on compare les
    résultats obtenus à partir des tri-grammes et des termes. En effet, en
    comparaison avec l'ensemble de termes, l'ensemble de tri-grammes contient
    plus de candidats non pertinants qui dégradent la qualité des groupements en
    sujets de TopicRank, tandis que la notion de spécificité incorporée dans
    TF-IDF permet de placer ces candidats en queue du classement et donc de
    diminuer leur effet négatif.

