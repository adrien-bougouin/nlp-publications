\section{Discussion}
\label{sec:discussion}
  Dans cette section, nous revenons sur les résultats présentés dans la
  section~\ref{sec:experiences} et pointons, pour les différentes disciplines,
  les variations qui, selon nous, influent sur la difficulté de la tâche
  d'extraction de termes-clés en domaines de spécialité. À partir des résultats
  obtenus, nous déduisons l'échelle de difficulté suivante (de la discipline la
  plus difficile à la plus facile)~:
  \begin{enumerate}
    \item{Chimie}
    \item{Psychologie}
    \item{Linguistique}
    \item{Sciences de l'Information}
    \item{Archéologie}
  \end{enumerate}
  Selon cette échelle de difficulté, ainsi que selon nos observations du contenu
  des notices, nous définissons trois catégories pour lesquelles la difficulté
  n'est pas la même~:
  \begin{enumerate}
    \item{Travaux expérimentaux (Chimie)}
    \item{Travaux analytiques (Psychologie, Linguistique et Sciences de
          l'Information)}
    \item{Travaux pratiques, i.e.~fondés sur des faits non sujets à subjectivité
          (Archéologie)}
  \end{enumerate}

  Dans le cas général, nous observons que la qualité de l'ensemble de candidats
  utilisé influe sur la performance des méthodes d'extraction automatique de
  termes-clés. Cependant, l'influence est différente selon la méthode
  d'extraction de termes-clés. Nous observons que les $\{1..3\}$-grammes
  dégradent fortement la performance de TopicRank en comparaison avec la méthode
  TF-IDF. TopicRank ne tirant pas profit d'une quelconque mesure de spécificité,
  nous en déduisons que la nature spécifique des termes-clés est un facteur
  important ayant des conséquences sur la difficulté d'extraction des
  termes-clés. Selon ce facteur, les disciplines pour lesquelles les termes-clés
  sont majoritairement des uni-grammes (e.g.~Archéologie) sont moins difficiles
  à traiter que des disciplines pour lesquelles les termes-clés ne sont pas
  majoritairement des uni-grammes. Par exemple en Chimie, le mot
  \og{}réaction\fg{} n'est pas spécifique dans le terme-clé \og{}réaction
  topotactique\fg{}.

  Après observation du contenu des notices, nous remarquons un second facteur~:
  l'organisation du discours dans les différentes disciplines. Pour chaque
  discipline, le lecteur visé n'est pas le même et le discours est donc organisé
  différemment. Dans le cas de documents se basant sur des faits concrets, tels
  que les documents d'Archéologie, le lecteur (archéologue ou non) a besoin
  d'une définition du contexte et des relations entre les faits donnés. Un
  document insistant sur déférents éléments importants et créant des liens entre
  ces éléments est plus aisé à traiter qu'un document se reposant sur un acquis
  supposé du lecteur (et donc non explicité). Une observation similaire
  peut-être faite pour les travaux analytiques, où les hypothèses sont
  clairement explicitées. En contradiction, les documents au sujet de travaux
  expérimentaux sont très techniques et se reposent sur un acquis supposé du
  lecteur. En Chimie, les notices sont très souvent énumératives et dépourvues
  de détails explicatifs, superflus pour un initié. Dans ce cas, moins de liens
  sont établis entre les termes-clés candidats, et la tâche d'extraction
  automatique de termes-clés est plus difficile.

