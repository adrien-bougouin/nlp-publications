\section{Extraction automatique de termes-clés}
\label{sec:extraction_automatique_de_termes_cles}
  \subsection{Extraction des termes-clés candidats}
  \label{subsec:extraction_de_termes_cles_candidats}
    \TODO{Tout revoir.}
    Après la préparation des données, l'étape à ne pas négliger est celle
    de l'extraction des candidats. Dans notre cas, seules les unités textuelles
    présentes dans le document peuvent être extraites en tant que termes-clés.
    Cependant, utiliser toutes les séquences de mots possibles ensemble de
    termes-clés candidats n'est pas la meilleure stratégie. En effet, plus il y
    a de candidats, alors plus il peut être difficile d'extraire les termes-clés
    et plus le temps de calcul est important. De ce fait, nous répétons nos
    expériences avec deux méthodes d'extraction de candidats différentes, l'une
    extrayant des tri-grammes et l'autre des termes. En adéquation avec les
    travaux précédents~\cite{witten1999kea}, les tri-grammes sont filtrés avec
    une liste de mots outils, qui regroupe les mots fonctionnels de la langue
    (conjonctions, prépositions, etc.) et les mots courants (e.g.
    \og{}beaucoup\fg{}, \og{}près\fg{}, etc.). Quand aux termes, ceux-ci sont
    extraits avec l'outil d'extraction terminologique mono- et multilingue
    TermSuite~\cite{rocheteau2011termsuite}. Une terminologie par discipline est
    préalablement extraite des corpus (32~119 termes en Archéologie, 16~557
    termes en Sciences de l'Information, 21~330 termes en linguistique, 24~680
    termes en Psychologie et 21~020 termes en chimie) afin de n'extraire des
    documents que les unités textuelles appartenant à la terminologie de la
    discipline du document.

  \subsection{Extraction de termes-clés}
  \label{subsec:extraction_de_termes_cles}
    \TODO{Tout revoir.}
    Afin d'avoir un meilleur aperçu de la difficulté d'extraction de termes-clés
    selon les disciplines, nous utilisons deux méthodes non-supervisées
    employant des techniques différentes, une méthode reposant sur la
    pondération TF-IDF~\cite{jones1972tfidf} et la méthode
    TopicRank~\cite{bougouin2013topicrank}.

    Le principe de la méthode utilisant la pondération TF-IDF consiste à
    extraite en tant que termes-clés les candidats contenant les mots les plus
    importants (fort poids TF-IDF). Un mot est considéré important s'il est
    fréquent dans le document et s'il est spécifique à celui-ci. La spécificité
    est déterminée à partir tous les documents de la collection. Un mot est
    considéré comme spécifique lorsqu'il apparaît dans très peu de documents.

    La méthode TopicRank extrait les termes-clés d'un document à partir d'une
    représentation sous forme de graphe de celui-ci. Tout d'abord, les
    termes-clés candidats sont groupés par sujets, puis ces sujets sont utilisés
    pour construire un graphe complet dans lequel ils représentent chacun un
    n\oe{}ud. L'algorithme d'ordonnancement
    TextRank~\cite{mihalcea2004textrank} est ensuite appliqué afin d'obtenir un
    score d'importance pour chaque n\oe{}ud du graphe. Enfin, les $k$ sujets les
    plus importants, selon TextRank, sont sélectionnés et, pour chacun d'eux, le
    candidat le plus représentatif est extrait en tant que terme-clé. Dans la
    méthode originale, le groupement en sujet est effectué par similarité
    lexicale. Lorsque les termes-clés candidats sont les termes extraits par
    TermSuite, nous utilisons le groupement terme/variantes de TermSuite à la
    place de la similarité lexicale.

