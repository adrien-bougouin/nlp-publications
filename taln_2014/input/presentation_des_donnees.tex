\section{Présentation des données}
\label{sec:presentation_des_donnees}
  Pour évaluer la performance des méthodes non-supervisées état de l'art pour
  l'extraction automatique de termes-clés dans divers domaines de spécialité,
  nous disposons de cinq collections dont les documents appartiennent à des
  disciplines différentes. Quatre collections, sur les cinq, représentent des
  disciplines de SHS (Archéologie, Sciences de l'Information, Linguistique et
  Psychologie), tandis que la cinquième représente la Chimie.

  Le corpus d'\textbf{Archéologie} est composé de 718 notices bibliographiques
  (résumés) fournies par l'Institut de l’Information Scientifique et
  Technique\footnote{\url{http://www.inist.fr}} (notices INIST). Celles-ci
  représentent des articles parus entre 2001 et 2012 dans 22 revues différentes
  (\textit{Paléo}\footnote{\url{http://paleo.revues.org}}, \textit{Le bulletin
  de la Société préhistorique
  française}\footnote{\url{http://www.prehistoire.org/515_p_21855/le-bulletin-de-la-spf.html}}, etc.).

  Le corpus de \textbf{Sciences de l'Information} contient 706 notices INIST
  d'articles publiés entre 2001 et 2012 dans six revues différentes
  (\textit{Documentaliste -- Sciences de
  l'information}\footnote{\url{http://www.cairn.info/revue-documentaliste-sciences-de-l-information.htm}},
  \textit{Document
  numérique}\footnote{\url{http://dn.revuesonline.com/accueil.jsp}}, etc.).

  Le corpus de \textbf{Linguistique} est constitué de 716 notices INIST
  d'articles parus entre 2000 à 2012 dans 12 revues différentes
  (\textit{Linx -- Revue des linguistes de l'Université Paris Ouest Nanterre La
  Défense}\footnote{\url{http://linx.revues.org}}, \textit{Travaux de
  linguistique}\footnote{\url{http://www.cairn.info/revue-travaux-de-linguistique.htm}}, etc.).

  Le corpus de \textbf{Psychologie} contient 720 notices INIST d'articles
  publiés entre 2001 et 2012 dans sept revues différentes
  (\textit{Enfance}\footnote{\url{http://www.cairn.info/revue-enfance.htm}},
  \textit{Revue internationale de psychologie et de gestion des comportements
  organisationnels}\footnote{\url{http://www.cairn.info/revue-internationale-de-psychosociologie-2005-25.htm}}, etc.)

  Le corpus de \textbf{Chimie} est composé de 782 notices INIST d'articles
  publiés entre 1983 et 2012 dans quatre revues (\textit{Comptes Rendus de
  l'Académie des
  Sciences}\footnote{\url{http://www.academie-sciences.fr/activite/cr.htm}},
  \textit{Comptes Rendus
  Chimie}\footnote{\url{http://www.elsevier.com/journals/comptes-rendus-chimie/1631-0748}}, etc.).

  Les termes-clés de référence associés aux notices INIST sont obtenus
  semi-automatiquement. Des indexeurs professionnels vérifient, corrigent et
  complètent les sorties d'un système dont les entrées sont le document à
  analyser et un (ou plusieurs) référentiel(s) terminologique(s). Le référentiel
  terminologique est une liste (disciplinaire) de termes associés (manuellement)
  à des déclencheurs, les déclencheurs étant des unités textuelles qui,
  lorsqu'elles sont présente dans le document, impliquent l'usage du terme
  associé en tant que terme-clé. En addition, les indexeurs disposent de règles
  d'indexation qui précisent quels types d'informations doivent être présents
  dans les termes-clés (e.g. en Archéologie il est important de connaître la
  période et la localisation de ce qui fait l'objet de l'article).

  Le tableau~\ref{tab:statistiques_des_corpus} présente les statistiques des
  cinq collections de données présentées ci-dessus. Nous observons des
  différences de tailles, avec les notices archéologiques, qui contiennent plus
  d'informations que les autres notices, et les notices de chimie, qui en
  contiennent moins. Après observation du contenu de ces notices, nous
  remarquons que dans les notices archéologiques il y a un effort de
  présentation du contexte historique auquel s'intéresse l'article, tandis que
  dans les notices de Chimie, un grand nombre des documents représentés étant
  des comptes rendus, parfois seuls les noms des expériences réalisées et les
  noms des éléments chimiques qui entre en jeu sont donnés. Nous remarquons
  aussi qu'en fonction des collections, le nombre moyen de termes-clés assignés
  aux documents varie, allant d'environ six termes-clés en Sciences de
  l'Information à environ 18 en Archéologie. Cette variation est due à
  différents facteurs~:
  \begin{itemize}
    \item{taille des notices~: plus une notice contient d'informations, alors
          plus il peut y avoir de termes-clés extraits~;}
    \item{disponibilité du contenu de l'article~: les indexeurs ont parfois
          accès aux articles intégraux afin de recueillir plus d'informations
          utiles à l'extraction de termes-clés~;}
    \item{référentiels terminologiques~: plus un référentiel est précis, plus il
          contient de déclencheurs, alors plus le nombre de termes-clés extraits
          peut être important~;}
    \item{règles d'indexation~: plus il y a de types d'informations nécessaires,
          alors plus il doit y avoir de termes-clés à extraire.}
  \end{itemize}
  Enfin, le nombre de mots qui composent un terme-clé varie selon les
  disciplines. Ainsi, il est fréquent que les termes-clés en Archéologie soient
  des entités nommées de type \textit{période} (\og{}Paléolithique\fg{},
  \og{}Mésolithique\fg{}, \og{}Néolithique\fg{}, etc.) ou \textit{lieu}
  (\og{}Asie\fg{}, \og{}Europe\fg{}, \og{}France\fg{}, etc.) composée d'un seul
  mot, tandis que les termes-clés en Chimie sont principalement composés de
  plusieurs mots, du fait la spécialisation systématique de certains termes tels
  qu'un \og{}composé\fg{}, qui peut être \og{}organique\fg{},
  \og{}aliphatique\fg{} ou encore \og{}éthylénique\fg{}.
  \begin{table}
    \centering
    \begin{tabular}{@{~}r|ccccc@{~}}
      \toprule
        & & \textbf{Sciences} & & &\\
        \textbf{Statistique} & \textbf{Archéologie} & \textbf{de} & \textbf{Linguistique} & \textbf{Psychologie} & \textbf{Chimie}\\
        & & \textbf{l'Information} & & &\\
      \hline
        Documents & 718 & 706 & 716 & 720 & 782\\
        Mots/doc. & 219,1 & 119,7 & 156,4 & 185,8 & 104,9\\
        Termes-clés/doc. & 17,7 & 5,8 & 8,0 & 11,0 & 12,9\\
        Mots/terme-clé & 1,3 & 1,7 & 1,7 & 1,6 & 2,2\\
      \bottomrule
    \end{tabular}
    \caption{Statistiques des corpus disciplinaires.
             \label{tab:statistiques_des_corpus}}
  \end{table}

